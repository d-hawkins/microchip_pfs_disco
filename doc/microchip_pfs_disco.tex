% -----------------------------------------------------------------------------
% microchip_pfs_disco.txt
%
% 7/25/2025 D. W. Hawkins (dwh@caltech.edu)
%
% Microchip PolarFire SoC Discovery Kit Tutorial.
%
% -----------------------------------------------------------------------------
%
% -----------------------------------------------------------------------------
% Document preamble
% -----------------------------------------------------------------------------
%
\documentclass[10pt,twoside]{article}

% Math symbols
\usepackage{amsmath}
\usepackage{amssymb}

% Headers/Footers
\usepackage{fancyhdr}

% Colors
% See https://en.wikibooks.org/wiki/LaTeX/Colors
\usepackage[dvipsnames]{xcolor}

% Importing and manipulating graphics
\usepackage{graphicx}
\usepackage{subfig}
\usepackage{pdflscape}

% Misc packages
\usepackage{verbatim}
\usepackage{dcolumn}
\usepackage{ifpdf}
\usepackage{enumerate}

% PDF Bookmarks and hyperref stuff
\usepackage[
  bookmarks=true,
  bookmarksnumbered=true,
  colorlinks=true,
  filecolor=blue,
  linkcolor=blue,
  urlcolor=blue,
  hyperfootnotes=true
  citecolor=blue
]{hyperref}

% Improved citation handling
% (include after the hyperref stuff)
\usepackage{cite}

% Pretty printing code
\usepackage{listings}

% Source code listings:
% https://en.wikibooks.org/wiki/LaTeX/Source_Code_Listings
% Defining new colors:
% https://en.wikibooks.org/wiki/LaTeX/Colors
\usepackage{xcolor}
\colorlet{keyword}{blue}
%\colorlet{comment}{green!70!black}
\colorlet{comment}{OliveGreen}
\colorlet{string}{green!50!blue}
\lstdefinestyle{vhdl}{
  language         = VHDL,
  basicstyle       = \small\ttfamily,
  keywordstyle     = \color{keyword}\bfseries,
  commentstyle     = \color{comment},
  stringstyle      = \color{string},
  tabsize          = 4,
  showstringspaces = false
}
% Verilog syntax highlighting
\lstdefinestyle{verilog}{
  language         = Verilog,
  basicstyle       = \small\ttfamily,
  keywordstyle     = \color{keyword}\bfseries,
  commentstyle     = \color{comment},
  stringstyle      = \color{string},
  tabsize          = 4,
  showstringspaces = false
}
% C code syntax highlighting
\lstdefinestyle{c}{
	language         = C,
	basicstyle       = \small\ttfamily,
	keywordstyle     = \color{keyword}\ttfamily,
	commentstyle     = \color{comment},
	stringstyle      = \color{string},
	tabsize          = 4,
	showstringspaces = false
}

% -----------------------------------------------------------------------------
% Setup the margins
% -----------------------------------------------------------------------------
% Footer Template

% Set left margin - The default is 1 inch, so the following
% command sets a 1.25-inch left margin.
\setlength{\oddsidemargin}{0.25in}
\setlength{\evensidemargin}{0.25in}

% Set width of the text - What is left will be the right
% margin. In this case, right margin is
% 8.5in - 1.25in - 6in = 1.25in.
\setlength{\textwidth}{6in}

% Set top margin - The default is 1 inch, so the following
% command sets a 0.75-inch top margin.
%\setlength{\topmargin}{-0.25in}
\setlength{\topmargin}{-0.2in}

% Set height of the header
\setlength{\headheight}{0.3in}

% Set vertical distance between the header and the text
\setlength{\headsep}{0.2in}

% Set height of the text
\setlength{\textheight}{8.5in}

% Set vertical distance between the text and the
% bottom of footer
\setlength{\footskip}{0.4in}

% -----------------------------------------------------------------------------
% Allow floats to take up more space on a page.
% -----------------------------------------------------------------------------

% see page 142 of the Companion for this stuff and the
% documentation for the fancyhdr package
\renewcommand{\textfraction}{0.05}
\renewcommand{\topfraction}{0.95}
\renewcommand{\bottomfraction}{0.95}
% dont make this too small
\renewcommand{\floatpagefraction}{0.35}
\setcounter{totalnumber}{5}

% Make sure the top/bottom rules appear on the page
\renewcommand{\headrulewidth}{0.4pt}
\renewcommand{\footrulewidth}{0.4pt}

% -----------------------------------------------------------------------------
% Set up the header/footer
% -----------------------------------------------------------------------------
%
% First page
% - set the head/foot rule width to zero to hide them
\fancypagestyle{plain}{
	\renewcommand{\headrulewidth}{0pt}
	\renewcommand{\footrulewidth}{0pt}
	\fancyhf{}
	\fancyfoot{}
}

% All other pages
\lhead{PFS-DISCO Tutorial}
\chead{}
\rhead{\today}
\lfoot{}
\cfoot{}
\rfoot{\thepage}

% =============================================================================
% Document contents
% =============================================================================
%
\begin{document}
\title{Microchip PolarFire SoC Discovery Tutorial}
\author{D. W. Hawkins (dwh@caltech.edu)}
\date{\today}

% Title page
\maketitle

% Table of contents
\tableofcontents

% Switch to the fancy page style
\pagestyle{fancy}

% Start the first section on an odd page
%\cleardoublepage
\clearpage

% =============================================================================
% Main Document
% =============================================================================
%
% =============================================================================
\section{Introduction}
% =============================================================================
\label{sec:intro}

This tutorial demonstrates hardware and software development for the Microchip
PolarFire SoC
\href{https://www.microchip.com/en-us/development-tool/mpfs-disco-kit}
{Discovery Kit}.
%
This tutorial was created under Windows 10 using
\href{https://www.microchip.com/en-us/products/fpgas-and-plds/fpga-and-soc-design-tools/fpga/libero-software-later-versions}
{Libero SoC 2025.1} and
\href{https://www.microchip.com/en-us/products/fpgas-and-plds/fpga-and-soc-design-tools/soc-fpga/softconsole}
{SoftConsole 2022.2}\footnote{Latest releases on 7/27/2025.}.
%
The Microchip PolarFire SoC Discovery Kit features:
%
\begin{itemize}
%
\item \textbf{Microchip PolarFire SoC}~\cite{Microchip_PFSoC_DS_2025}
\begin{itemize}
\item Part number: MPFS095T-1FCSG325E
\item 93,516 logic elements (4-input LUTs plus register)
\item 292 Math Blocks (18-bit pre-adder, 18 $\times$ 19-bit multiply, 48-bit accumulator)
\item 308 $\times$ 20kbit LSRAM
\item 876 $\times$ 768-bit uSRAM
\end{itemize}
%
\item \textbf{Clocks}
\begin{itemize}
\item 125MHz LVDS reference clock (for the MSS)
\item 50MHz single-ended fabric clock
\item 2MHz and 160MHz internal oscillators
\end{itemize}
%
\item \textbf{Memory}
\begin{itemize}
\item 1GB 16-bit DDR4 memory
\item eMMC
\item QSPI Flash
\end{itemize}
%
\item \textbf{Interfaces}
\begin{itemize}
\item FTDI FT4232H USB interface for JTAG (Embedded FlashPro5) and 3 UARTs
\item 1Gb Ethernet (SGMII connection to VSC8221 PHY and RJ45 connector)
\item microSD card interface
\item mikroBUS connector
\item 40-pin Raspberry Pi 4 interface
\item Raspberry Pi MIPI RX connector
\item 10-pin 7-segment display connector
\item 8 LEDs (4 red and 4 yellow)
\item 8 DIP switches
\item 2 push-button switches
\end{itemize}
%
\end{itemize}
%
Figure~\ref{fig:pfs_disco_photos} shows photos of the PolarFire SoC
Discovery Kit\footnote{The images are from the
\href{https://github.com/polarfire-soc/polarfire-soc-documentation/blob/master/reference-designs-fpga-and-development-kits/mpfs-discovery-kit-embedded-software-user-guide.md}
{Discovery Kit Embedded Software User Guide} github.}.
%
The Microchip PolarFire SoC Discovery Kit contains a smaller device than the
\href{https://www.microchip.com/en-us/development-tool/mpfs-icicle-kit-es}
{Icicle Kit} (MPFS250T-FCVG484EES) or
\href{https://www.microchip.com/en-us/development-tool/MPFS250-VIDEO-KIT}
{Video Kit} (MPFS250TS-1FCG1152I), but a larger device than the
\href{https://www.beagleboard.org/boards/beaglev-fire}{BeagleV-Fire} (MPFS025T).


% -----------------------------------------------------------------------------
% PolarFire Discovery Kit images
% -----------------------------------------------------------------------------
%
% The images were cropped and scaled differently. A picture environment was
% used so that the images could be aligned.
%
\setlength{\unitlength}{1mm}
\begin{figure}[p]
  \begin{picture}(120,180)(0,0)
    \put(0, 100){\includegraphics[width=\textwidth]
    {figures/pfs_disco_top.pdf}}
    \put(3,   5){\includegraphics[width=0.9\textwidth]
    {figures/pfs_disco_bottom.pdf}}
    \put(60, 95){(a) Top view}
    \put(60,  0){(b) Bottom}
  \end{picture}
  \caption{PolarFire SoC Discovery Kit.}
  \label{fig:pfs_disco_photos}
\end{figure}

\clearpage
% =============================================================================
\section{Getting Started}
% =============================================================================
\label{sec:getting_started}

The Microchip PolarFire SoC
\href{https://www.microchip.com/en-us/development-tool/mpfs-disco-kit}
{Discovery Kit} web page has getting started resources:
%
\begin{itemize}
\item Discovery Kit Quick Start Guide~\cite{Microchip_DISCO_QS_2024}

The Quick Start Guide contains photos of the kit that identify the components
on the board (photos are of the Rev 1 design), and provide the recommended
jumper settings. The guide details how to use the factory default FIR filter
and FFT demonstration design and FIR filter Windows GUI.

\item Discovery Kit User Guide~\cite{Microchip_DISCO_UG_2025}

The User Guide contains photos of the kit that identify the components
on the board (photos are of the Rev 2 design), and provide the recommended
jumper settings. The User Guide contains details on the components and
the interfaces. Appendix B contains tables of pin assignments, but no recommended
settings, eg., I/O standard, drive strength, or slew rate.

\item Discovery Kit Schematic~\cite{Microchip_DISCO_SCH_2023}

The schematic provides critical details on bank voltages, pin assignments,
device reset generation, clocks, and interface connections.

The Microchip web site does not include the Cadence Capture design file (.DSN)
file. This file is provided with the Video Kit and is useful for automated
pin and net name checking (using Cadence Capture Tcl scripts). The design
file has been requested from Microchip.

\item FIR filter and FFT demonstration design~\cite{Microchip_AN5165_2024}

The factory default design. The design does not include details on the UART
communications protocol or a bit-true model of the signal processing logic.
The design does include a testbench, so it would be possible to reverse-engineer
the UART protocol and to develop a bit-true model.

\item Libero SoC design flow~\cite{Microchip_AN5282_2024}

Demonstrates a Libero SoC SmartDesign for an AND gate. The AND gate inputs are
connected to the two push button switches, while the output is connected to
LED1.

\textcolor{magenta}{This document does not demonstrate how to instantiate the RISC-V processors.}


\item \href{https://github.com/polarfire-soc/polarfire-soc-discovery-kit-reference-design}{Discovery Kit reference design}

\textcolor{red}{TODO:} Add a chapter exploring the designs in this repo.

This is the hardware reference design on which the bare-metal applications can run.

\item \href{https://github.com/polarfire-soc/polarfire-soc-bare-metal-examples}{Bare-metal examples}

\textcolor{red}{TODO:} Add a chapter exploring the designs in this repo.

\item Linux examples
\item Links to videos

See Appendix~\ref{sec:resources}.

\end{itemize}
%
This tutorial references these resources (and others).

\clearpage
% -----------------------------------------------------------------------------
\subsection{Factory Restore}
% -----------------------------------------------------------------------------
\label{sec:factory_restore}

The Discovery Kit FIR filter example design is programmed (or verified) as follows:
%
\begin{enumerate}
\item Download and unzip the FIR filter design~\cite{Microchip_AN5165_2024}
%
\begin{itemize}
\item For example, unzip to
\begin{verbatim}
C:\temp\mpfs_an5165_v2024p1_df
\end{verbatim}
\end{itemize}
%
\item Power the Discovery Kit using a USB-C cable
\item Start \emph{FPExpress v2025.1}
\item Select \emph{Project$\rightarrow$New Job Project}
\item \emph{Create New Job Project} dialog:
%
\begin{itemize}
\item For \emph{Import FlashPro Express job file}, browse
and select the job file, eg.,
\begin{verbatim}
C:\temp\mpfs_an5165_v2024p1_df\Programming_Job
\end{verbatim}
\item For \emph{FlashPro Express job project location}, browse
and select the directory as the job file.
\item Click \emph{OK}
\end{itemize}
%
\item \textbf{Program}
\begin{itemize}
\item The action pull-down menu defaults to \emph{PROGRAM}
\item Click the \emph{RUN} button to run the \emph{PROGRAM} action
\item Figure~\ref{fig:pfs_disco_fpexpress}(a) shows the FlashPro Express GUI after program
\end{itemize}
%
\item \textbf{Verify}
\begin{itemize}
\item Change the action pull-down menu to \emph{VERIFY}
\item Click the \emph{RUN} button to run the \emph{VERIFY} action
\item Figure~\ref{fig:pfs_disco_fpexpress}(b) shows the FlashPro Express GUI after verify
\end{itemize}
%
\item Save and exit the FlashPro Express application
\end{enumerate}
%
The Discovery Kit User Guide has details on the embedded FlashPro5
programmer (App. A~\cite{Microchip_DISCO_UG_2025}).

\vskip5mm
\begin{center}
\fbox{\parbox{0.85\linewidth}{
\textbf{\textcolor{red}{WARNING:}}\newline
The FIR filter example design uses only the FPGA fabric, so cannot
be used for RISC-V software development!}}
\end{center}

% -----------------------------------------------------------------------------
% FlashPro Express
% -----------------------------------------------------------------------------
%
\begin{figure}[p]
  \begin{center}
    \includegraphics[width=0.95\textwidth]
    {figures/pfs_disco_fpexpress_program.png}\\
    (a) PROGRAM\\
    \vskip5mm
    \includegraphics[width=0.95\textwidth]
    {figures/pfs_disco_fpexpress_verify.png}\\
    (b) VERIFY
  \end{center}
  \caption{FlashPro Express GUI after program and verify.}
  \label{fig:pfs_disco_fpexpress}
\end{figure}

\clearpage
% -----------------------------------------------------------------------------
% FIR Filter GUI screen shots
% -----------------------------------------------------------------------------
%
\begin{figure}[t]
  \begin{center}
    \includegraphics[width=0.85\textwidth]
    {figures/an5165_filter_gui_a.png}
    \vskip5mm
  \end{center}
  \caption{FIR filter GUI: filter coefficients.}
  \label{fig:an5165_filter_gui_a}
\end{figure}

% -----------------------------------------------------------------------------
\subsection{FIR Filter Demonstration}
% -----------------------------------------------------------------------------

%
\begin{enumerate}
\item Install the FIR Filter GUI
%
\begin{itemize}
\item Run the installer:
\begin{verbatim}
C:\temp\mpfs_an5165_v2024p1_df\GUI_Installer\Volume\setup.exe
\end{verbatim}
\item Install directories:
\begin{itemize}
\item GUI: \verb+C:\software\Microchip\DISCO-KIT\FIR_Filter_GUI+
\item National Instruments: \verb+C:\Program Files (x86)\National Instruments+
\end{itemize}
\item The software can be uninstalled using Windows add/remove programs
\end{itemize}
%
\item Power the Discovery Kit using a USB-C cable
%
\item Run the FIR Filter GUI
\begin{itemize}
\item \emph{Start Menu$\rightarrow$Filter\_GUI}
\item Click on the link icon and establish the connection to the kit
\item Click on \emph{Generate Filter}
\item Figure~\ref{fig:an5165_filter_gui_a} shows the filter coefficients
\item Click on \emph{Generate Signal}
\item Figure~\ref{fig:an5165_filter_gui_b} shows the input signal
\item Click on the play button
\item Figure~\ref{fig:an5165_filter_gui_c} shows the filtered output signal
\end{itemize}
\end{enumerate}
%
AN5165 contains additional screen shots and details~\cite{Microchip_AN5165_2024}.

\clearpage
% -----------------------------------------------------------------------------
% FIR Filter GUI screen shots
% -----------------------------------------------------------------------------
%
\begin{figure}[p]
  \begin{center}
    \includegraphics[width=0.85\textwidth]
    {figures/an5165_filter_gui_b.png}
    \vskip5mm
  \end{center}
  \caption{FIR filter GUI: input signal.}
  \label{fig:an5165_filter_gui_b}
\end{figure}

\begin{figure}[p]
  \begin{center}
    \includegraphics[width=0.85\textwidth]
    {figures/an5165_filter_gui_c.png}
    \vskip5mm
  \end{center}
  \caption{FIR filter GUI: output signal.}
  \label{fig:an5165_filter_gui_c}
\end{figure}

\clearpage
% -----------------------------------------------------------------------------
% Discovery Kit Reference Designs
% -----------------------------------------------------------------------------
%
\begin{table}
\caption{Discovery Kit reference hardware designs.}
\label{tab:pfs_disco_hardware_designs}
\begin{center}
\begin{tabular}{|l|p{100mm}|}
\hline
Script Argument & Description\\
\hline\hline
&\\
(no argument)   & Discovery Kit reference hardware design\\
FIR\_DEMO       & AN5165 Filter demonstration\\
BFM\_SIMULATION & SmartDesign testbench with FIC interface BFMs\\
I2C\_LOOPBACK   & I2C loopback via the fabric\\
MIV\_RV32\_CFG1 & Mi-V softcore RISC-V example 1\\
MIV\_RV32\_CFG2 & Mi-V softcore RISC-V example 2\\
MIV\_RV32\_CFG3 & Mi-V softcore RISC-V example 3\\
&\\
\hline
\end{tabular}
\end{center}
\end{table}

% -----------------------------------------------------------------------------
\subsection{Discovery Kit Reference Hardware Designs}
% -----------------------------------------------------------------------------

The Discovery Kit reference hardware designs are available on github:
%
\begin{quote}
\href{https://github.com/polarfire-soc/polarfire-soc-discovery-kit-reference-design}
{https://github.com/polarfire-soc/polarfire-soc-discovery-kit-reference-design}
\end{quote}
%
Table~\ref{tab:pfs_disco_hardware_designs} shows the different hardware designs.
The hardware designs are built using a Libero Tcl script, with the different
designs selected based on a script argument. The repository README.md
is recommended reading. The README.md contains a block diagram of the
reference hardware design, and information on building the design variants.

Section~\ref{sec:riscv_blinky} demonstrates bare-metal software running on the
reference hardware design.

\vskip10mm
\begin{center}
\fbox{\parbox{0.85\linewidth}{
\textbf{\textcolor{red}{IMPORTANT:}}\newline
The Discovery Kit must be programmed with the reference hardware design
before using the bare-metal or Linux software examples.}}
\end{center}
\vskip10mm

% -----------------------------------------------------------------------------
\subsection{Bare-Metal Software Examples}
% -----------------------------------------------------------------------------

The PolarFire SoC bare-metal software examples are available on github:
%
\begin{quote}
\href{https://github.com/polarfire-soc/polarfire-soc-bare-metal-examples}
{https://github.com/polarfire-soc/polarfire-soc-bare-metal-examples}
\end{quote}
%
The bare-metal examples were developed for the PolarFire SoC Icicle Kit.
%
The examples ported to the Discovery Kit are:
%
\begin{itemize}
\item mss-gpio: \href{https://github.com/polarfire-soc/polarfire-soc-bare-metal-examples/blob/main/driver-examples/mss/mss-gpio/mpfs-gpio-interrupt}{mpfs-gpio-interrupt}
\item mss-mmuart: \href{https://github.com/polarfire-soc/polarfire-soc-bare-metal-examples/blob/main/driver-examples/mss/mss-mmuart/mpfs-mmuart-interrupt}{mpfs-mmuart-interrupt}
\item mss-rtc: \href{https://github.com/polarfire-soc/polarfire-soc-bare-metal-examples/blob/main/driver-examples/mss/mss-rtc/mpfs-rtc-time}{mpfs-rtc-time}
\end{itemize}
%
Section~\ref{sec:riscv_blinky} demonstrates the use of the GPIO and MMUART
examples.

\clearpage
% -----------------------------------------------------------------------------
\subsection{Linux}
% -----------------------------------------------------------------------------

\noindent\textcolor{red}{TODO:}
\begin{itemize}
\item Create a micro-SD card and boot Linux.
\item Describe the boot sequence:
\begin{itemize}
\item HSS = First-stage boot-loader
\item U-Boot = Second-stage bootloader
\item Linux
\end{itemize}
\item Blink the LEDs:
\begin{itemize}
\item Using HSS commands
\item Using U-Boot commands
\item Using /dev/mem and a Linux application
\end{itemize}
\end{itemize}

% -----------------------------------------------------------------------------
\subsection{PolarFire SoC Documentation}
% -----------------------------------------------------------------------------

The PolarFire SoC github organization and documentation repository have a
lot of information:
%
\begin{itemize}
\item
\href{https://github.com/polarfire-soc}
{https://github.com/polarfire-soc}
\item
\href{https://github.com/polarfire-soc/polarfire-soc-documentation}
{https://github.com/polarfire-soc/polarfire-soc-documentation}
\end{itemize}
%
Recommended reading:
%
\begin{itemize}
\item
\href{https://github.com/polarfire-soc/polarfire-soc-documentation/tree/master/bare-metal-embedded-software}
{bare-metal-embedded-software}
\item
\href{https://github.com/polarfire-soc/polarfire-soc-documentation/blob/master/bare-metal-embedded-software/bare-metal-software-project-structure.md}
{bare-metal-software-project-structure.md}
\item
\href{https://github.com/polarfire-soc/polarfire-soc-documentation/blob/master/bare-metal-embedded-software/bare-metal-driver-user-guides}
{bare-metal-driver-user-guides}
\item
\href{https://github.com/polarfire-soc/polarfire-soc-documentation/blob/master/reference-designs-fpga-and-development-kits/mpfs-discovery-kit-embedded-software-user-guide.md}
{mpfs-discovery-kit-embedded-software-user-guide.md}
\item
\href{https://github.com/polarfire-soc/polarfire-soc-documentation/blob/master/knowledge-base/boot-modes/boot-modes-fundamentals.md}
{boot-modes-fundamentals.md}
\item
\href{https://github.com/polarfire-soc/polarfire-soc-documentation/blob/master/reference-designs-fpga-and-development-kits/updating-linux-in-mpfs-kit.md}
{updating-linux-in-mpfs-kit.md}
\end{itemize}
%
Support:
%
\begin{itemize}
\item
\href{https://github.com/orgs/polarfire-soc/discussions}
{https://github.com/orgs/polarfire-soc/discussions}
\end{itemize}


\clearpage
% =============================================================================
\section{PolarFire SoC Discovery Power, Reset, and Clocks}
% =============================================================================
\label{sec:power}

This section reviews the implementation of the PolarFire SoC Discovery
Kit power supplies, resets, and clocks. This information is specific to
the board design, and is required to define pin constraints (I/O standards).
The kit has power supply jumper options, and it is important to understand
how these jumpers need to be configured to avoid damaging the board when
connecting to external devices using the expansion connectors.

% -----------------------------------------------------------------------------
\subsection{Power Supply Jumpers}
% -----------------------------------------------------------------------------

The PolarFire SoC Discovery Kit schematic~\cite{Microchip_DISCO_SCH_2023}
contains several jumpers:
%
\begin{itemize}
%
\item \textsf{5P0V\_IN}: 5V power source select
\begin{itemize}
\item Jumper on schematic page 13
\item J47 selects between the USB power or power jack (J7)
\end{itemize}
%
\item \textsf{VDDI1\_5}: FPGA Bank 1 and 5 power
and \textsf{VDDAUX1}: FPGA VDDAUX1 power
\begin{itemize}
\item Jumpers on schematic page 11
\item J45 and J46 select between 2.5V or 3.3V (default)
\item The schematic notes are:
\begin{itemize}
\item 2.5V for MIPI and Ethernet PHY operation
\item 3.3V for RPi and MikroBus operation
\end{itemize}
\item J45 and J46 must be set to the same voltage (p8~\cite{Microchip_DISCO_UG_2025})
\item Bank 1 contains GPIO signals (p9~\cite{Microchip_DISCO_UG_2025})
\item Bank 5 contains MSS SGMII signals (p9~\cite{Microchip_DISCO_UG_2025})
\item The SGMII PHY is the Microchip (previously Vitesse) VSC8221~\cite{Microchip_VSC8221_2006}
\item Schematic pages 5 and 6 shows that \textsf{VDDI1\_5} powers the VSC8221 VDDIO pins
\item The VSC8221 data sheet indicates VDDIO can be 2.5V or 3.3V~\cite{Microchip_VSC8221_2006}
\item Schematic page 5 shows that \textsf{VDDI1\_5} powers the 125MHz oscillator
\item The 125MHz oscillator schematic symbol has a note that it supports 2.25V to 3.63V
\item The \href{https://www.microchip.com/en-us/product/at24cm01}{AT24CM01} EEPROM supports 1.7V to 5.5V
\item Schematic page 7 has the Bank 1 FPGA connections, which include:
\begin{itemize}
\item The Raspberry Pi header (RPi prefixed signals)
\item The MikroBus header (MBUS prefixed signals)
\item The Ethernet PHY  (VSC prefixed signals)
\item The Raspberry Pi MIPI connector (GPIO\_MIPI prefixed signals)
\end{itemize}
\item \textcolor{magenta}{The J45 and J46 jumpers need to be set correctly for any external device}
\end{itemize}
%
\item \textsf{VCCB}: 7-segment display power
\begin{itemize}
\item Jumper on schematic page 7
\item J49 selects between 3.3V or 5.0V
\item The dual-voltage buffer U6 (TX0108PWR) performs voltage translation
from 3.3V or 5.0V to 1.8V for the FPGA interface
\item The buffer protects the FPGA from damage
\item \textcolor{magenta}{The J49 jumper needs to be set correctly for the external 7-segment display}
\end{itemize}
%
\end{itemize}

\clearpage
% -----------------------------------------------------------------------------
% Resets
% -----------------------------------------------------------------------------
%
\begin{figure}[t]
  \begin{center}
    \includegraphics[width=\textwidth]
    {figures/pfs_resets}
  \end{center}
  \caption{PolarFire SoC resets.}
  \label{fig:pfs_resets}
\end{figure}
% -----------------------------------------------------------------------------

% -----------------------------------------------------------------------------
\subsection{Reset}
% -----------------------------------------------------------------------------

The PolarFire SoC power-on-reset sequence is described in detail in the
\emph{PolarFire Family Power-Up and Resets User Guide}~\cite{Microchip_PFSoC_PU_2025}.
%
Figure~\ref{fig:pfs_resets} shows the PolarFire SoC resets:
\begin{itemize}
\item External device reset (DEVRST\_N) pin resets the System Controller
\item The System Controller manages the resets to the processor
\item The System Controller manages the resets to the fabric
\end{itemize}
%
The next sections describe how to generate fabric resets, and how the
System Controller manages the relative timing of the
fabric reset deassertion versus processor reset deassertion.

\clearpage
% -----------------------------------------------------------------------------
% Power-on to functional timing
% -----------------------------------------------------------------------------
%
\begin{figure}[t]
  \begin{center}
    \includegraphics[width=\textwidth]
    {figures/pfs_power_on_to_functional_timing.png}
  \end{center}
  \caption{PolarFire SoC power-on to functional timing
           (per Figure 2-2, p8~\cite{Microchip_PFSoC_PU_2025}.}
  \label{fig:pfs_power_on_to_functional}
\end{figure}

% -----------------------------------------------------------------------------
\subsubsection{Fabric Power-on-Reset}
% -----------------------------------------------------------------------------

Flash-based FPGAs required a power-on-reset to ensure fabric registers are
initialized.
%
Figure~\ref{fig:pfs_power_on_to_functional} shows the PolarFire SoC resets.
%
The PolarFire SoC initialization monitor (PFSOC\_INIT\_MONITOR), visible on
the left side of the figure, provides System Controller configuration
status signals to the fabric logic.
%
Fabric logic \emph{must} use the PFSOC\_INIT\_MONITOR signals
to qualify any external reset inputs, since the I/Os are not enabled until
after the fabric logic is enabled. The initialization monitor signals are
asynchronous to any fabric clocks, so fabric resets based on the monitor
signals must be synchronized to their respective clock domains using
reset synchronizer components.
The Microchip SmartDesign examples instantiate the CORERESET\_PF reset
synchronizer component.
The \emph{PolarFire Family Power-Up and Resets User Guide} has SmartDesign
examples of PolarFire and PolarFire SoC initialization,
eg., see Figure 3-3 on page 46, and Figure 3-3 on
page 52~\cite{Microchip_PFSoC_PU_2025}.
%
The blinky LED example in Section~\ref{sec:fabric_blinky} generates the fabric
reset using the IP Catalog \emph{PolarFireSoC Initialization Monitor}
(v1.0.309) component, combinatorial logic, and a custom reset synchronizer
component.

Figure~\ref{fig:pfs_power_on_to_functional} shows the PolarFire SoC power-on
to functional timing (copied from Figure 2-2, p8~\cite{Microchip_PFSoC_PU_2025}).
The power-on-reset timing diagrams can also be found in the device
datasheet~\cite{Microchip_PFSoC_DS_2025}. The PolarFire SoC documentation
refers to the power-on-reset time as the \emph{power-up to functional time}
(PUFT).

\clearpage
% -----------------------------------------------------------------------------
% Probes
% -----------------------------------------------------------------------------
%
\begin{table}[t]
\caption{Discovery Kit Raspberry Pi header pins used to probe power-on-reset.}
\label{tab:pfs_disco_probes}
\begin{center}
\begin{tabular}{|c|l|c|c|l|l|}
\hline
RPi & Schematic        & PFSoC & Probe & Probed & Note\\
Pin & Net Name         & Pin   & Index & Signal &\\
\hline\hline
&&&&&\\
 1 & 3.3V              &       &       & Power &\\
&&&&&\\
 3 & RPi\_GPIO2\_SDA   & E18   & 0     & Logic low              & 10k pull-up\\
 5 & RPi\_GPIO3\_SCL   & F18   & 1     & DEVICE\_INIT\_DONE     & 10k pull-up\\
 7 & RPi\_GPIO4\_GCLK  & E12   & 2     & BANK\_0\_CALIB\_STATUS & \\
11 & RPi\_GPIO17\_GEN0 & G18   & 3     & Design reset (RST\_N)  & \\
&&&&&\\
39 & Ground            &       &       & Power                  &\\
&&&&&\\
\hline
\end{tabular}
\end{center}
\end{table}
% -----------------------------------------------------------------------------

The Discovery Kit FPGA fabric power-on-reset to functional timing was investigated
using the Raspberry Pi (RPi) 40-pin header and the fabric blinky LED design
(see Section~\ref{sec:fabric_blinky}).
Table~\ref{tab:pfs_disco_probes} shows the RPi header pins
probed using two oscilloscope channels.
%
Figures~\ref{fig:pfs_disco_power_on_a} and~\ref{fig:pfs_disco_power_on_b} show
the measured power-on-reset timing:
%
\begin{enumerate}
\item \textbf{Device Reset}

The Discovery kit schematic (p9~\cite{Microchip_DISCO_SCH_2023}) shows the
device reset pin (DEVRST\_N) is driven by a Microchip MCP121-315 3.3V
voltage supervisor~\cite{Microchip_MCP121_2023} which has a power-on delay of 120ms.

Figure~\ref{fig:pfs_disco_power_on_a}(a) shows the 3.3V power supply on channel 1
and an FPGA output driven low on channel 2. The pulse observed on channel 2
has a rising-edge that follows the 3.3V power supply due to the signal 10k
pull-up to 3.3V, and a falling-edge that occurs slightly after 120ms
due to the voltage monitor 120ms power-on delay time plus the fabric ready time.
Figure~\ref{fig:pfs_power_on_to_functional} shows the fabric ready delay is
the time taken until FPGA\_POR\_N deasserts.

\item \textbf{Device Initialization Done}

Figure~\ref{fig:pfs_disco_power_on_a}(b) shows the falling-edge of the low
output on channel 1 and DEVICE\_INIT\_DONE on channel 2. The channel 2
signal starts out high due to the 10k pull-up to 3.3V on the on the signal,
transitions low when the output driver is enabled, and then transitions
high when DEVICE\_INIT\_DONE asserts.
%
Figure~\ref{fig:pfs_power_on_to_functional} indicates that the time between
the deassertion of FPGA\_POR\_N and the assertion of DEVICE\_INIT\_DONE
is design dependent: for the Discovery Kit blinky LED design this time is
just under 600$\mu$s.

\item \textbf{Bank Calibration Done}

Figure~\ref{fig:pfs_disco_power_on_b}(a) shows the falling-edge of the low
output on channel 1 and BANK\_0\_CALIB\_STATUS on channel 2.
%
The Discovery Kit fabric drive LEDs (LED1 through LED7) are located on Bank 0,
so the blinky LED design uses BANK\_0\_CALIB\_STATUS as one of the design
reset sources.

\item \textbf{Fabric Logic Reset}

Figure~\ref{fig:pfs_disco_power_on_b}(b) shows the falling-edge of the low
output on channel 1 and the blinky LED reset on channel 2.
The blinky LED reset is synchronized to the 50MHz clock.
%
Figures~\ref{fig:pfs_disco_power_on_b}(a) and (b) look the same, as
BANK\_0\_CALIB\_STATUS is one of the inputs to the design reset synchronizer.

\end{enumerate}

% -----------------------------------------------------------------------------
% Power-on-reset waveforms
% -----------------------------------------------------------------------------
%
\begin{figure}[p]
  \begin{center}
    \includegraphics[width=0.8\textwidth]
    {figures/pfs_disco_power_on_0_output_low.png}\\
    (a) 3.3V and a fabric output driven low\\
    \vskip5mm
    \includegraphics[width=0.8\textwidth]
    {figures/pfs_disco_power_on_1_device_init_done.png}\\
    (b) Fabric output driven low and BANK\_0\_CALIB\_STATUS
  \end{center}
  \caption{Discovery Kit power-on-reset waveforms.}
  \label{fig:pfs_disco_power_on_a}
\end{figure}

\begin{figure}[p]
  \begin{center}
    \includegraphics[width=0.8\textwidth]
    {figures/pfs_disco_power_on_2_calib_done.png}\\
    (a) Fabric output driven low and CALIB\_DONE\\
    \vskip5mm
    \includegraphics[width=0.8\textwidth]
    {figures/pfs_disco_power_on_3_reset.png}\\
    (b) Fabric output driven low and 50MHz clock-domain active-low reset
  \end{center}
  \caption{Discovery Kit power-on-reset waveforms.}
  \label{fig:pfs_disco_power_on_b}
\end{figure}

% -----------------------------------------------------------------------------
\subsubsection{Fabric Reset Push Button}
% -----------------------------------------------------------------------------

The Discovery Kit FIR filter design uses SWITCH1 as a push button reset.
%
The top-level SmartDesign uses a debounce component
(see Figure 2-2, p5~\cite{Microchip_AN5165_2024}).
%
The PolarFire SoC Discovery Kit schematic shows that the switches have
an RC-filter with a time-constant of 1ms. The PolarFire SoC inputs can be
configured to enable a Schmitt trigger input. The combination of RC-filter
and Schmitt trigger should be sufficient, so a debounce circuit is not
required. The SmartDesign debouncer is likely something that
was inherited from the FIR filter design from a different development kit,
eg., this FIR filter design is available for
\href{https://www.microchip.com/en-us/application-notes/an4753}{RTG4},
\href{https://www.microchip.com/en-us/application-notes/DG0438}{SmartFusion2}, and
\href{https://www.microchip.com/en-us/application-notes/DG0504}{IGLOO2} kits.

The \href{https://github.com/polarfire-soc/polarfire-soc-discovery-kit-reference-design}
{Discovery Kit Reference Design} script was run with the FIR\_DEMO argument and
the I/O Editor used to view the pin constraints: the Schmitt trigger was not
enabled on the SWITCH1 input.
%
Figure~\ref{fig:pfs_disco_reset} shows the Discovery Kit reset waveforms measured
from the blinky LED example in Section~\ref{sec:fabric_blinky} with the SWITCH1
Schmitt trigger input enabled. Similar waveforms were observed with the Schmitt
trigger disabled. Conclusion: the Discovery Kit does not need switch debouncing.

% -----------------------------------------------------------------------------
% Reset push button waveforms
% -----------------------------------------------------------------------------
%
\begin{figure}[p]
  \begin{center}
    \includegraphics[width=0.8\textwidth]
    {figures/pfs_disco_reset_a.png}\\
    (a) 3.3V and reset assertion at 100ns/div\\
    \vskip5mm
    \includegraphics[width=0.8\textwidth]
    {figures/pfs_disco_reset_b.png}\\
    (b) 3.3V and reset assertion at 20ms/div
  \end{center}
  \caption{Discovery Kit push button reset waveforms.}
  \label{fig:pfs_disco_reset}
\end{figure}

\clearpage
% -----------------------------------------------------------------------------
\subsubsection{Processor Reset and Boot}
% -----------------------------------------------------------------------------


The PolarFire SoC MSS reset sequence is described in detail in the
\emph{PolarFire Family Power-Up and Resets User Guide}~\cite{Microchip_PFSoC_PU_2025}.
%
Figure~\ref{fig:pfs_mss_boot_modes}(a) shows the MSS Pre-Boot execution flow
that leads to the selection of one of four different boot modes.
%
The two boot modes used in this tutorial are:
%
\begin{itemize}
\item \textbf{Boot mode 0: Idle boot}

Figure~\ref{fig:pfs_mss_boot_modes}(b) shows the idle boot mode flow.
The PolarFire SoC MSS processors enter busy loops until the debugger
connects (p29~\cite{Microchip_PFSoC_PU_2025}).

\item \textbf{Boot mode 1: Non-secure boot}

Figure~\ref{fig:pfs_mss_boot_modes}(c) shows the non-secure boot mode flow.
The PolarFire SoC MSS processors execute from reset vectors stored in
eNVM (see Table 2-5 on p30 for the reset vector addresses~\cite{Microchip_PFSoC_PU_2025}).
\end{itemize}
%
Section~\ref{sec:riscv_blinky} exercises these two boot modes.

%------------------------------------------------------------------------------
% PolarFire SoC MSS Boot Modes
%------------------------------------------------------------------------------
%
\begin{figure}[p]
  \begin{minipage}{\textwidth}
    \begin{center}
    \includegraphics[width=\textwidth]
    {figures/pfs_mss_boot_modes.png}\\
    (a) Boot modes
    \end{center}
  \end{minipage}
  \vskip10mm
  \begin{minipage}{0.49\textwidth}
    \begin{center}
    \includegraphics[width=0.45\textwidth]
    {figures/pfs_mss_boot_mode_0.png}\\
    (b) Boot mode 0: Idle boot flow
    \end{center}
  \end{minipage}
  \hfil
  \begin{minipage}{0.49\textwidth}
    \begin{center}
    \includegraphics[width=0.45\textwidth]
    {figures/pfs_mss_boot_mode_1.png}\\
    (c) Boot mode 1: Non-secure boot flow
    \end{center}
  \end{minipage}
  \caption{PolarFire SoC MSS boot modes.}
  \label{fig:pfs_mss_boot_modes}
\end{figure}
% -----------------------------------------------------------------------------

\clearpage
% -----------------------------------------------------------------------------
% Clocks
% -----------------------------------------------------------------------------
%
\begin{figure}[t]
  \begin{center}
    \includegraphics[width=\textwidth]
    {figures/pfs_disco_mss_clocks}
  \end{center}
  \caption{PolarFire SoC Discovery Kit MSS clock configuration with 125MHz reference clock.}
  \label{fig:pfs_disco_mss_clocks}
\end{figure}
% -----------------------------------------------------------------------------

% -----------------------------------------------------------------------------
\subsection{Clocks}
% -----------------------------------------------------------------------------

The PolarFire SoC Discovery Kit has two external clock
sources (p5 and p13~\cite{Microchip_DISCO_UG_2025}):
%
\begin{itemize}
\item 125MHz MSS Reference clock (DSC1123BL5-125.0000)

The \href{https://www.microchip.com/en-us/product/DSC1123}{DSC1123} is a MEMS
oscillator with a startup time of 5ms (max).

\item 50 MHz Oscillator (DSC1001DL5-050.0000)

The \href{https://www.microchip.com/en-us/product/DSC1001}
{DSC1001} is a CMOS oscillator with a startup time of 1.3ms (max).

\end{itemize}
%
The MCP121 voltage supervisor power-on-reset time of 120ms
is sufficient for the oscillators to be stable when used by the PolarFire SoC.
%
Figure~\ref{fig:pfs_disco_mss_clocks} shows the MSS Configurator view of the
PolarFire Discovery Kit clocks using the
\href{https://github.com/polarfire-soc/polarfire-soc-discovery-kit-reference-design/blob/main/script_support/MPFS_DISCOVERY_KIT_MSS.cfg}
{reference design configuration}.
%
\begin{itemize}
\item The RISC-V processor clocks are configured for 600MHz
\item The AXI and L2 cache is configured for 300MHz
\item The AHB/APB bus configured for 150MHz
\item The DDR clock is configured for 800MHz (1600Mbps data rate)
\item The SGMII interface is configured for 625MHz (1.25Gbps serial rate)
\end{itemize}


\clearpage
% =============================================================================
\section{Example 1: Fabric-only Blinky LEDs}
% =============================================================================
\label{sec:fabric_blinky}

% -----------------------------------------------------------------------------
% Libero GUI
% -----------------------------------------------------------------------------
%
\begin{figure}[t]
  \begin{center}
    \includegraphics[width=\textwidth]
    {figures/ex1_fabric_blinky.pdf}
  \end{center}
  \caption{Fabric-only Blinky LEDs block diagram.}
  \label{fig:ex1_diagram}
\end{figure}
% -----------------------------------------------------------------------------

Figure~\ref{fig:ex1_diagram} shows a block diagram of the fabric-only Blinky
LEDs design. This section demonstrates how to use the Libero SoC GUI to manually
create a project, create physical design constraints, and timing constraints,
and then how to reproduce the design using scripts.
%
Table~\ref{tab:ex1_source} shows the source and scripts described in this
section.

% -----------------------------------------------------------------------------
% Project files
% -----------------------------------------------------------------------------
%
\begin{table}[b]
\caption{Discovery Kit fabric-only blinky LEDs project source.}
\label{tab:ex1_source}
\begin{center}
\begin{tabular}{|l|l|}
\hline
Filename & Description\\
\hline\hline
&\\
\texttt{designs/fabric\_blinky}              & Design directory\\
\quad\texttt{src/pfs\_disco.sv}              & Top-level design\\
\quad\texttt{scripts/pfs\_disco.sdc}         & Timing constraints\\
\quad\texttt{scripts/pfs\_disco\_io.pdc}     & Pin constraints\\
\quad\texttt{scripts/pfs\_disco\_fp.pdc}     & Floorplan constraints\\
\quad\texttt{scripts/pfs\_disco.ndc}         & Output register constraints\\
\quad\texttt{scripts/pfs\_disco.fdc}         & TMR constraints\\
\quad\texttt{scripts/libero.tcl}             & Libero build script\\
\quad\texttt{scripts/synplify.tcl}           & Synplify build script\\
&\\
\texttt{ip}                                    & Common IP directory\\
\quad\texttt{blinky/src/blinky.sv}             & Blinky LED design\\
\quad\texttt{cdc\_sync\_bit/src/cdc\_sync\_bit.sv}         & Synchronizer\\
\quad\texttt{pfs\_init\_monitor/src/pfs\_init\_monitor.sv} & PFS\_INIT\_MONITOR wrapper\\
&\\
\hline
\end{tabular}
\end{center}
\end{table}

\clearpage
% -----------------------------------------------------------------------------
\subsection{Libero Manual Project Creation}
% -----------------------------------------------------------------------------

Create the Libero SoC project as follows:
%
\begin{enumerate}
\item Start \emph{Libero SoC 2025.1}
\item Select \emph{Project$\rightarrow$New Project}
\item \textbf{Project details}

Configure the project name and location, eg.,
\begin{itemize}
\item Project name: \verb+pfs_disco+
\item Project location:
\verb+$TUTORIAL/designs/fabric_blinky/build+%$

where \verb+TUTORIAL+ is the path to the tutorial source, eg.,\newline
\verb+TUTORIAL = C:/github/microchip_pfs_disco+.
\item Click \emph{Next}
\end{itemize}
The Libero project is created in directory \verb+build/pfs_disco+. The scripted
flow uses the directory name \verb+build/libero+.
%
\item \textbf{Device selection}

Select the PolarFire Discovery device:
\begin{itemize}
\item Change the part filter family to PolarFireSoC
\item Change the die to MPFS095T
\item Change the package to FCSG325
\item Select part number MPFS095T-1FCSG325E (-1 speed grade)
\item Click \emph{Next}
\end{itemize}
%
\item \textbf{Device settings}
\begin{itemize}
\item The default settings are acceptable, i.e.,
\item Core Voltage: 1.0
\item Default I/O technology: LVCMOS 1.8V
\item (Checked) Reserve pins for probes
\item (Unchecked) System controller suspend mode
\item Click \emph{Next}
\end{itemize}
%
\newpage
\item \textbf{Add HDL source files}
\begin{itemize}
\item Click on the \emph{Link File} button and link the SystemVerilog files:
\begin{itemize}
\item \verb+$TUTORIAL/designs/fabric_blinky/src/pfs_disco.sv+
%$
\item \verb+$TUTORIAL/ip/blinky/src/blinky.sv+
%$
\item \verb+$TUTORIAL/ip/cdc_sync_bit/src/cdc_sync_bit.sv+
%$
\item \verb+$TUTORIAL/ip/pfs_init_monitor/src/pfs_init_monitor.sv+
%$
\end{itemize}
\item Click \emph{Next}
\item Click \emph{Finish}
\end{itemize}
%
\item To configure Libero SoC to use SystemVerilog, select
\emph{Project$\rightarrow$Project Settings}, then \emph{Design flow},
change the Verilog language radio button to SystemVerilog, click \emph{Save}, and
then \emph{Close}.
%
\item Click on the \emph{Design Hierarchy} tab in the Libero GUI.
\item Click on the \emph{Build Hierarchy} button.
\item Select the top-level of the hierarchy, \verb+pfs_disco+, right-mouse-click and
select \emph{Set As Root}.
\item Click on the \emph{Design Flow} tab in the Libero GUI.
\item Click on the \emph{Save} button.
%
\item Configure the synthesis tool using \emph{Project$\rightarrow$Tool Profiles},
and then \emph{Synthesis}.
%
\item Click on the green play button to synthesize and place-and-route the design.
\item The Libero GUI will update with green check marks next to \emph{Implement Design},
\emph{Synthesize}, and \emph{Place and Route}.
Figure~\ref{fig:pfs_disco_libero_gui_manual} shows a screen shot of the Libero SoC 2025.1 GUI.
%
\item The Tcl commands correspondng to most of the GUI actions can be written to a script using:

Select \emph{Project$\rightarrow$Export Script File}, and configure the settings to
\begin{itemize}
\item Script file: \verb+$TUTORIAL/designs/fabric_blinky/build/exported.tcl+
%$
\item Check: Include commands from current session only
\item Select: Relative file names
\item Click \emph{OK}.
\end{itemize}
The Tcl commands in exported.tcl script were the basis for \verb+scripts/libero.tcl+.
\end{enumerate}

\vskip5mm
\begin{center}
\fbox{\parbox{0.85\linewidth}{
\textbf{\textcolor{red}{WARNING:}}\newline
The design is not ready for hardware download, as pin assignments were not defined!}}
\end{center}

\clearpage
% -----------------------------------------------------------------------------
% Libero GUI
% -----------------------------------------------------------------------------
%
\begin{landscape}
\begin{figure}[p]
  \begin{center}
    \includegraphics[width=215mm]
    {figures/ex1_libero_gui_manual.png}
  \end{center}
  \caption{Libero GUI after manual project creation.}
  \label{fig:ex1_libero_gui_manual}
\end{figure}
\end{landscape}

\clearpage
% -----------------------------------------------------------------------------
\subsection{Pin Constraints}
% -----------------------------------------------------------------------------

In the Libero GUI:
%
\begin{enumerate}
\item Select the \emph{Design Flow} tab.
\item Select the \emph{Manage Constraints} step.
\item Right-mouse-click and select \emph{Open Constraints Manager View}.
\item The \emph{Constraints Manager} opens with the \emph{I/O Attributes}
tab selected.
\item Click the button/pull-down \emph{Edit$\rightarrow$Edit with I/O Editor}
button to edit the pin constraints.
\item Figure~\ref{fig:ex1_libero_gui_pins_default} shows the default pin
assignments. These pin assignments do not represent any specific hardware
design and need to be customized for the target hardware.

\item The pin assignments editor can be used to define the pin constraints specific
to a board, or can be used to create a template I/O constraints .pdc file, which
can then be edited with the board-specific constraints.
%
\item Update the \emph{I/O Editor} column for \emph{Pin Number} to match
Table~\ref{tab:ex1_pins}.

The default \emph{I/O Standard} of LVCMOS18 is correct for the clock,
asynchronous reset input and LED outputs, but is not correct for the probes.
These issues were resolved by exporting the .PDC file, and then manually
editing the bank voltages and I/O standard settings.

The \emph{I/O Editor} view includes a slew rate column. GPIO pins support
PDC slew rate constraints, whereas HSIO pins do not.
The GPIO default setting of OFF results in the fastest slew rate
(p26~\cite{Microchip_PFSoC_IO_2025}). The probes do not need fast slew rate,
so the slew rate was reduced.

\item Export a Physical Design Constraints I/O file:
\begin{itemize}
\item \emph{File$\rightarrow$Export Physical Constraint (PDC)$\rightarrow$I/O Constraint}
\item Change the exported filename to\newline
\verb+$TUTORIAL/designs/fabric_blinky/build/pfs_disco_io.pdc+%$
\item Check the \emph{Full Constraints} checkbox
\item Click \emph{OK}
\item Exit \emph{I/O Editor} and ignore the pin constraints changes.
\end{itemize}

The exported PDC was used as the basis for \verb+scripts/pfs_disco_io.pdc+.
\item Exit the I/O Editor GUI.
\item In the \emph{Constraint Manager}, \emph{I/O Attributes} tab, link the
pin constraints file\newline
\verb+$TUTORIAL/designs/fabric_blinky/scripts/pfs_disco_io.pdc+,\newline %$
check the  \emph{Place and Route} checkbox, and click the \emph{Save} button.

\item Click the green play button to run place-and-route with the new pin assignments.

\item Figure~\ref{fig:ex1_libero_gui_pins} shows the Discovery Kit pin
assignments.

\end{enumerate}

\vskip5mm
\begin{center}
\fbox{\parbox{0.85\linewidth}{
\textbf{\textcolor{red}{WARNING:}}\newline
Since this design does not have any interfaces with critical timing, this
design could now be downloaded to hardware. However, in general, designs
should have timing constraints, and be verified to meet those constraints,
before testing on hardware.}}
\end{center}

\clearpage
% -----------------------------------------------------------------------------
% Discovery Kit Blinky LED Pin Assignments
% -----------------------------------------------------------------------------
%
\begin{table}[p]
\caption{Discovery Kit blinky LED pin assignments.}
\label{tab:ex1_pins}
\begin{center}
\begin{tabular}{|l|c|c|c|c|l|}
\hline
Port name & Pin Name & Voltage & Drive & Slew & Note\\
\hline\hline
&&&&&\\
\texttt{arst\_n}  & T19   & LVCMOS18 &&& Push button SWITCH1\\
&&&&&\\
\texttt{clk}      & R18   & LVCMOS18 &&& 50MHz CLKIN input\\
&&&&&\\
\texttt{led[0]}   & T18   & LVCMOS18 & 4 & OFF & HSIO, VDDI0 = 1.8V\\
\texttt{led[1]}   & V17   & LVCMOS18 & 4 & OFF &\\
\texttt{led[2]}   & U20   & LVCMOS18 & 4 & OFF &\\
\texttt{led[3]}   & U21   & LVCMOS18 & 4 & OFF &\\
\texttt{led[4]}   & AA18  & LVCMOS18 & 4 & OFF &\\
\texttt{led[5]}   & V16   & LVCMOS18 & 4 & OFF &\\
\texttt{led[6]}   & U15   & LVCMOS18 & 4 & OFF &\\
&&&&&\\
\texttt{uart\_rx} & W21   & LVCMOS18 & 4 & OFF &\\
\texttt{uart\_tx} & Y21   & LVCMOS18 & 4 & OFF &\\
&&&&&\\
\texttt{probe[0]} & E18   & LVCMOS33 & 4 & ON  & GPIO, VDDI1 = 3.3V\\
\texttt{probe[1]} & F18   & LVCMOS33 & 4 & ON  &\\
\texttt{probe[2]} & E12   & LVCMOS33 & 4 & ON  &\\
\texttt{probe[3]} & G18   & LVCMOS33 & 4 & ON  &\\
&&&&&\\
\hline
\end{tabular}
\end{center}
\end{table}

\clearpage
% -----------------------------------------------------------------------------
% Libero GUI I/O Editor
% -----------------------------------------------------------------------------
%
\begin{landscape}
\begin{figure}[p]
  \begin{center}
    \includegraphics[width=215mm]
    {figures/ex1_libero_gui_pins_default.png}
  \end{center}
  \caption{Libero I/O Editor default pin assignments after manual project creation.}
  \label{fig:ex1_libero_gui_pins_default}
\end{figure}
\end{landscape}

\begin{landscape}
\begin{figure}[p]
  \begin{center}
    \includegraphics[width=215mm]
    {figures/ex1_libero_gui_pins.png}
  \end{center}
  \caption{Libero I/O Editor with Discovery Kit pin assignments.}
  \label{fig:ex1_libero_gui_pins}
\end{figure}
\end{landscape}

\clearpage
% -----------------------------------------------------------------------------
\subsection{Timing Constraints}
% -----------------------------------------------------------------------------

In the Libero GUI:
%
\begin{enumerate}
\item Select the \emph{Design Flow} tab.
\item Select the \emph{Manage Constraints} step.
\item Right-mouse-click and select \emph{Open Constraints Manager View}.
\item The \emph{Constraints Manager} opens with the \emph{I/O Attributes}
tab selected.
\item Click on the \emph{Timing} tab.
\item Click on the button/pull-down \emph{Edit$\rightarrow$Edit Synthesis Constraints}.
\item The \emph{Constraints Editor} allows you to construct timing constraints
that adhere to Synopsys Design Constraints (SDC) syntax.
Constraints are saved in the Libero project in the file
\verb+constraint\user.sdc+. For example, a 50MHz clock constraint was
created and saved, and the corresponding SDC Tcl command was:
%
\begin{verbatim}
create_clock -name {clk_50mhz} -period 20 -waveform {0 10} \
    [get_ports {clk_50mhz}]
\end{verbatim}

\item Exit the \emph{Constraints Editor} GUI.
\item Remove \verb+user.sdc+ from the project.
\item In the \emph{Constraint Manager}, \emph{Timing} tab, link the
timing constraints file\newline
\verb+$TUTORIAL/designs/fabric_blinky/scripts/pfs_disco.sdc+,\newline %$
leave the \emph{Synthesis} checkbox unchecked,
check the \emph{Place and Route} checkbox,
check the \emph{Timing Verification} checkbox,
and click the \emph{Save} button.

The \emph{Synthesis} checkbox was left unchecked, as this checkbox is associated
with the timing constraint file passed to Synplify for synthesis. The default
Synplify settings are sufficient for most designs, so a Synplify SDC file
is not required.

\item Open the SDC file and review the LED clock-to-output timing parameters.

\item Click the green play button to run place-and-route with constraints.

\item In the \emph{Design Flow} GUI, select the \emph{Open SmartTime} step,
right-mouse-click and select \emph{Open Interactively}.
\end{enumerate}

The SmartTime GUI can be used to view different timing paths within the
design, and can be used to generate a schematic diagram of the delay paths.
%
Figures~\ref{fig:ex1_libero_smarttime_fabric_max}
and~\ref{fig:ex1_libero_smarttime_fabric_min} show the maximum and
minimum delay paths for the LED clock-to-output delays with the output
registers in the fabric.
The reported slack is under 0.1ns, as the SDC constraint for the maximum delay
was set to equal the reported delay rounded up to the nearest 0.1ns,
while the minimum delay was set equal to the reported delay rounded down
to the nearest 0.1ns.
%
Figure~\ref{fig:ex1_libero_smarttime_path}(a) shows the delay
path for the LED output registers in the fabric: the output register
is an SLE.
%
Figure~\ref{fig:ex1_libero_chipplanner}(a) shows the ChipPlanner
view for the LED output registers in the fabric: the blinky logic
was selected so it is highlighed in white, while the I/O pads
are highlighed in yellow.

\clearpage
% -----------------------------------------------------------------------------
% LED clock-to-output delay
% -----------------------------------------------------------------------------
%
\begin{table}
\caption{Blinky LED clock-to-output delays (in nanoseconds).}
\label{tab:ex1_tco}
\begin{center}
\begin{tabular}{|l||r|r||r|r|}
\hline
          & \multicolumn{2}{c||}{Fabric Registers} &
           \multicolumn{2}{c|}{I/O Registers}\\
\cline{2-5}
Parameter & Min & Max & Min & Max\\
\hline\hline
&\phantom{XXXX}&\phantom{XXXX}&\phantom{XXXX}&\phantom{XXXX}\\
Clock-to-Output & 3.769 & 6.651 &  3.587 & 6.010\\
Constraint      & 3.700 & 6.700 &  3.500 & 6.100\\
Slack           & 0.069 & 0.049 &  0.087 & 0.090\\
&&&&\\
\hline
\end{tabular}
\end{center}
\end{table}

% -----------------------------------------------------------------------------
\subsection{I/O Register Constraints}
% -----------------------------------------------------------------------------

The placement of I/O registers in PolarFire SoC designs can be
contolled using a Netlist Design Constraint (NDC) file. In the Libero GUI:
%
\begin{enumerate}
\item Select the \emph{Design Flow} tab.
\item Select the \emph{Manage Constraints} step.
\item Right-mouse-click and select \emph{Open Constraints Manager View}.
\item The \emph{Constraints Manager} opens with the \emph{I/O Attributes}
tab selected.
\item Click on the \emph{Netlist Attributes} tab.
\item Link the netlist constraints file\newline
\verb+$TUTORIAL/designs/fabric_blinky/scripts/pfs_disco.ndc+,\newline %$
check the \emph{Synthesis} checkbox,
and click the \emph{Save} button.
\item Edit \verb+pfs_disco.ndc+ to disable (or enable) the use of I/O registers.
\item Edit \verb+pfs_disco.sdc+ to set the timing constraints for fabric (or I/O) registers.

\item Click the green play button to run place-and-route.

\item In the \emph{Design Flow} GUI, select the \emph{Open SmartTime} step,
right-mouse-click and select \emph{Open Interactively}.
\end{enumerate}
%
Figures~\ref{fig:ex1_libero_smarttime_ioff_max}
and~\ref{fig:ex1_libero_smarttime_ioff_min} show the maximum and
minimum delay paths for the LED clock-to-output delays with the output
registers in the I/O.
The reported slack is under 0.1ns, as the SDC constraint for the maximum delay
was set to equal the reported delay rounded up to the nearest 0.1ns,
while the minimum delay was set equal to the reported delay rounded down
to the nearest 0.1ns.
%
Figure~\ref{fig:ex1_libero_smarttime_path}(b) shows the delay
path for the LED output registers in the I/O: the output register
is an IOREG.
%
Figure~\ref{fig:ex1_libero_chipplanner}(b) shows the ChipPlanner
view for the LED output registers in the I/O.
A close comparison of the white and yellow highlighted components in
Figure~\ref{fig:ex1_libero_chipplanner}(a) and (b) shows
that (a) has additional registers near the output pads that are not
highlighted in (b): these are the fabric output registers.

Table~\ref{tab:ex1_tco} summarizes the LED clock-to-output delays for
the output registers placed in the fabric vs I/O. The constraint
values can be used as \emph{datasheet} values when analyzing external
device interface timing. If external device interface timing requirements
cannot be met with either of these output register configurations shown in
Table~\ref{tab:ex1_tco}, then the clock-to-output delay can be adjusted
using the output drive strength, or using the programmable output delay,
or using BUFD delays within the fabric.

\clearpage
% -----------------------------------------------------------------------------
% Libero GUI SmartTime Timing Analyzer
% -----------------------------------------------------------------------------
%
% Bring up both the max and min GUIs prior to screen capture, so that the
% window title boxes contain the project path.
%
\begin{landscape}
\begin{figure}[p]
  \begin{center}
    \includegraphics[width=215mm]
    {figures/ex1_libero_smarttime_fabric_max.png}
  \end{center}
  \caption{Libero SmartTime maximum delay with LED output registers in the fabric.}
  \label{fig:ex1_libero_smarttime_fabric_max}
\end{figure}
\end{landscape}

\begin{landscape}
\begin{figure}[p]
  \begin{center}
    \includegraphics[width=215mm]
    {figures/ex1_libero_smarttime_fabric_min.png}
  \end{center}
  \caption{Libero SmartTime minimum delay with LED output registers in the fabric.}
  \label{fig:ex1_libero_smarttime_fabric_min}
\end{figure}
\end{landscape}

\clearpage
% -----------------------------------------------------------------------------
% Libero GUI SmartTime Timing Analyzer
% -----------------------------------------------------------------------------
%
% Bring up both the max and min GUIs prior to screen capture, so that the
% window title boxes contain the project path.
%
\begin{landscape}
\begin{figure}[p]
  \begin{center}
    \includegraphics[width=215mm]
    {figures/ex1_libero_smarttime_ioff_max.png}
  \end{center}
  \caption{Libero SmartTime maximum delay with LED output registers in the I/O.}
  \label{fig:ex1_libero_smarttime_ioff_max}
\end{figure}
\end{landscape}

\begin{landscape}
\begin{figure}[p]
  \begin{center}
    \includegraphics[width=215mm]
    {figures/ex1_libero_smarttime_ioff_min.png}
  \end{center}
  \caption{Libero SmartTime minimum delay with LED output registers in the I/O.}
  \label{fig:ex1_libero_smarttime_ioff_min}
\end{figure}
\end{landscape}

\clearpage
% -----------------------------------------------------------------------------
% Libero GUI SmartTime Timing Analyzer Path Delays
% -----------------------------------------------------------------------------
%
\begin{landscape}
\begin{figure}[p]
  \begin{center}
    \includegraphics[width=215mm]
    {figures/ex1_libero_smarttime_fabric_path.png}\\
    (a) Fabric registers\\
    \vskip5mm
    \includegraphics[width=215mm]
    {figures/ex1_libero_smarttime_ioff_path.png}\\
    (b) I/O registers
  \end{center}
  \caption{Libero SmartTime timing path for with LED output registers in the fabric vs I/O.}
  \label{fig:ex1_libero_smarttime_path}
\end{figure}
\end{landscape}

% -----------------------------------------------------------------------------
% Libero GUI ChipPlanner Views
% -----------------------------------------------------------------------------
%
\begin{figure}[p]
  \begin{center}
    \includegraphics[width=0.73\textwidth]
    {figures/ex1_libero_chipplanner_fabric.png}\\
    (a) Fabric registers\\
    \vskip5mm
    \includegraphics[width=0.73\textwidth]
    {figures/ex1_libero_chipplanner_ioff.png}\\
    (b) I/O registers
  \end{center}
  \caption{Libero ChipPlanner view with LED output registers in the fabric vs I/O.}
  \label{fig:ex1_libero_chipplanner}
\end{figure}

\clearpage
% -----------------------------------------------------------------------------
\subsection{Floorplan Constraints}
% -----------------------------------------------------------------------------

The Libero ChipPlanner view of a design can be accessed via:
%
\begin{enumerate}
\item Select the \emph{Design Flow} tab.
\item Select the \emph{Manage Constraints} step.
\item Right-mouse-click and select \emph{Open Constraints Manager View}.
\item The \emph{Constraints Manager} opens with the \emph{I/O Attributes}
tab selected.
\item Click on the \emph{Floor Planner} tab.
\item Click on the \emph{View} button.
\item The ChipPlanner tool is then launched as a separate application.
\end{enumerate}
%
The ChipPlanner view shows the PolarFire SoC die, the logic within the die,
and the logic in the I/O banks.
%
Figure~\ref{fig:ex1_libero_chipplanner} shows the ChipPlanner view of
the LED design: the logic in the fabric is mainly the counter used to
blink the LEDs.

Floorplan constraints are user-defined constraints that lock logic at a
specific physical location on the die. Components that should be considered
for explicit floorplanning are clock buffers, reset buffers, and reset
synchronizers. Reset synchronizers are an interesting case, as the Libero
tool does not have any unique constraint for identifying registers that
should be placed tightly together. Designs that include synchronizer
components should also include floorplanning constraints and timing
constraints to ensure the synchronizers are implemented correctly.

The reset synchronizer registers placement was investigated as follows;
%
\begin{enumerate}
\item \textbf{No additional constraints}

Figure~\ref{fig:ex1_libero_chipplanner_reset}(a) shows the
ChipPlanner view of the synchronizer registers, with the synchronizer
registers highlighted in white:
the four registers were not placed in the same row of registers.

\item \textbf{PDC placement constraints}

Floorplan constraints were used to place the four synchronizer registers
into the a row of registers located close to the global clock and reset
buffers that are located near the center bottom of the die.
Figure~\ref{fig:ex1_libero_chipplanner_reset}(c) shows the
ChipPlanner view of the registers, with the synchronizer registers
highlighted in white: the four registers are placed in a row of
registers just above the global reset buffer.

\item \textbf{PDC placement plus SDC delay constraints}

SDC delay constraints were added and adjusted until the slack for
the minimum and maximum was under 10ps.

\item \textbf{SDC delay constraints}

The PDC constraints were removed from the project (unchecked in the
constraints manager view) to determine if tight SDC constraints were
sufficient to place the registers in the same row.

Figure~\ref{fig:ex1_libero_chipplanner_reset}(b) shows the
SDC constraints did result in the registers being placed in the same row,
however, the figure shows that the synchronizer registers were not packed
tightly together. The figure also shows that the synchronizer was placed
by the counter, rather than the desired location, by the global reset buffer.

\end{enumerate}
%
This test demonstrated that both PDC and SDC constraints should be used
for synchronizer placement. The PDC constraints ensure the registers are
placed where the user wants them, while the SDC constraints are
\emph{insurance} that checks the register placement is as the user intended.

\clearpage
% -----------------------------------------------------------------------------
% Libero GUI ChipPlanner view of reset synchronizer
% -----------------------------------------------------------------------------
%
\begin{figure}[p]
  \begin{center}
    \includegraphics[width=\textwidth]
    {figures/ex1_libero_chipplanner_reset_a.png}\\
    (a) No SDC or floorplan constraints\\
    \vskip5mm
    \includegraphics[width=\textwidth]
    {figures/ex1_libero_chipplanner_reset_b.png}\\
    (b) With SDC and without floorplan constraints
    \vskip5mm
    \includegraphics[width=\textwidth]
    {figures/ex1_libero_chipplanner_reset_c.png}\\
    (c) With SDC and floorplan constraints
  \end{center}
  \caption{Libero ChipPlanner view of the reset synchronizer registers.}
  \label{fig:ex1_libero_chipplanner_reset}
\end{figure}

\clearpage
% -----------------------------------------------------------------------------
% Synplify Elite Hierarchy
% -----------------------------------------------------------------------------
%
\begin{figure}[t]
  \begin{center}
    \includegraphics[width=\textwidth]
    {figures/ex1_synplify_elite_hierarchy.png}
  \end{center}
  \caption{Synplify Elite \emph{Hierarchical Area Report}.}
  \label{fig:ex1_synplify_elite_hierarchy}
\end{figure}

% -----------------------------------------------------------------------------
\subsection{Triple Modular Redundancy (TMR)}
% -----------------------------------------------------------------------------
\label{sec:ex1_tmr}

Synopsys Synplify Elite can be used to implement various TMR schemes,
eg., Local TMR (LTMR), Distributed TMR (DTMR), and Block TMR (BTMR).
%
The following process was used to triplicate the blinky LED component
within the Discovery Kit design:
%
\begin{enumerate}
\item In the Libero GUI, select the \emph{Design Flow} tab.
\item Use the \emph{Constraints Manager}, \emph{Floor Planner} tab, to
link the floorplan constraints file\newline
\verb+$TUTORIAL/designs/fabric_blinky/scripts/pfs_disco_fp.pdc+,\newline %$
check the \emph{Place and Route} checkbox,
and click the \emph{Save} button.

Review the FDC file for the floorplan options.
\item Use the \emph{Constraints Manager}, \emph{Netlist Attributes} tab, to
link the TMR constraints file\newline
\verb+$TUTORIAL/designs/fabric_blinky/scripts/pfs_disco.fdc+,\newline %$
check the \emph{Synthesis} checkbox,
and click the \emph{Save} button.

Review the FDC file for the DTMR constraint syntax.
\item Select the \emph{Design Flow} \emph{Synthesize} step, right-mouse-click and select
\emph{Clean}.

This clears the synthesis results created by \emph{Synplify Base}.

\item Use \emph{Project$\rightarrow$Tool Profiles},
then \emph{Synthesis}, and select the \verb+Synplify Elite+ tool profile.
\item Select the \emph{Synthesize} step, right-mouse-click and select
\emph{Open Interactively}.

\item (Optional) Delete the (empty) FDC constraint file:
\verb+synthesis.fdc+%$
\item Click the \emph{Run} button and wait for synthesis to complete.
\item In the \emph{Project Status} window, click the \emph{Hierarchical Area Report} link.

Figure~\ref{fig:ex1_synplify_elite_hierarchy} shows the report with the
triplicated \verb+blinky_31s_7s+ hierarchy expanded.

\item In the Libero GUI, the \emph{Synthesize} step will now be green.

\item Exit \emph{Synplify Elite}.

\item In the Libero GUI, click the green play button to run place-and-route
for the triplicated design.

\item Open \emph{ChipPlanner} to view the TMR instance placement.
\end{enumerate}

Figure~\ref{fig:ex1_libero_chipplanner_tmr} shows the ChipPlanner view
with blinky instance TMR1 selected. The instance was selected using the
\emph{ChipPlanner} \emph{Logical} tab, by expanding the \verb+u5+ component
to see the three TMR instances, and then selecting instance TMR1.
Figure~\ref{fig:ex1_libero_chipplanner_tmr} shows that logic for the
three TMR instances is intermingled on the die.

The triplicated logic can be floorplanned to be placed into unique (mutually
exclusive) regions.
%
Figure~\ref{fig:ex1_libero_chipplanner_tmr_fp}(a) shows the ChipPlanner
view after three unique regions were defined (see the floorplan PDC file) and
the three blinky TMR instances assigned to a unique region.
%
Figure~\ref{fig:ex1_libero_chipplanner_tmr_fp}(b) shows the ChipPlanner
view for a floorplanned non-TMR design.
%
The floorplan regions in
Figure~\ref{fig:ex1_libero_chipplanner_tmr_fp} were created in the
same location on the die as the placement in
Figure~\ref{fig:ex1_libero_chipplanner_tmr}, to help show how
floorplanning can \emph{clean up} the placement of the TMR instances.
%
The Chipplanner views in
Figures~\ref{fig:ex1_libero_chipplanner_tmr}
and~\ref{fig:ex1_libero_chipplanner_tmr_fp} can be reproduced
by editing the floorplan FDC, the NDC (to place the LED output registers),
and the SDC (to configure the timing constraints).

% -----------------------------------------------------------------------------
% ChipPlanner view of DTMR without floorplanning
% -----------------------------------------------------------------------------
%
\begin{landscape}
\begin{figure}[p]
  \begin{center}
    \includegraphics[width=210mm]
    {figures/ex1_libero_chipplanner_tmr.png}
  \end{center}
  \caption{Libero ChipPlanner view of the TMR design without floorplanning.}
  \label{fig:ex1_libero_chipplanner_tmr}
\end{figure}
\end{landscape}

% -----------------------------------------------------------------------------
% ChipPlanner view of No TMR vs DTMR with floorplanning
% -----------------------------------------------------------------------------
%
\begin{figure}[p]
  \begin{center}
    \includegraphics[width=0.73\textwidth]
    {figures/ex1_libero_chipplanner_tmr_fp.png}\\
    (a) TMR\\
    \vskip5mm
    \includegraphics[width=0.73\textwidth]
    {figures/ex1_libero_chipplanner_no_tmr_fp.png}\\
    (b) No TMR
  \end{center}
  \caption{Libero ChipPlanner view of the TMR and non-TMR design with floorplanning.}
  \label{fig:ex1_libero_chipplanner_tmr_fp}
\end{figure}

\clearpage
% -----------------------------------------------------------------------------
\subsection{Libero Scripted Flow}
% -----------------------------------------------------------------------------

Table~\ref{tab:pfs_disco_variants} shows a subset of the Discovery Kit blinky
LED design variants that have been analyzed in this tutorial, along with
additional variants that are used to investigate the clock-to-output
delay variation with drive strength and load capacitance.
%
It is important to be able to reproduce the results of each variant,
and that is where scripting helps.

The requirements for the Libero scripted flow are:
%
\begin{enumerate}
\item Customize the RTL parameters.
\item Customize the SDC timing constraints.
\item Customize the PDC pin constraints.
\item Customize the PDC floorplan constraints.
\item Customize the NDC netlist constraints.
\item Customize the FDC netlist constraints (for TMR).
\item Use Synplify Base or Elite (for TMR).
\item Generate custom timing reports.
\end{enumerate}
%
The Discovery Kit variant scripts support all of these requirements.
The variant constraints files each differ with minor  parameter changes,
so template files are used, and the variant scripts update the variant
parameters, and create the customized constraint files in the build area.

The Libero scripted flow is implemented using the following design scripts:
%
\begin{itemize}
\item \verb+variant.tcl+ Design variant procedures
\item \verb+libero_variants.tcl+ Libero script that generates all variants
\item \verb+timing_report.tcl+ LED output timing report
\end{itemize}
%
The design variants are built as follows:
%
\begin{enumerate}
\item Start \emph{Libero SoC 2025.1}
\item Select \emph{Project$\rightarrow$Execute Script}
\item Use the \emph{Script file:} browse button to select\newline
\verb+$TUTORIAL/designs/fabric_blinky/scripts/libero_variants.tcl+%$
\item Click \emph{Run}
\end{enumerate}
%
The script generates Libero projects for all of the design variants in
Table~\ref{tab:pfs_disco_variants}, runs synthesis, place-and-route,
and generates timing reports (total time for all projects is about 35 minutes).

\clearpage
The timing analysis results are reproducible. A reference copy of the
timing reports are located in
\verb+$TUTORIAL/designs/fabric_blinky/timing_reports+ %$
along with a bash script that can be used to compare (diff) the
reference timing reports to those in the build area.
%
For example, under Windows 10 using a WSL2 bash terminal, the timing
reports in the build area are checked as follows:
%
\begin{verbatim}
$ export TUTORIAL=/mnt/c/github/microchip_pfs_disco
$ cd $TUTORIAL/designs/fabric_blinky/timing_reports
$ ./compare_to_build.sh
Variant 1: PASS
Variant 2: PASS
Variant 3: PASS
Variant 4: PASS
Variant 5: PASS
Variant 6: PASS
Variant 7: PASS
Variant 8: PASS
Variant 9: PASS
Variant 10: PASS
Variant 11: PASS
Variant 12: PASS
\end{verbatim}
%
The clock-to-output delays in Table~\ref{tab:pfs_disco_variants} were
obtained from the timing report files. The timing report clock-to-output
delays match the SmartTime reported values, but reviewing text files is
faster than repeatedly opening a Libero project, then SmartTime, and then
reviewing the SmartTime clock-to-output delay for maximum (default SmartTime
window) and minimum (another SmartTime window).

% -----------------------------------------------------------------------------
% Design Variants
% -----------------------------------------------------------------------------
%
\begin{table}[p]
\caption{Discovery Kit design variants}
\label{tab:pfs_disco_variants}
\begin{center}
\begin{tabular}{|c||c|c|c||c|c||c|c|}
\hline
Variant & \multicolumn{3}{c||}{Output} & Floorplan & TMR &
          \multicolumn{2}{c|}{Clock-to-Output}\\
\cline{2-4}
\cline{7-8}
        & Registers & Drive & Load & Region & & Min & Max\\
\hline\hline
&&&&&&\phantom{XXXXX}&\phantom{XXXXX}\\
 1 &   Fabric &  4 &  5 & No  & No  & 3.769 & 6.605\\
 2 &   I/O    &  4 &  5 & No  & No  & 3.587 & 6.010\\
&&&&&&&\\
 3 &   Fabric &  4 &  5 & Yes & No  & 3.769 & 6.605\\
 4 &   I/O    &  4 &  5 & Yes & No  & 3.587 & 6.010\\
&&&&&&&\\
 5 &   Fabric &  4 &  5 & Yes & Yes & 3.769 & 6.605\\
 6 &   I/O    &  4 &  5 & Yes & Yes & 3.587 & 6.010\\
&&&&&&&\\
 7 &   Fabric &  8 &  5 & Yes & No  & 3.571 & 6.397\\
 8 &   I/O    &  8 &  5 & Yes & No  & 3.373 & 5.802\\
 9 &   Fabric & 12 &  5 & Yes & No  & 3.454 & 6.271\\
10 &   I/O    & 12 &  5 & Yes & No  & 3.254 & 5.676\\
&&&&&&&\\
11 &   Fabric &  4 & 10 & Yes & No  & 4.138 & 6.972\\
12 &   I/O    &  4 & 10 & Yes & No  & 3.955 & 6.377\\
13 &   Fabric &  4 & 20 & Yes & No  & 4.877 & 7.706\\
14 &   I/O    &  4 & 20 & Yes & No  & 4.678 & 7.111\\
&&&&&&&\\
\hline
\end{tabular}
\end{center}
\end{table}

\clearpage
% =============================================================================
\section{Example 2: RISC-V plus Fabric Blinky LEDs}
% =============================================================================
\label{sec:riscv_blinky}

This section demonstrates how to use the Discovery Kit reference hardware
and software to blink all eight LEDs using the GPIO registers in the MSS.
%
Figure~\ref{fig:pfs_disco_reference_hardware_design} shows the
\href{https://github.com/polarfire-soc/polarfire-soc-discovery-kit-reference-design}
{PolarFire SoC Discovery Kit Reference Hardware Design.}
%
The RISC-V Blinky LEDs application uses the following GPIOs shown in the figure:
%
\begin{itemize}
\item \texttt{LED8} is driven by MSS \texttt{GPIO\_1\_9} at address \texttt{0x2012\_1000}
\begin{itemize}
\item
Figure~\ref{fig:pfs_disco_reference_hardware_mss_cfg_gpio_1} shows the MSS Configurator
GPIO\_1 settings.
\end{itemize}
%
\item \texttt{LED[1:7]} are driven by the OR of:
\begin{itemize}
\item MSS \texttt{GPIO\_2\_[17:23]} at address \texttt{0x2012\_1000}
\item FIC3 CoreGPIO at address \texttt{0x4000\_0000}
\end{itemize}
\end{itemize}
%
The PolarFire GPIO registers addresses are defined in the PolarFire SoC
Registers Map which can be downloaded via the zip file linked from
page 20 of the Technical Reference Manual~\cite{Microchip_PFSoC_TRM_2025}.

% -----------------------------------------------------------------------------
\subsection{MSS I/O pin number to MSS GPIO software API mapping}
% -----------------------------------------------------------------------------

The mapping from the LED8 schematic pin name to the MSS GPIO software API
arguments required some analysis. The Discovery Kit schematic shows that LED8
connects to pin number E1 with pin name \textsf{MSSIO23B2}, i.e., MSS I/O 23 in
bank 2 (p7~\cite{Microchip_DISCO_SCH_2023}).
%
\begin{center}
\textcolor{magenta}{\bf How does MSS I/O 23 in Bank 2 become GPIO\_1\_9?}
\end{center}
%
The PolarFire SoC MSS has 38 general purpose I/O pads called MSS I/Os (MSSIO),
with 14 pins in Bank 4 and 24 pins in Bank 2 (p48~\cite{Microchip_PFSoC_TRM_2025}).
%
The Discovery Kit schematic show how these 38 MSS I/O pins map to schematic
symbol banks, pin numbers, and pin names, while the MSS Configurator GUI shows how
the 38 MSS I/O pins map to banks, pin numbers, and software registers.
%
The Discovery Kit schematic~\cite{Microchip_DISCO_SCH_2023} page 4 shows
that Bank 4 contains pin names \textsf{MSSIO0B4} through \textsf{MSSIO13B4},
i.e., MSS I/Os \textsf{MSSIO[0:13]}, while page 7 shows that Bank 2
contains pin names \textsf{MSSIO14B2} through \textsf{MSSIO37B2}, i.e.,
MSS I/Os \textsf{MSSIO[14:37]}.

Figure~\ref{fig:pfs_disco_reference_hardware_mss_cfg_gpio_1} shows the
MSS Configurator \emph{Peripherals} tab with the GPIO\_1 settings
selected. The first two columns of the table are \textbf{BANK} and
\textbf{IO MUX}, where \textbf{BANK} is the bank number and \textbf{IO MUX}
is the MSS I/O index.
%
The Bank 4 MSS I/O bits \textsf{MSSIO[0:13]} map to MSS GPIO control
register bits \textsf{GPIO\_0\_[0:13]}, while
the Bank 2 MSS I/O bits \textsf{MSSIO[14:37]} map to MSS GPIO control
register bits \textsf{GPIO\_1\_[0:27]}. This provides the answer to
the question posed above:
%
\begin{center}
\textcolor{magenta}{\bf MSS I/O 23 in Bank 2 = GPIO\_1 bit (23 - 14) = GPIO\_1\_9}
\end{center}

% -----------------------------------------------------------------------------
\subsection{MSS GPIO software API documentation}
% -----------------------------------------------------------------------------

The System Tick function shown in Figure~\ref{fig:pfs_disco_reference_software_design}
blinks LED1 using the MSS GPIO software API call MSS\_GPIO\_set\_output().
%
The bare-metal driver API is embedded in the source code (using doxygen-like
annotations), and the source is used to generate an API user guide in the
\href{https://github.com/polarfire-soc/polarfire-soc-documentation}
{PolarFire SoC Documentation} github repo.
%
The
\href{https://github.com/polarfire-soc/polarfire-soc-documentation/blob/master/bare-metal-embedded-software/bare-metal-driver-user-guides/polarfire-soc-mss-driver-user-guides/mss-gpio/mss-gpio-driver-user-guide.md}
{MSS GPIO Bare Metal Driver} contains the MSS GPIO software API
documentation. Relevant API calls for the blinky LED software are:
%
\begin{itemize}
\item
\href{https://github.com/polarfire-soc/polarfire-soc-documentation/blob/master/bare-metal-embedded-software/bare-metal-driver-user-guides/polarfire-soc-mss-driver-user-guides/mss-gpio/mss-gpio-driver-user-guide.md#mss_gpio_init}
{MSS\_GPIO\_init()}
\item
\href{https://github.com/polarfire-soc/polarfire-soc-documentation/blob/master/bare-metal-embedded-software/bare-metal-driver-user-guides/polarfire-soc-mss-driver-user-guides/mss-gpio/mss-gpio-driver-user-guide.md#mss_gpio_config}
{MSS\_GPIO\_config()}
\item
\href{https://github.com/polarfire-soc/polarfire-soc-documentation/blob/master/bare-metal-embedded-software/bare-metal-driver-user-guides/polarfire-soc-mss-driver-user-guides/mss-gpio/mss-gpio-driver-user-guide.md#mss_gpio_set_output}
{MSS\_GPIO\_set\_output()}
\item
\href{https://github.com/polarfire-soc/polarfire-soc-documentation/blob/master/bare-metal-embedded-software/bare-metal-driver-user-guides/polarfire-soc-mss-driver-user-guides/mss-gpio/mss-gpio-driver-user-guide.md#mss_gpio_set_outputs}
{MSS\_GPIO\_set\_outputs()}
\end{itemize}

\clearpage
% -----------------------------------------------------------------------------
\subsection{Reference Hardware}
% -----------------------------------------------------------------------------

The Discovery Kit Reference Hardware Design is built as follows:
%
\begin{enumerate}
\item Clone the github repository

For example, using Windows 10 WSL
%
\begin{verbatim}
$ cd /mnt/c/github
$ git clone https://github.com/polarfire-soc/
      polarfire-soc-discovery-kit-reference-design.git
\end{verbatim}

\item Start Libero SoC 2025.1

\item Use \emph{Project$\rightarrow$Execute Script} to select and run
%
\begin{verbatim}
c:/github/polarfire-soc-discovery-kit-reference-design/
      MPFS_DISCOVERY_KIT_REFERENCE_DESIGN.tcl
\end{verbatim}

The Libero SoC GUI updates repeatedly as the script creates the SmartDesign.

The script run-time is about 5 minutes.

\item Click on the green play button to run synthesis and place-and-route

The synthesis and place-and-route run-time is about 10 minutes.

\item Double-mouse-click on \emph{Export FlashPro Express Job}

Make a note of the output job directory as this is needed with FlashPro.

\item Exit Libero

\end{enumerate}
%
Use FlashPro to program the Discovery Kit following the instructions in
Section~\ref{sec:factory_restore}, but use the job file just created.
After programming with the reference hardware design, the Discovery Kit
LED8 will be on (red) and LED1 through LED7 will be off. The next
sections describe how to blink each of the LEDs.

% -----------------------------------------------------------------------------
% Reference hardware design
% -----------------------------------------------------------------------------
%
\begin{landscape}
\begin{figure}[p]
  \begin{center}
    \includegraphics[width=200mm]
    {figures/pfs_disco_reference_hardware_design.pdf}
  \end{center}
  \caption{PolarFire SoC Discovery Kit reference hardware design.}
  \label{fig:pfs_disco_reference_hardware_design}
\end{figure}
\end{landscape}
% -----------------------------------------------------------------------------

% -----------------------------------------------------------------------------
% MSS Configurator for GPIO_1
% -----------------------------------------------------------------------------
%
\begin{landscape}
\begin{figure}[p]
  \begin{center}
    \includegraphics[width=215mm]
    {figures/pfs_disco_reference_hardware_mss_cfg_gpio_1.png}
  \end{center}
  \caption{Discovery Kit MSS Configurator GPIO\_1 settings.}
  \label{fig:pfs_disco_reference_hardware_mss_cfg_gpio_1}
\end{figure}
\end{landscape}
% -----------------------------------------------------------------------------

% -----------------------------------------------------------------------------
\subsection{Reference Software}
% -----------------------------------------------------------------------------

The Discovery Kit Reference Software Design is built as follows:
%
\begin{enumerate}
\item Clone the github bare-metal examples repository

For example, using Windows 10 WSL
%
\begin{verbatim}
$ cd /mnt/c/github
$ git clone https://github.com/polarfire-soc/
      polarfire-soc-bare-metal-examples
\end{verbatim}

\item Start SoftConsole 2022.2

\item Create a new Eclipse workspace, eg.,

\begin{verbatim}
C:\github\polarfire-soc-bare-metal-examples\ws-mss-gpio
\end{verbatim}

\item Close the Welcome window

\item Select \emph{File$\rightarrow$Import} and then \emph{Existing Projects into Workspace}

\item Click \emph{Next}

\item Point the root directory at the bare-metal examples github repo

\item Click \emph{Deselect All}

\item Check \texttt{mpfs-gpio-interrupt}

\item Click \emph{Finish}

\item Use the \emph{Project Explorer} to view the source

Figure~\ref{fig:pfs_disco_reference_software_design} shows the SoftConsole GUI
with the source code for hardware thread (hart) 1, i.e., \verb+u54_1.c+.
The source code shown in the figure is the System Tick interrupt handle,
which blinks LED1.

\item Configure the build configuration and build the project
%
\begin{itemize}
\item Select \emph{Project$\rightarrow$Build Configurations$\rightarrow$Set Active}
\item Select \emph{LIM-Debug-DiscoveryKit}
\item This configuration uses boot mode 0
\item Select \emph{Project$\rightarrow$Build All}
\end{itemize}

\item Download to the Discovery Kit
%
\begin{itemize}
\item Power the Discovery Kit using a USB-C cable
\item Select \emph{Run$\rightarrow$Debug Configurations}
\item Select \emph{mpfs-gpio-interrupt hw all-harts debug}
\item Change the build configuration from \emph{Use Active} to \emph{LIM-Debug-DiscoveryKit}
\item Click \emph{Apply}
\item Click \emph{Debug}
\end{itemize}

% -----------------------------------------------------------------------------
% Reference software design
% -----------------------------------------------------------------------------
%
\begin{landscape}
\begin{figure}[p]
  \begin{center}
    \includegraphics[width=200mm]
    {figures/pfs_disco_softconsole_u54_systick.png}
  \end{center}
  \caption{PolarFire SoC Discovery Kit reference software design.}
  \label{fig:pfs_disco_reference_software_design}
\end{figure}
\end{landscape}
% -----------------------------------------------------------------------------

\newpage
\item Resume and then pause the application
%
\begin{itemize}
\item The debug view opens with line 33 highlighted
\item Press the green play button to Resume (run the application)
\item The orange LED1 (near the corner of the board) will start blinking
\end{itemize}
\end{enumerate}
%
\textcolor{OliveGreen}{\bf Success! We have blinked LED1!}

% -----------------------------------------------------------------------------
\subsection{Software Debug}
% -----------------------------------------------------------------------------

An important part of software development is using a debugger to confirm your
understanding of the address map of your processor and hardware design.
%
The fact that LED1 is blinking means that the GPIO example has configured the
MSS GPIO registers correctly. That means we should be able to toggle all of
the LEDs using MSS GPIO registers. We can confirm our understanding of the
address map without writing a line of code as follows:
%
\begin{enumerate}
\item In the SoftConsole GUI, click on the Suspend (pause) button

\item Change the GUI to the debug perspective (click the bug on the right-hand-side)

\item Click on the \emph{Memory} tab near the bottom of the GUI.

This debug feature is the
\href{https://help.eclipse.org/latest/index.jsp?topic=%2Forg.eclipse.cdt.doc.user%2Freference%2Fcdt_u_memoryview.htm}
{Eclipse Memory View}.

\item \textbf{Toggle LED8}
%
\begin{itemize}
\item The Discovery Kit schematic shows LED8 connected to MSSIO23, pin E1, Bank 2 (p7~\cite{Microchip_DISCO_SCH_2023})
\item Open the MSS Configurator for the hardware reference design.
\item Select the \emph{Peripherals} tab.
\item The right-side of the GUI shows IO MUX bit 32, Bank 2, E1, configured for GPIO\_1\_9.
\item Figure~\ref{fig:pfs_disco_reference_hardware_mss_cfg_gpio_1} shows the MSS Configurator GUI after
selecting \emph{GPIO\_1 (Bank 2 I/Os)}.
\item The left-side of the GUI shows the pull-down menus for GPIO\_1\_9 configured for
\emph{MSS I/Os Bank2} with the direction set to \emph{Output}.
\item The PolarFire SoC Registers Map contains the GPIO\_1 register definitions under
the link GPIO\_IOBANK1\_LO.
\item The LED8 configuration register settings are:
\begin{itemize}
\item CONFIG\_9[0] = EN\_OUT = 1
\item CONFIG\_9[1] = EN\_OE\_BUF = 1
\item 0x2012\_1024 = 0x05
\end{itemize}
%
\item The LED8 output register settings (GPOUT) are:
\begin{itemize}
\item GPOUT[9] = 1 (LED on) or 0 (LED off)
\item 0x2012\_1089 = 0x02 (LED on) or 0x00 (LED off)
\end{itemize}
%
\item Use the Eclipse Memory View to add a new monitor at address 0x2012\_1000.
\item Change byte 0x2012\_1024 to 0x05.
\item LED8 will turn off.
\item Change byte 0x2012\_1089 to 0x02 and then 0x00.
\item LED8 will turn on and off.
\end{itemize}
\textcolor{OliveGreen}{\bf Success! We have blinked LED8!}

\newpage
\item \textbf{Toggle LED1 to LED7 using MSS GPIO}
%
\begin{itemize}
\item Figure~\ref{fig:pfs_disco_reference_hardware_design} shows that the
Discovery Kit hardware reference design controls LED[1:7] using
MSS GPIO\_2\_[17:23].
\item Open the MSS Configurator for the hardware reference design.
\item Select the \emph{Peripherals} tab.
\item Selecting \emph{GPIO\_2 (Fabric)} on the left-side of the GUI shows the pull-down
menus for GPIO\_2\_17 to 23 configured for \emph{Fabric I/O} with the direction set to \emph{Output}.
\item The PolarFire SoC Registers Map contains the GPIO\_2 register definitions under
the link GPIO\_FAB\_LO.
\item The LED1 to LED7 configuration register settings do not matter, as the
hardware reference design only connects to the GPIO outputs (not the output enables).
%
\item The LED1 to 7 output register (GPOUT) settings are:
\begin{itemize}
\item GPOUT[n] = 1 (LED on) or 0 (LED off) for n = 17, 18, 19, 20, 21, 22, 23
\item 0x2012\_208A = 0x02 (LED1 on) or 0x00 (LED1 off)
\item 0x2012\_208A = 0x04 (LED2 on) or 0x00 (LED2 off)
\item 0x2012\_208A = 0x08 (LED3 on) or 0x00 (LED3 off)
\item 0x2012\_208A = 0x10 (LED4 on) or 0x00 (LED4 off)
\item 0x2012\_208A = 0x20 (LED5 on) or 0x00 (LED5 off)
\item 0x2012\_208A = 0x40 (LED6 on) or 0x00 (LED6 off)
\item 0x2012\_208A = 0x80 (LED7 on) or 0x00 (LED7 off)
\end{itemize}
%
\item Use the Eclipse Memory View to add a new monitor at address 0x2012\_2000.
\item Change the byte at 0x2012\_108A to turn each of the LEDs on and off.
\item Turn the LEDs off before performing the next steps.
\end{itemize}
\textcolor{OliveGreen}{\bf Success! We have blinked LED1 through LED7 using MSS GPIO!}

\item \textbf{Toggle LED1 to LED7 using FIC\_3 CoreGPIO}
%
\begin{itemize}
\item Figure~\ref{fig:pfs_disco_reference_hardware_design} shows that the
Discovery Kit hardware reference design can also control LED[1:7] using
the CoreGPIO at base address 0x4000\_0100.
%
\item The \href{https://ww1.microchip.com/downloads/aemDocuments/documents/FPGA/ProductDocuments/UserGuides/ip_cores/directcores/CoreGPIO_HB.pdf}{CoreGPIO}
output data register is at offset address 0xA0, i.e., at address 0x4000\_01A0.
%
\item The LED1 to 7 CoreGPIO output register settings are:
\begin{itemize}
\item 0x4000\_01A0 = 0x01 (LED1 on) or 0x00 (LED1 off)
\item 0x4000\_01A0 = 0x02 (LED2 on) or 0x00 (LED2 off)
\item 0x4000\_01A0 = 0x04 (LED3 on) or 0x00 (LED3 off)
\item 0x4000\_01A0 = 0x08 (LED4 on) or 0x00 (LED4 off)
\item 0x4000\_01A0 = 0x10 (LED5 on) or 0x00 (LED5 off)
\item 0x4000\_01A0 = 0x20 (LED6 on) or 0x00 (LED6 off)
\item 0x4000\_01A0 = 0x40 (LED7 on) or 0x00 (LED7 off)
\end{itemize}
%
\item Use the Eclipse Memory View to add a new monitor at address 0x4000\_0100.
\item Change the byte at 0x4000\_01A0 to turn each of the LEDs on and off.
\end{itemize}
\textcolor{OliveGreen}{\bf Success! We have blinked LED1 through LED7 using CoreGPIO!}
\end{enumerate}
%
Understanding how to perform direct register accesses is a powerful debug tool.
Peripheral accesses are normally performed using a software API, eg.,
the System Tick function shown in Figure~\ref{fig:pfs_disco_reference_software_design}
blinks LED1 using the function call MSS\_GPIO\_set\_output().
If blinking the other LEDs using this function call does not work, then the
fact that direct register accesses have confirmed that all LEDs can be blinked
shows that any problem must be with the API usage.

% -----------------------------------------------------------------------------
\subsection{Software Modification}
% -----------------------------------------------------------------------------

The manual memory edits that determined the steps needed to toggle the LEDs
can be incorporated into the \texttt{mpfs-gpio-interrupt} by editing
\verb+u54_1.c+ as follows:
%
\begin{enumerate}
\item \textbf{Enable LED8 for output}

Insert one of the following code snippets
in the body of \verb+u54_1()+ at line 120:

\begin{enumerate}
\item Direct memory access
\begin{lstlisting}[style=c]
// LED8 (pin MSSIO23 = GPIO_1_(23-14) = GPIO_1_9)
*(volatile uint32_t *)0x20121024 = 0x05;
\end{lstlisting}
%
\item MSS GPIO API
\begin{lstlisting}[style=c]
// LED8 (pin MSSIO23 = GPIO_1_(23-14) = GPIO_1_9)
MSS_GPIO_init(GPIO1_LO);
MSS_GPIO_config(GPIO1_LO, MSS_GPIO_9, MSS_GPIO_OUTPUT_MODE);
\end{lstlisting}
\end{enumerate}

\item \textbf{Modify the System Tick code to generate an LED count}

Replace the System Tick code with one of the following:

\begin{enumerate}
\item Direct memory access
\begin{lstlisting}[style=c]
void SysTick_Handler_h1_IRQHandler(void)
{
    uint32_t hart_id = read_csr(mhartid);
    static volatile uint16_t value = 0u;
    uint8_t led, led1to7, led8;
    uint32_t gpio1_out;
    uint32_t gpio2_out;

    if (1u == hart_id)
    {
        value++;

        // Slow the LED count rate
        led = (value >> 2) & 0xFF;

        // LED8 and LED[1:7] settings
        led8    = led & 1;
//      led8    = (led >> 7) & 1;
        led1to7 = led & 0x7F;

        // LED8 output
        gpio1_out = led8 << 9;
        *(volatile uint32_t *)0x20121088 = gpio1_out;

        // LED[1:7] output via MSS GPIO
        gpio2_out = led1to7 << 17;
        *(volatile uint32_t *)0x20122088 = gpio2_out;

        // LED[1:7] output via CoreGPIO
//      gpio2_out = led1to7;
//      *(volatile uint32_t *)0x400001A0 = gpio2_out;
    }
}
\end{lstlisting}
%
\item MSS GPIO API
\begin{lstlisting}[style=c]
void SysTick_Handler_h1_IRQHandler(void)
{
    uint32_t hart_id = read_csr(mhartid);
    static volatile uint16_t value = 0u;
    uint8_t led, led1to7, led8;
    uint32_t gpio1_out;
    uint32_t gpio2_out;

    if (1u == hart_id)
    {
        value++;

        // Slow the LED count rate
        led = (value >> 2) & 0xFF;

        // LED8 and LED[1:7] settings
        led8    = led & 1;
//      led8    = (led >> 7) & 1;
        led1to7 = led & 0x7F;

        // LED8 output using single-bit API
//      MSS_GPIO_set_output(GPIO1_LO, MSS_GPIO_9, led8);

        // LED8 output using multi-bit API
        gpio1_out = led8 << 9;
        MSS_GPIO_set_outputs(GPIO1_LO, gpio1_out);

        // LED[1:7] output via MSS GPIO
        gpio2_out = led1to7 << 17;
        MSS_GPIO_set_outputs(GPIO2_LO, gpio2_out);
    }
}
\end{lstlisting}
%
\end{enumerate}
\end{enumerate}
%
After editing the \texttt{mpfs-gpio-interrupt} source, build the project, download in debug mode,
and resume the application. The LEDs on the board will show an incrementing count. The count rate
can be adjusted by editing the System Tick code to adjust the \verb+value+ right-shift value.
The LED8 blink rate in the code above is the same as LED1, but the code can be edited to have
LED8 blink as the MSB of an 8-bit LED count.

The modifications to \texttt{mpfs-gpio-interrupt} were a bit too hacky. Closer inspection of the
code shows several more calls to the MSS GPIO API for GPIO2. For example, the three GPIO IRQ
handlers each contain the call
\verb+MSS_GPIO_set_outputs(GPIO2_LO, 0u)+, and this clears all
bits in the fabric GPIO output register. An application that needs to independently control
the Raspberry Pi signals on GPIO\_2\_[0:16] and the LEDs on GPIO\_2\_[17:23] would need to
perform atomic read-modify-write accesses to the GPIO2 output register, eg., see
\href{https://github.com/polarfire-soc/polarfire-soc-documentation/blob/master/bare-metal-embedded-software/bare-metal-driver-user-guides/polarfire-soc-mss-driver-user-guides/mss-gpio/mss-gpio-driver-user-guide.md#example2}
{Example2}.


\clearpage
% =============================================================================
\section{Future Designs}
% =============================================================================
\label{sec:future}

The following are a list of ideas for future designs:
%
\begin{enumerate}
\item \textbf{Boot Linux}

Boot Linux.

\item \textbf{Boot vectors}

Write assembly code for the boot vector to blink the LEDs.

\item \textbf{HSS Payload}

Use the HSS to boot a user application, eg., as a first-stage bootloader.

\item \textbf{FIC AXI4 and AHB BFMs}

Demonstrate the use of the simulation BFMs, eg., start with the reference
hardware design, generate the BFM\_SIMULATION version, and blink the LEDs.

\item \textbf{AN5165: FIR Filter design}

Create a bit-true model of the digital filter and the FFT logic.
Use the simulation to understand the UART protocol.
Reimplement the simulation using CocoTB.

For the bit-true model, investigate the bit depths used in the design:
%
\begin{itemize}
\item 16-bits data times 16-bits coeffs = 32-bit products
\item 127 programmable coefficients have log2(127) = 7-bits growth
\item FIR filter worst-case output bit width is 32+7 = 39-bits
\item The FIR code show the output connects to FIRO[29:14], so that is
only 16-bits, but it is not the worst-case 16-bits.
\item Will this FIR saturate or wrap?
\item Is the FIR GUI scaling the coefficients and input signal to
accommodate the known bit-widths? (Review the text input and output view)
\end{itemize}

\item \textbf{FIR Filter design with RISC-V interface}

Reimplement the FIR filter design:
\begin{itemize}
\item AXI4-Stream interface FIR filter
\item AXI4-Stream interface FFT
\item AXI4-Stream FIFOs connecting components
\item Memory mapped FIFOs for input and output data
\item AXI4 or AHB control registers interface
\item DMA controller for input and output buffer loading and unloading
\end{itemize}
%
Implement the AXI4 or AHB control interface using the MSS BFMs.
Also implement the control using a UART-to-AXI4 or UART-to-AHB bridge.
That way this design could also work on the FPGA only kits.

\item \textbf{Heater}

Does the board have current sensors? How about the Video Kit?

Buy a USB-C interface/bridge that measures current. I would like the USB power
monitor to have a USB-C interface and I would like to be able to read the
measurements over USB (not just an LCD screen on the unit), so that I can
plot a power time-series.

\end{enumerate}



\clearpage
\appendix
% =============================================================================
\section{Resources}
% =============================================================================
\label{sec:resources}

This section contains resources found during the development of this tutorial.

\begin{itemize}
% ---------------------------------------
\item \textbf{Microchip Github}
% ---------------------------------------
%
\begin{itemize}
\item
\href{https://github.com/polarfire-soc/polarfire-soc-documentation/blob/master/README.md}
{PolarFire SoC Documentation Index}
%
\item
\href{https://github.com/polarfire-soc/polarfire-soc-documentation/blob/master/knowledge-base/polarfire-soc-software-tool-flow.md}
{PolarFire SoC Software Tool Flow}
%
\item
\href{https://github.com/polarfire-soc/polarfire-soc-documentation/blob/master/reference-designs-fpga-and-development-kits/mpfs-discovery-kit-embedded-software-user-guide.md}
{MPFS Discovery Kit Embedded Software User Guide}
%
\begin{itemize}
\item
This repo contains the high-resolution images from the User Guide (Rev 2 hardware).
\item
This repo contains instructions on updating the kit to support Linux and bare-metal applications.
\item
MMUART1 (1st COM port) displays the Hart Software Service (HSS) boot messages.
\item
MMUART4 (2nd COM port) displays U-Boot and Linux messages.
\item
The repo contains udev rules for the FTDI channels
\end{itemize}
%
\item
\href{https://github.com/polarfire-soc/polarfire-soc-documentation/blob/master/bare-metal-embedded-software/bare-metal-software-project-structure.md}
{Bare Metal Software Projects Structure}
%
\end{itemize}
%
% ---------------------------------------
\item \textbf{Microchip YouTube Channel}
% ---------------------------------------
%
\begin{itemize}
\item \href{https://www.youtube.com/user/MicrochipTechnology}{MicrochipTechnology} YouTube page
\item The Hugh Breslin videos are informative
\item
\href{https://www.youtube.com/playlist?list=PLtQdQmNK_0DSh2Mr18m8BNYRTK3sxeAPX}
{Bare Metal Examples} Playlist (6 videos)
\end{itemize}

% ---------------------------
\item \textbf{Discovery Kit}
% ---------------------------
%
\begin{itemize}
\item
\href{https://www.youtube.com/watch?v=GmitNBnw22I}
{PolarFire SoC Discovery Kit - Your Low-Cost Entry to RISC-V and FPGA Technology}
\begin{itemize}
\item
Tim McCarthy, Microchip
\item
11m35s total time
\item
This video is the introduction linked from the kit web page.
\item The kit shown in the video must be a Rev 1, as it uses a green screw
terminal for the alternate 5V power input, rather than the power socket used
on the Rev 2.
\item 2m55s: The factory restore process
\item The video shows factory restore from \verb+mpfs_an5165_v2023p2_df.zip+
\item The latest factor restore release is
\href{https://ww1.microchip.com/downloads/aemDocuments/documents/FPGA/SOCDesignFiles/mpfs_an5165_v2024p1_df.zip}
{\texttt{mpfs\_an5165\_v2024p1\_df.zip}}
\item 3m26s: FPExpress v2023.2 is used
\item Create a new FPExpress project for \verb+top.job+
\item Save the project in the same folder as the job file
\item Run PROGRAM to update the kit
\item Run VERIFY to show the kit now matches the job file
\item (Optional) Power cycle and run VERIFY
\item 4m30s: DSP FIR filter demo starts
\item Install the filter GUI and interact with the design
\item The filter GUI used the COM port associated with the fabric FTDI USB-UART
\item The GUI includes a text view of the input and output values
\end{itemize}

\clearpage
\item
\href{https://www.adiuvoengineering.com/post/microchip-discovery-board}
{Adam Taylor - Discovery Kit Bare-Metal Example} (1/2025)
\begin{itemize}
\item
The article uses the MSS Configurator file for the Discovery Kit
reference hardware design, but modifies it to delete all but the GPIO fabric
interface (GPIO 2). He makes a comment about this being necessary \emph{as
we need something in the fabric to ensure Synplify correctly synthesises the
project}. This comment likely reflects what was observed when I created
a processor-only system, in that Synplify fails as it has no nets to route.
\item
The design does not use the PFOSC\_INIT\_MONITOR component for reset
generation, rather he ties the reset input high.
\item Uses the \verb+mfps-gpio-interrupt+ bare-metal example for
the SoftConsole reference software design.
\item The article includes screen shots of the Hart 0 (e51) and 1 (u54\_1) UART console outputs.
\end{itemize}
%
\item
\href{https://www.controlpaths.com/2024/07/21/getting-started-mpfs-discovery}
{Pablo Trujillo - Getting started with the MPFS Discovery Kit} (7/2024)
\begin{itemize}
\item
The article uses the MSS Configurator file for the Discovery Kit
reference hardware design, but modifies it to disable some of the
interfaces (eg., the FIC interfaces).
\item
The MSS SmartDesign component instance screen shots show: REFCLK,
MMUART 1 and 4, SD card, SPI, DDR, GPIO\_1[9,20], and GPIO\_2[6:0].
\item GPIO\_2[6:0] are used to control LED[7:1].
\item The reset input is tied high.
\item Unused inputs are tied high, and unconnected outputs are marked as unused.
\item Uses the \verb+mfps-gpio-interrupt+ bare-metal example for
the SoftConsole reference software design.
\item The article includes a screen shot of the Hart 1 (u54\_1) UART console output.
\end{itemize}
%
\item
\href{https://www.hackster.io/news/hackster-s-fpgadventures-hands-on-with-the-low-cost-microchip-polarfire-soc-discovery-kit-e92d6224691d}
{Gareth Halfacree - Hands-On with the Low-Cost Microchip PolarFire SoC Discovery Kit}
\begin{itemize}
\item
This article summarizes the Discovery Kit hardware and software features.
\item
Gareth has another series of articles on the
\href{https://www.hackster.io/news/hackster-s-fpgadventures-unboxing-and-testing-the-microchip-polarfire-soc-icicle-kit-9f194a9639f6}
{PolarFire SoC Icicle Kit} (there are links to the whole series at the end of the first article).
\item Starware
\end{itemize}
%
\end{itemize}

% ---------------------------------------
\item \textbf{Numato PolarFire SoC SOM}
% ---------------------------------------

\href{https://numato.com/kb/hello-world-project-on-eaglecore-polarfire-soc-som}
{Hello World} project targeting the EagleCore PolarFire SoC SOM (3/2025)
%
\begin{itemize}
\item Uses MSS Configurator to configure the MSS subsystem
\item Uses Libero SoC to create a SmartDesign containing the MSS\newline
(the SmartDesign screen shot shows only UART and DDR dedicated pins), the
reset is exported to the top-level and assigned to pin H7, and the
design is synthesized
\item Uses SoftConsole to import the \verb+mpfs-mmuart-interrupt+ example
\item Update the project with the Numato PolarFire SoC XML file
\item Change the design source to reduce the number of processors used, to just the E51 core
\item Replace the E51 code with their custom code, which sends Hello World
using the polled transmit routine
\item The build configuration targets the \emph{eNVM Scratchpad-Release}
\item The application is programmed using non-secure boot mode 1
\end{itemize}

\newpage
% ---------------------------------------
\item \textbf{PolarFire SoC Icicle Kit}
% ---------------------------------------
%
\begin{itemize}
%
\item AC492: Running bare-metal applications~\cite{Microchip_AC492_2020}

This is an out-of-date application note written for the Icicle Kit. It describes
how to use System Services for reading the Device Serial Number and the Device Design ID.
It describes running bare-metal MicroPython (from DDR on a U54 RISC-V core).
%
\item
\href{https://www.controlpaths.com/2021/12/20/getting-started-with-microchips-fpga-icicle-kit-and-polarfire-soc}
{Pablo Trujillo - Getting started with Microchip's FPGA Icicle Kit and PolarFire SoC} (12/2022)
%
\begin{itemize}
\item Describes how to build the Icicle kit with the hardwre reference design.
\item Describes how to build the Hart Software Services (HSS) image.
\item Downloads and programs the reference design 2021.11 release zip file\newline
(contains the bitstream and HSS job file).
\item Downloads the Yocto SD card image, creates the card, and boots Linux.
\item Comments that the \verb+devmem+ tool is not available in the default Linux image.
\item Describes how to rebuild Linux using Yocto to include \verb+devmem+.
\end{itemize}
%
\item
\href{https://www.controlpaths.com/2022/10/10/creating-a-custom-polarfire-soc-design}
{Pablo Trujillo - Creating a custom PolarFire SoC design} (10/2022)
%
\begin{itemize}
\item
Shows how to create a basic hardware design targeting the Icicle kit.
\item
Contains an application that uses the Monitor Hart 0 (E51) and Hart 1 (U54\#1).
\item
The SmartDesign MSS screen shot shows: REFCLK, 4 MMUARTs, 4 LEDs, and SW4 connected
to the processor reset.
\item
Develops a bare-metal application based on \verb+mpfs-mmuart-interrupt+.
\item
Replaces the MSS XML in the bare-metal design with that created by the custom hardware design.
The reference design contains a Python script that parses the XML file to generate the
software support files.
\item
Creates customized application code that uses the E51 to read from the UART and then wake up the U54\#1 core.
\end{itemize}
%
\item
\href{https://www.controlpaths.com/2023/03/26/running-ubuntu-in-the-icicle-kit}
{Pablo Trujillo - Running Ubuntu in the Microchip's Icicle Kit} (3/2023)
%
\begin{itemize}
\item
Shows how to program the Icicle Kit with the hardware reference design, and how
to configured the kit to boot Ubuntu.
\end{itemize}
%
\end{itemize}

\end{itemize}





\clearpage
% -----------------------------------------------------------------------------
\section{Revision History}
% -----------------------------------------------------------------------------
%
\begin{table}[h]
%\caption{Revision History}
\begin{center}
\begin{tabular}{|c|p{100mm}|}
\hline
Date & Description\\
\hline\hline
&\\
07/25/2025  & Created the document.\\
07/29/2025  & Added the fabric blinky LED design.\\
08/02/2025  & Added the RISC-V blinky LED design.\\
&\\
\hline
\end{tabular}
\end{center}
\end{table}

%\clearpage
%------------------------------------------------------------------------------
% Do the bibliography
%------------------------------------------------------------------------------
%
% Note, you can't have spaces in the list of bibliography files
%
\bibliography{sections/refs}
\bibliographystyle{plain}

%------------------------------------------------------------------------------
\end{document}
