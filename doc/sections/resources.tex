% =============================================================================
\section{Resources}
% =============================================================================
\label{sec:resources}

This section contains resources found during the development of this tutorial.

\begin{itemize}
% ---------------------------------------
\item \textbf{Microchip Github}
% ---------------------------------------
%
\begin{itemize}
\item
\href{https://github.com/polarfire-soc/polarfire-soc-documentation/blob/master/README.md}
{PolarFire SoC Documentation Index}
%
\item
\href{https://github.com/polarfire-soc/polarfire-soc-documentation/blob/master/knowledge-base/polarfire-soc-software-tool-flow.md}
{PolarFire SoC Software Tool Flow}
%
\item
\href{https://github.com/polarfire-soc/polarfire-soc-documentation/blob/master/reference-designs-fpga-and-development-kits/mpfs-discovery-kit-embedded-software-user-guide.md}
{MPFS Discovery Kit Embedded Software User Guide}
%
\begin{itemize}
\item
This repo contains the high-resolution images from the User Guide (Rev 2 hardware).
\item
This repo contains instructions on updating the kit to support Linux and bare-metal applications.
\item
MMUART1 (1st COM port) displays the Hart Software Service (HSS) boot messages.
\item
MMUART4 (2nd COM port) displays U-Boot and Linux messages.
\item
The repo contains udev rules for the FTDI channels
\end{itemize}
%
\item
\href{https://github.com/polarfire-soc/polarfire-soc-documentation/blob/master/bare-metal-embedded-software/bare-metal-software-project-structure.md}
{Bare Metal Software Projects Structure}
%
\end{itemize}
%
% ---------------------------------------
\item \textbf{Microchip YouTube Channel}
% ---------------------------------------
%
\begin{itemize}
\item \href{https://www.youtube.com/user/MicrochipTechnology}{MicrochipTechnology} YouTube page
\item The Hugh Breslin videos are informative
\item
\href{https://www.youtube.com/playlist?list=PLtQdQmNK_0DSh2Mr18m8BNYRTK3sxeAPX}
{Bare Metal Examples} Playlist (6 videos)
\end{itemize}

% ---------------------------
\item \textbf{Discovery Kit}
% ---------------------------
%
\begin{itemize}
\item
\href{https://www.youtube.com/watch?v=GmitNBnw22I}
{PolarFire SoC Discovery Kit - Your Low-Cost Entry to RISC-V and FPGA Technology}
\begin{itemize}
\item
Tim McCarthy, Microchip
\item
11m35s total time
\item
This video is the introduction linked from the kit web page.
\item The kit shown in the video must be a Rev 1, as it uses a green screw
terminal for the alternate 5V power input, rather than the power socket used
on the Rev 2.
\item 2m55s: The factory restore process
\item The video shows factory restore from \verb+mpfs_an5165_v2023p2_df.zip+
\item The latest factor restore release is
\href{https://ww1.microchip.com/downloads/aemDocuments/documents/FPGA/SOCDesignFiles/mpfs_an5165_v2024p1_df.zip}
{\texttt{mpfs\_an5165\_v2024p1\_df.zip}}
\item 3m26s: FPExpress v2023.2 is used
\item Create a new FPExpress project for \verb+top.job+
\item Save the project in the same folder as the job file
\item Run PROGRAM to update the kit
\item Run VERIFY to show the kit now matches the job file
\item (Optional) Power cycle and run VERIFY
\item 4m30s: DSP FIR filter demo starts
\item Install the filter GUI and interact with the design
\item The filter GUI used the COM port associated with the fabric FTDI USB-UART
\item The GUI includes a text view of the input and output values
\end{itemize}

\clearpage
\item
\href{https://www.adiuvoengineering.com/post/microchip-discovery-board}
{Adam Taylor - Discovery Kit Bare-Metal Example} (1/2025)
\begin{itemize}
\item
The article uses the MSS Configurator file for the Discovery Kit
reference hardware design, but modifies it to delete all but the GPIO fabric
interface (GPIO 2). He makes a comment about this being necessary \emph{as
we need something in the fabric to ensure Synplify correctly synthesises the
project}. This comment likely reflects what was observed when I created
a processor-only system, in that Synplify fails as it has no nets to route.
\item
The design does not use the PFOSC\_INIT\_MONITOR component for reset
generation, rather he ties the reset input high.
\item Uses the \verb+mfps-gpio-interrupt+ bare-metal example for
the SoftConsole reference software design.
\item The article includes screen shots of the Hart 0 (e51) and 1 (u54\_1) UART console outputs.
\end{itemize}
%
\item
\href{https://www.controlpaths.com/2024/07/21/getting-started-mpfs-discovery}
{Pablo Trujillo - Getting started with the MPFS Discovery Kit} (7/2024)
\begin{itemize}
\item
The article uses the MSS Configurator file for the Discovery Kit
reference hardware design, but modifies it to disable some of the
interfaces (eg., the FIC interfaces).
\item
The MSS SmartDesign component instance screen shots show: REFCLK,
MMUART 1 and 4, SD card, SPI, DDR, GPIO\_1[9,20], and GPIO\_2[6:0].
\item GPIO\_2[6:0] are used to control LED[7:1].
\item The reset input is tied high.
\item Unused inputs are tied high, and unconnected outputs are marked as unused.
\item Uses the \verb+mfps-gpio-interrupt+ bare-metal example for
the SoftConsole reference software design.
\item The article includes a screen shot of the Hart 1 (u54\_1) UART console output.
\end{itemize}
%
\item
\href{https://www.hackster.io/news/hackster-s-fpgadventures-hands-on-with-the-low-cost-microchip-polarfire-soc-discovery-kit-e92d6224691d}
{Gareth Halfacree - Hands-On with the Low-Cost Microchip PolarFire SoC Discovery Kit}
\begin{itemize}
\item
This article summarizes the Discovery Kit hardware and software features.
\item
Gareth has another series of articles on the
\href{https://www.hackster.io/news/hackster-s-fpgadventures-unboxing-and-testing-the-microchip-polarfire-soc-icicle-kit-9f194a9639f6}
{PolarFire SoC Icicle Kit} (there are links to the whole series at the end of the first article).
\item Starware
\end{itemize}
%
\end{itemize}

% ---------------------------------------
\item \textbf{Numato PolarFire SoC SOM}
% ---------------------------------------

\href{https://numato.com/kb/hello-world-project-on-eaglecore-polarfire-soc-som}
{Hello World} project targeting the EagleCore PolarFire SoC SOM (3/2025)
%
\begin{itemize}
\item Uses MSS Configurator to configure the MSS subsystem
\item Uses Libero SoC to create a SmartDesign containing the MSS\newline
(the SmartDesign screen shot shows only UART and DDR dedicated pins), the
reset is exported to the top-level and assigned to pin H7, and the
design is synthesized
\item Uses SoftConsole to import the \verb+mpfs-mmuart-interrupt+ example
\item Update the project with the Numato PolarFire SoC XML file
\item Change the design source to reduce the number of processors used, to just the E51 core
\item Replace the E51 code with their custom code, which sends Hello World
using the polled transmit routine
\item The build configuration targets the \emph{eNVM Scratchpad-Release}
\item The application is programmed using non-secure boot mode 1
\end{itemize}

\newpage
% ---------------------------------------
\item \textbf{PolarFire SoC Icicle Kit}
% ---------------------------------------
%
\begin{itemize}
%
\item AC492: Running bare-metal applications~\cite{Microchip_AC492_2020}

This is an out-of-date application note written for the Icicle Kit. It describes
how to use System Services for reading the Device Serial Number and the Device Design ID.
It describes running bare-metal MicroPython (from DDR on a U54 RISC-V core).
%
\item
\href{https://www.controlpaths.com/2021/12/20/getting-started-with-microchips-fpga-icicle-kit-and-polarfire-soc}
{Pablo Trujillo - Getting started with Microchip's FPGA Icicle Kit and PolarFire SoC} (12/2022)
%
\begin{itemize}
\item Describes how to build the Icicle kit with the hardwre reference design.
\item Describes how to build the Hart Software Services (HSS) image.
\item Downloads and programs the reference design 2021.11 release zip file\newline
(contains the bitstream and HSS job file).
\item Downloads the Yocto SD card image, creates the card, and boots Linux.
\item Comments that the \verb+devmem+ tool is not available in the default Linux image.
\item Describes how to rebuild Linux using Yocto to include \verb+devmem+.
\end{itemize}
%
\item
\href{https://www.controlpaths.com/2022/10/10/creating-a-custom-polarfire-soc-design}
{Pablo Trujillo - Creating a custom PolarFire SoC design} (10/2022)
%
\begin{itemize}
\item
Shows how to create a basic hardware design targeting the Icicle kit.
\item
Contains an application that uses the Monitor Hart 0 (E51) and Hart 1 (U54\#1).
\item
The SmartDesign MSS screen shot shows: REFCLK, 4 MMUARTs, 4 LEDs, and SW4 connected
to the processor reset.
\item
Develops a bare-metal application based on \verb+mpfs-mmuart-interrupt+.
\item
Replaces the MSS XML in the bare-metal design with that created by the custom hardware design.
The reference design contains a Python script that parses the XML file to generate the
software support files.
\item
Creates customized application code that uses the E51 to read from the UART and then wake up the U54\#1 core.
\end{itemize}
%
\item
\href{https://www.controlpaths.com/2023/03/26/running-ubuntu-in-the-icicle-kit}
{Pablo Trujillo - Running Ubuntu in the Microchip's Icicle Kit} (3/2023)
%
\begin{itemize}
\item
Shows how to program the Icicle Kit with the hardware reference design, and how
to configured the kit to boot Ubuntu.
\end{itemize}
%
\end{itemize}

\end{itemize}




