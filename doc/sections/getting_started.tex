% =============================================================================
\section{Getting Started}
% =============================================================================
\label{sec:getting_started}

The Microchip PolarFire SoC
\href{https://www.microchip.com/en-us/development-tool/mpfs-disco-kit}
{Discovery Kit} web page has getting started resources:
%
\begin{itemize}
\item Discovery Kit Quick Start Guide~\cite{Microchip_DISCO_QS_2024}

The Quick Start Guide contains photos of the kit that identify the components
on the board (photos are of the Rev 1 design), and provide the recommended
jumper settings. The guide details how to use the factory default FIR filter
and FFT demonstration design and FIR filter Windows GUI.

\item Discovery Kit User Guide~\cite{Microchip_DISCO_UG_2025}

The User Guide contains photos of the kit that identify the components
on the board (photos are of the Rev 2 design), and provide the recommended
jumper settings. The User Guide contains details on the components and
the interfaces. Appendix B contains tables of pin assignments, but no recommended
settings, eg., I/O standard, drive strength, or slew rate.

\item Discovery Kit Schematic~\cite{Microchip_DISCO_SCH_2023}

The schematic provides critical details on bank voltages, pin assignments,
device reset generation, clocks, and interface connections.

The Microchip web site does not include the Cadence Capture design file (.DSN)
file. This file is provided with the Video Kit and is useful for automated
pin and net name checking (using Cadence Capture Tcl scripts). The design
file has been requested from Microchip.

\item FIR filter and FFT demonstration design~\cite{Microchip_AN5165_2024}

The factory default design. The design does not include details on the UART
communications protocol or a bit-true model of the signal processing logic.
The design does include a testbench, so it would be possible to reverse-engineer
the UART protocol and to develop a bit-true model.

\item Libero SoC design flow~\cite{Microchip_AN5282_2024}

Demonstrates a Libero SoC SmartDesign for an AND gate. The AND gate inputs are
connected to the two push button switches, while the output is connected to
LED1.

\textcolor{magenta}{This document does not demonstrate how to instantiate the RISC-V processors.}


\item \href{https://github.com/polarfire-soc/polarfire-soc-discovery-kit-reference-design}{Discovery Kit reference design}

\textcolor{red}{TODO:} Add a chapter exploring the designs in this repo.

This is the hardware reference design on which the bare-metal applications can run.

\item \href{https://github.com/polarfire-soc/polarfire-soc-bare-metal-examples}{Bare-metal examples}

\textcolor{red}{TODO:} Add a chapter exploring the designs in this repo.

\item Linux examples
\item Links to videos

See Appendix~\ref{sec:resources}.

\end{itemize}
%
This tutorial references these resources (and others).

\clearpage
% -----------------------------------------------------------------------------
\subsection{Factory Restore}
% -----------------------------------------------------------------------------
\label{sec:factory_restore}

The Discovery Kit FIR filter example design is programmed (or verified) as follows:
%
\begin{enumerate}
\item Download and unzip the FIR filter design~\cite{Microchip_AN5165_2024}
%
\begin{itemize}
\item For example, unzip to
\begin{verbatim}
C:\temp\mpfs_an5165_v2024p1_df
\end{verbatim}
\end{itemize}
%
\item Power the Discovery Kit using a USB-C cable
\item Start \emph{FPExpress v2025.1}
\item Select \emph{Project$\rightarrow$New Job Project}
\item \emph{Create New Job Project} dialog:
%
\begin{itemize}
\item For \emph{Import FlashPro Express job file}, browse
and select the job file, eg.,
\begin{verbatim}
C:\temp\mpfs_an5165_v2024p1_df\Programming_Job
\end{verbatim}
\item For \emph{FlashPro Express job project location}, browse
and select the directory as the job file.
\item Click \emph{OK}
\end{itemize}
%
\item \textbf{Program}
\begin{itemize}
\item The action pull-down menu defaults to \emph{PROGRAM}
\item Click the \emph{RUN} button to run the \emph{PROGRAM} action
\item Figure~\ref{fig:pfs_disco_fpexpress}(a) shows the FlashPro Express GUI after program
\end{itemize}
%
\item \textbf{Verify}
\begin{itemize}
\item Change the action pull-down menu to \emph{VERIFY}
\item Click the \emph{RUN} button to run the \emph{VERIFY} action
\item Figure~\ref{fig:pfs_disco_fpexpress}(b) shows the FlashPro Express GUI after verify
\end{itemize}
%
\item Save and exit the FlashPro Express application
\end{enumerate}
%
The Discovery Kit User Guide has details on the embedded FlashPro5
programmer (App. A~\cite{Microchip_DISCO_UG_2025}).

\vskip5mm
\begin{center}
\fbox{\parbox{0.85\linewidth}{
\textbf{\textcolor{red}{WARNING:}}\newline
The FIR filter example design uses only the FPGA fabric, so cannot
be used for RISC-V software development!}}
\end{center}

% -----------------------------------------------------------------------------
% FlashPro Express
% -----------------------------------------------------------------------------
%
\begin{figure}[p]
  \begin{center}
    \includegraphics[width=0.95\textwidth]
    {figures/pfs_disco_fpexpress_program.png}\\
    (a) PROGRAM\\
    \vskip5mm
    \includegraphics[width=0.95\textwidth]
    {figures/pfs_disco_fpexpress_verify.png}\\
    (b) VERIFY
  \end{center}
  \caption{FlashPro Express GUI after program and verify.}
  \label{fig:pfs_disco_fpexpress}
\end{figure}

\clearpage
% -----------------------------------------------------------------------------
% FIR Filter GUI screen shots
% -----------------------------------------------------------------------------
%
\begin{figure}[t]
  \begin{center}
    \includegraphics[width=0.85\textwidth]
    {figures/an5165_filter_gui_a.png}
    \vskip5mm
  \end{center}
  \caption{FIR filter GUI: filter coefficients.}
  \label{fig:an5165_filter_gui_a}
\end{figure}

% -----------------------------------------------------------------------------
\subsection{FIR Filter Demonstration}
% -----------------------------------------------------------------------------

%
\begin{enumerate}
\item Install the FIR Filter GUI
%
\begin{itemize}
\item Run the installer:
\begin{verbatim}
C:\temp\mpfs_an5165_v2024p1_df\GUI_Installer\Volume\setup.exe
\end{verbatim}
\item Install directories:
\begin{itemize}
\item GUI: \verb+C:\software\Microchip\DISCO-KIT\FIR_Filter_GUI+
\item National Instruments: \verb+C:\Program Files (x86)\National Instruments+
\end{itemize}
\item The software can be uninstalled using Windows add/remove programs
\end{itemize}
%
\item Power the Discovery Kit using a USB-C cable
%
\item Run the FIR Filter GUI
\begin{itemize}
\item \emph{Start Menu$\rightarrow$Filter\_GUI}
\item Click on the link icon and establish the connection to the kit
\item Click on \emph{Generate Filter}
\item Figure~\ref{fig:an5165_filter_gui_a} shows the filter coefficients
\item Click on \emph{Generate Signal}
\item Figure~\ref{fig:an5165_filter_gui_b} shows the input signal
\item Click on the play button
\item Figure~\ref{fig:an5165_filter_gui_c} shows the filtered output signal
\end{itemize}
\end{enumerate}
%
AN5165 contains additional screen shots and details~\cite{Microchip_AN5165_2024}.

\clearpage
% -----------------------------------------------------------------------------
% FIR Filter GUI screen shots
% -----------------------------------------------------------------------------
%
\begin{figure}[p]
  \begin{center}
    \includegraphics[width=0.85\textwidth]
    {figures/an5165_filter_gui_b.png}
    \vskip5mm
  \end{center}
  \caption{FIR filter GUI: input signal.}
  \label{fig:an5165_filter_gui_b}
\end{figure}

\begin{figure}[p]
  \begin{center}
    \includegraphics[width=0.85\textwidth]
    {figures/an5165_filter_gui_c.png}
    \vskip5mm
  \end{center}
  \caption{FIR filter GUI: output signal.}
  \label{fig:an5165_filter_gui_c}
\end{figure}

\clearpage
% -----------------------------------------------------------------------------
% Discovery Kit Reference Designs
% -----------------------------------------------------------------------------
%
\begin{table}
\caption{Discovery Kit reference hardware designs.}
\label{tab:pfs_disco_hardware_designs}
\begin{center}
\begin{tabular}{|l|p{100mm}|}
\hline
Script Argument & Description\\
\hline\hline
&\\
(no argument)   & Discovery Kit reference hardware design\\
FIR\_DEMO       & AN5165 Filter demonstration\\
BFM\_SIMULATION & SmartDesign testbench with FIC interface BFMs\\
I2C\_LOOPBACK   & I2C loopback via the fabric\\
MIV\_RV32\_CFG1 & Mi-V softcore RISC-V example 1\\
MIV\_RV32\_CFG2 & Mi-V softcore RISC-V example 2\\
MIV\_RV32\_CFG3 & Mi-V softcore RISC-V example 3\\
&\\
\hline
\end{tabular}
\end{center}
\end{table}

% -----------------------------------------------------------------------------
\subsection{Discovery Kit Reference Hardware Designs}
% -----------------------------------------------------------------------------

The Discovery Kit reference hardware designs are available on github:
%
\begin{quote}
\href{https://github.com/polarfire-soc/polarfire-soc-discovery-kit-reference-design}
{https://github.com/polarfire-soc/polarfire-soc-discovery-kit-reference-design}
\end{quote}
%
Table~\ref{tab:pfs_disco_hardware_designs} shows the different hardware designs.
The hardware designs are built using a Libero Tcl script, with the different
designs selected based on a script argument. The repository README.md
is recommended reading. The README.md contains a block diagram of the
reference hardware design, and information on building the design variants.

Section~\ref{sec:riscv_blinky} demonstrates bare-metal software running on the
reference hardware design.

\vskip10mm
\begin{center}
\fbox{\parbox{0.85\linewidth}{
\textbf{\textcolor{red}{IMPORTANT:}}\newline
The Discovery Kit must be programmed with the reference hardware design
before using the bare-metal or Linux software examples.}}
\end{center}
\vskip10mm

% -----------------------------------------------------------------------------
\subsection{Bare-Metal Software Examples}
% -----------------------------------------------------------------------------

The PolarFire SoC bare-metal software examples are available on github:
%
\begin{quote}
\href{https://github.com/polarfire-soc/polarfire-soc-bare-metal-examples}
{https://github.com/polarfire-soc/polarfire-soc-bare-metal-examples}
\end{quote}
%
The bare-metal examples were developed for the PolarFire SoC Icicle Kit.
%
The examples ported to the Discovery Kit are:
%
\begin{itemize}
\item mss-gpio: \href{https://github.com/polarfire-soc/polarfire-soc-bare-metal-examples/blob/main/driver-examples/mss/mss-gpio/mpfs-gpio-interrupt}{mpfs-gpio-interrupt}
\item mss-mmuart: \href{https://github.com/polarfire-soc/polarfire-soc-bare-metal-examples/blob/main/driver-examples/mss/mss-mmuart/mpfs-mmuart-interrupt}{mpfs-mmuart-interrupt}
\item mss-rtc: \href{https://github.com/polarfire-soc/polarfire-soc-bare-metal-examples/blob/main/driver-examples/mss/mss-rtc/mpfs-rtc-time}{mpfs-rtc-time}
\end{itemize}
%
Section~\ref{sec:riscv_blinky} demonstrates the use of the GPIO and MMUART
examples.

\clearpage
% -----------------------------------------------------------------------------
\subsection{Linux}
% -----------------------------------------------------------------------------

Section~\ref{sec:linux_boot} provides the procedure to program the Discovery Kit,
create an SD card, and boot Linux.

\vskip5mm
\noindent\textcolor{red}{TODO:}
\begin{itemize}
\item Describe the boot sequence:
\begin{itemize}
\item HSS = First-stage boot-loader
\item U-Boot = Second-stage bootloader
\item Linux
\end{itemize}
\item Blink the LEDs:
\begin{itemize}
\item Using HSS commands (are there memory commands?)
\item Using U-Boot commands (see Section~\ref{sec:linux_boot})
\item Using /dev/mem and a Linux application\newline
(the \verb+devmem+ tool does not exist in the default Yocto image)
\end{itemize}
\end{itemize}

% -----------------------------------------------------------------------------
\subsection{PolarFire SoC Documentation}
% -----------------------------------------------------------------------------

The PolarFire SoC github organization and documentation repository have a
lot of information:
%
\begin{itemize}
\item
\href{https://github.com/polarfire-soc}
{https://github.com/polarfire-soc}
\item
\href{https://github.com/polarfire-soc/polarfire-soc-documentation}
{https://github.com/polarfire-soc/polarfire-soc-documentation}
\end{itemize}
%
Recommended reading:
%
\begin{itemize}
\item
\href{https://github.com/polarfire-soc/polarfire-soc-documentation/tree/master/bare-metal-embedded-software}
{bare-metal-embedded-software}
\item
\href{https://github.com/polarfire-soc/polarfire-soc-documentation/blob/master/bare-metal-embedded-software/bare-metal-software-project-structure.md}
{bare-metal-software-project-structure.md}
\item
\href{https://github.com/polarfire-soc/polarfire-soc-documentation/blob/master/bare-metal-embedded-software/bare-metal-driver-user-guides}
{bare-metal-driver-user-guides}
\item
\href{https://github.com/polarfire-soc/polarfire-soc-documentation/blob/master/reference-designs-fpga-and-development-kits/mpfs-discovery-kit-embedded-software-user-guide.md}
{mpfs-discovery-kit-embedded-software-user-guide.md}
\item
\href{https://github.com/polarfire-soc/polarfire-soc-documentation/blob/master/knowledge-base/boot-modes/boot-modes-fundamentals.md}
{boot-modes-fundamentals.md}
\item
\href{https://github.com/polarfire-soc/polarfire-soc-documentation/blob/master/reference-designs-fpga-and-development-kits/updating-linux-in-mpfs-kit.md}
{updating-linux-in-mpfs-kit.md}
\end{itemize}
%
Support:
%
\begin{itemize}
\item
\href{https://github.com/orgs/polarfire-soc/discussions}
{https://github.com/orgs/polarfire-soc/discussions}
\end{itemize}

