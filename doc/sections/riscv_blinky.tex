% =============================================================================
\section{Example 2: RISC-V plus Fabric Blinky LEDs}
% =============================================================================
\label{sec:riscv_blinky}

This section demonstrations how to use the Discovery Kit reference hardware
and software to blink all eight LEDs using the GPIO registers in the MSS.
%
Figure~\ref{fig:pfs_disco_reference_hardware_design} shows the
\href{https://github.com/polarfire-soc/polarfire-soc-discovery-kit-reference-design}
{PolarFire SoC Discovery Kit Reference Hardware Design.}
%
The RISC-V Blinky LEDs application uses the following GPIOs shown in the figure:
%
\begin{itemize}
\item \texttt{LED8} is driven by MSS \texttt{GPIO\_1\_9} at address \texttt{0x2012\_1000}
\begin{itemize}
\item
Figure~\ref{fig:pfs_disco_reference_hardware_mss_cfg_gpio_1} shows the MSS Configurator
GPIO\_1 settings.
\end{itemize}
%
\item \texttt{LED[1:7]} are driven by the OR of:
\begin{itemize}
\item MSS \texttt{GPIO\_2\_[17:23]} at address \texttt{0x2012\_1000}
\item FIC3 CoreGPIO at address \texttt{0x4000\_0000}
\end{itemize}
\end{itemize}
%
The PolarFire GPIO registers addresses are defined in the PolarFire SoC
Registers Map which can be downloaded via the zip file linked from
page 20 of the Technical Reference Manual~\cite{Microchip_PFSoC_TRM_2025}.

% -----------------------------------------------------------------------------
\subsection{MSS I/O pin number to MSS GPIO software API mapping}
% -----------------------------------------------------------------------------

The mapping from the LED8 schematic pin name to the MSS GPIO software API
arguments required some analysis. The Discovery Kit schematic shows that LED8
connects to pin number E1 with pin name \textsf{MSSIO23B2}, i.e., MSS I/O 23 in
bank 2 (p7~\cite{Microchip_DISCO_SCH_2023}).
%
\begin{center}
\textcolor{magenta}{\bf How does MSS I/O 23 in Bank 2 become GPIO\_1\_9?}
\end{center}
%
The PolarFire SoC MSS has 38 general purpose I/O pads called MSS I/Os (MSSIO),
with 14 pins in Bank 4 and 24 pins in Bank 2 (p48~\cite{Microchip_PFSoC_TRM_2025}).
%
The Discovery Kit schematic show how these 38 MSS I/O pins map to schematic
symbol banks, pin numbers, and pin names, while the MSS Configurator GUI shows how
the 38 MSS I/O pins map to banks, pin numbers, and software registers.
%
The Discovery Kit schematic~\cite{Microchip_DISCO_SCH_2023} page 4 shows
that Bank 4 contains pin names \textsf{MSSIO0B4} through \textsf{MSSIO13B4},
i.e., MSS I/Os \textsf{MSSIO[0:13]}, while page 7 shows that Bank 2
contains pin names \textsf{MSSIO14B2} through \textsf{MSSIO37B2}, i.e.,
MSS I/Os \textsf{MSSIO[14:37]}.

Figure~\ref{fig:pfs_disco_reference_hardware_mss_cfg_gpio_1} shows the
MSS Configurator \emph{Peripherals} tab with the GPIO\_1 settings
selected. The first two columns of the table are \textbf{BANK} and
\textbf{IO MUX}, where \textbf{BANK} is the bank number and \textbf{IO MUX}
is the MSS I/O index.
%
The Bank 4 MSS I/O bits \textsf{MSSIO[0:13]} map to MSS GPIO control
register bits \textsf{GPIO\_0\_[0:13]}, while
the Bank 2 MSS I/O bits \textsf{MSSIO[14:37]} map to MSS GPIO control
register bits \textsf{GPIO\_1\_[0:27]}. This provides the answer to
the question posed above:
%
\begin{center}
\textcolor{magenta}{\bf MSS I/O 23 in Bank 2 = GPIO\_1 bit (23 - 14) = GPIO\_1\_9}
\end{center}

% -----------------------------------------------------------------------------
\subsection{MSS GPIO software API documentation}
% -----------------------------------------------------------------------------

The System Tick function shown in Figure~\ref{fig:pfs_disco_reference_software_design}
blinks LED1 using the MSS GPIO software API call MSS\_GPIO\_set\_output().
%
The bare-metal driver API is embedded in the source code (using doxygen-like
annotations), and the source is used to generate an API user guide in the
\href{https://github.com/polarfire-soc/polarfire-soc-documentation}
{PolarFire SoC Documentation} github repo.
%
The
\href{https://github.com/polarfire-soc/polarfire-soc-documentation/blob/master/bare-metal-embedded-software/bare-metal-driver-user-guides/polarfire-soc-mss-driver-user-guides/mss-gpio/mss-gpio-driver-user-guide.md}
{MSS GPIO Bare Metal Driver} contains the MSS GPIO software API
documentation. Relevant API calls for the blinky LED software are:
%
\begin{itemize}
\item
\href{https://github.com/polarfire-soc/polarfire-soc-documentation/blob/master/bare-metal-embedded-software/bare-metal-driver-user-guides/polarfire-soc-mss-driver-user-guides/mss-gpio/mss-gpio-driver-user-guide.md#mss_gpio_init}
{MSS\_GPIO\_init()}
\item
\href{https://github.com/polarfire-soc/polarfire-soc-documentation/blob/master/bare-metal-embedded-software/bare-metal-driver-user-guides/polarfire-soc-mss-driver-user-guides/mss-gpio/mss-gpio-driver-user-guide.md#mss_gpio_config}
{MSS\_GPIO\_config()}
\item
\href{https://github.com/polarfire-soc/polarfire-soc-documentation/blob/master/bare-metal-embedded-software/bare-metal-driver-user-guides/polarfire-soc-mss-driver-user-guides/mss-gpio/mss-gpio-driver-user-guide.md#mss_gpio_set_output}
{MSS\_GPIO\_set\_output()}
\item
\href{https://github.com/polarfire-soc/polarfire-soc-documentation/blob/master/bare-metal-embedded-software/bare-metal-driver-user-guides/polarfire-soc-mss-driver-user-guides/mss-gpio/mss-gpio-driver-user-guide.md#mss_gpio_set_outputs}
{MSS\_GPIO\_set\_outputs()}
\end{itemize}

\clearpage
% -----------------------------------------------------------------------------
\subsection{Reference Hardware}
% -----------------------------------------------------------------------------

The Discovery Kit Reference Hardware Design is built as follows:
%
\begin{enumerate}
\item Clone the github repository

For example, using Windows 10 WSL
%
\begin{verbatim}
$ cd /mnt/c/github
$ git clone https://github.com/polarfire-soc/
      polarfire-soc-discovery-kit-reference-design.git
\end{verbatim}

\item Start Libero SoC 2025.1

\item Use \emph{Project$\rightarrow$Execute Script} to select and run
%
\begin{verbatim}
c:/github/polarfire-soc-discovery-kit-reference-design/
      MPFS_DISCOVERY_KIT_REFERENCE_DESIGN.tcl
\end{verbatim}

The Libero SoC GUI updates repeatedly as the script creates the SmartDesign.

The script run-time is about 5 minutes.

\item Click on the green play button to run synthesis and place-and-route

The synthesis and place-and-route run-time is about 10 minutes.

\item Double-mouse-click on \emph{Export FlashPro Express Job}

Make a note of the output job directory as this is needed with FlashPro.

\item Exit Libero

\end{enumerate}
%
Use FlashPro to program the Discovery Kit following the instructions in
Section~\ref{sec:factory_restore}, but use the job file just created.
After programming with the reference hardware design, the Discovery Kit
LED8 will be on (red) and LED1 through LED7 will be off. The next
sections describe how to blink each of the LEDs.

% -----------------------------------------------------------------------------
% Reference hardware design
% -----------------------------------------------------------------------------
%
\begin{landscape}
\begin{figure}[p]
  \begin{center}
    \includegraphics[width=200mm]
    {figures/pfs_disco_reference_hardware_design.pdf}
  \end{center}
  \caption{PolarFire SoC Discovery Kit reference hardware design.}
  \label{fig:pfs_disco_reference_hardware_design}
\end{figure}
\end{landscape}
% -----------------------------------------------------------------------------

% -----------------------------------------------------------------------------
% MSS Configurator for GPIO_1
% -----------------------------------------------------------------------------
%
\begin{landscape}
\begin{figure}[p]
  \begin{center}
    \includegraphics[width=215mm]
    {figures/pfs_disco_reference_hardware_mss_cfg_gpio_1.png}
  \end{center}
  \caption{Discovery Kit MSS Configurator GPIO\_1 settings.}
  \label{fig:pfs_disco_reference_hardware_mss_cfg_gpio_1}
\end{figure}
\end{landscape}
% -----------------------------------------------------------------------------

% -----------------------------------------------------------------------------
\subsection{Reference Software}
% -----------------------------------------------------------------------------

The Discovery Kit Reference Software Design is built as follows:
%
\begin{enumerate}
\item Clone the github bare-metal examples repository

For example, using Windows 10 WSL
%
\begin{verbatim}
$ cd /mnt/c/github
$ git clone https://github.com/polarfire-soc/
      polarfire-soc-bare-metal-examples
\end{verbatim}

\item Start SoftConsole 2022.2

\item Create a new Eclipse workspace, eg.,

\begin{verbatim}
C:\github\polarfire-soc-bare-metal-examples\ws-mss-gpio
\end{verbatim}

\item Close the Welcome window

\item Select \emph{File$\rightarrow$Import} and then \emph{Existing Projects into Workspace}

\item Click \emph{Next}

\item Point the root directory at the bare-metal examples github repo

\item Click \emph{Deselect All}

\item Check \texttt{mpfs-gpio-interrupt}

\item Click \emph{Finish}

\item Use the \emph{Project Explorer} to view the source

Figure~\ref{fig:pfs_disco_reference_software_design} shows the SoftConsole GUI
with the source code for hardware thread (hart) 1, i.e., \verb+u54_1.c+.
The source code shown in the figure is the System Tick interrupt handle,
which blinks LED1.

\item Configure the build configuration and build the project
%
\begin{itemize}
\item Select \emph{Project$\rightarrow$Build Configurations$\rightarrow$Set Active}
\item Select \emph{LIM-Debug-DiscoveryKit}
\item This configuration uses boot mode 0
\item Select \emph{Project$\rightarrow$Build All}
\end{itemize}

\item Download to the Discovery Kit
%
\begin{itemize}
\item Power the Discovery Kit using a USB-C cable
\item Select \emph{Run$\rightarrow$Debug Configurations}
\item Select \emph{mpfs-gpio-interrupt hw all-harts debug}
\item Change the build configuration from \emph{Use Active} to \emph{LIM-Debug-DiscoveryKit}
\item Click \emph{Apply}
\item Click \emph{Debug}
\end{itemize}

% -----------------------------------------------------------------------------
% Reference software design
% -----------------------------------------------------------------------------
%
\begin{landscape}
\begin{figure}[p]
  \begin{center}
    \includegraphics[width=200mm]
    {figures/pfs_disco_softconsole_u54_systick.png}
  \end{center}
  \caption{PolarFire SoC Discovery Kit reference software design.}
  \label{fig:pfs_disco_reference_software_design}
\end{figure}
\end{landscape}
% -----------------------------------------------------------------------------

\newpage
\item Resume and then pause the application
%
\begin{itemize}
\item The debug view opens with line 33 highlighted
\item Press the green play button to Resume (run the application)
\item The orange LED1 (near the corner of the board) will start blinking
\end{itemize}
\end{enumerate}
%
\textcolor{OliveGreen}{\bf Success! We have blinked LED1!}

% -----------------------------------------------------------------------------
\subsection{Software Debug}
% -----------------------------------------------------------------------------

An important part of software development is using a debugger to confirm your
understanding of the address map of your processor and hardware design.
%
The fact that LED1 is blinking means that the GPIO example has configured the
MSS GPIO registers correctly. That means we should be able to toggle all of
the LEDs using MSS GPIO registers. We can confirm our understanding of the
address map without writing a line of code as follows:
%
\begin{enumerate}
\item In the SoftConsole GUI, click on the Suspend (pause) button

\item Change the GUI to the debug perspective (click the bug on the right-hand-side)

\item Click on the \emph{Memory} tab near the bottom of the GUI.

This debug feature is the
\href{https://help.eclipse.org/latest/index.jsp?topic=%2Forg.eclipse.cdt.doc.user%2Freference%2Fcdt_u_memoryview.htm}
{Eclipse Memory View}.

\item \textbf{Toggle LED8}
%
\begin{itemize}
\item The Discovery Kit schematic shows LED8 connected to MSSIO23, pin E1, Bank 2 (p7~\cite{Microchip_DISCO_SCH_2023})
\item Open the MSS Configurator for the hardware reference design.
\item Select the \emph{Peripherals} tab.
\item The right-side of the GUI shows IO MUX bit 32, Bank 2, E1, configured for GPIO\_1\_9.
\item Figure~\ref{fig:pfs_disco_reference_hardware_mss_cfg_gpio_1} shows the MSS Configurator GUI after
selecting \emph{GPIO\_1 (Bank 2 I/Os)}.
\item The left-side of the GUI shows the pull-down menus for GPIO\_1\_9 configured for
\emph{MSS I/Os Bank2} with the direction set to \emph{Output}.
\item The PolarFire SoC Registers Map contains the GPIO\_1 register definitions under
the link GPIO\_IOBANK1\_LO.
\item The LED8 configuration register settings are:
\begin{itemize}
\item CONFIG\_9[0] = EN\_OUT = 1
\item CONFIG\_9[1] = EN\_OE\_BUF = 1
\item 0x2012\_1024 = 0x05
\end{itemize}
%
\item The LED8 output register settings (GPOUT) are:
\begin{itemize}
\item GPOUT[9] = 1 (LED on) or 0 (LED off)
\item 0x2012\_1089 = 0x02 (LED on) or 0x00 (LED off)
\end{itemize}
%
\item Use the Eclipse Memory View to add a new monitor at address 0x2012\_1000.
\item Change byte 0x2012\_1024 to 0x05.
\item LED8 will turn off.
\item Change byte 0x2012\_1089 to 0x02 and then 0x00.
\item LED8 will turn on and off.
\end{itemize}
\textcolor{OliveGreen}{\bf Success! We have blinked LED8!}

\newpage
\item \textbf{Toggle LED1 to LED7 using MSS GPIO}
%
\begin{itemize}
\item Figure~\ref{fig:pfs_disco_reference_hardware_design} shows that the
Discovery Kit hardware reference design controls LED[1:7] using
MSS GPIO\_2\_[17:23].
\item Open the MSS Configurator for the hardware reference design.
\item Select the \emph{Peripherals} tab.
\item Selecting \emph{GPIO\_2 (Fabric)} on the left-side of the GUI shows the pull-down
menus for GPIO\_2\_17 to 23 configured for \emph{Fabric I/O} with the direction set to \emph{Output}.
\item The PolarFire SoC Registers Map contains the GPIO\_2 register definitions under
the link GPIO\_FAB\_LO.
\item The LED1 to LED7 configuration register settings do not matter, as the
hardware reference design only connects to the GPIO outputs (not the output enables).
%
\item The LED1 to 7 output register (GPOUT) settings are:
\begin{itemize}
\item GPOUT[n] = 1 (LED on) or 0 (LED off) for n = 17, 18, 19, 20, 21, 22, 23
\item 0x2012\_208A = 0x02 (LED1 on) or 0x00 (LED1 off)
\item 0x2012\_208A = 0x04 (LED2 on) or 0x00 (LED2 off)
\item 0x2012\_208A = 0x08 (LED3 on) or 0x00 (LED3 off)
\item 0x2012\_208A = 0x10 (LED4 on) or 0x00 (LED4 off)
\item 0x2012\_208A = 0x20 (LED5 on) or 0x00 (LED5 off)
\item 0x2012\_208A = 0x40 (LED6 on) or 0x00 (LED6 off)
\item 0x2012\_208A = 0x80 (LED7 on) or 0x00 (LED7 off)
\end{itemize}
%
\item Use the Eclipse Memory View to add a new monitor at address 0x2012\_2000.
\item Change the byte at 0x2012\_108A to turn each of the LEDs on and off.
\item Turn the LEDs off before performing the next steps.
\end{itemize}
\textcolor{OliveGreen}{\bf Success! We have blinked LED1 through LED7 using MSS GPIO!}

\item \textbf{Toggle LED1 to LED7 using FIC\_3 CoreGPIO}
%
\begin{itemize}
\item Figure~\ref{fig:pfs_disco_reference_hardware_design} shows that the
Discovery Kit hardware reference design can also control LED[1:7] using
the CoreGPIO at base address 0x4000\_0100.
%
\item The \href{https://ww1.microchip.com/downloads/aemDocuments/documents/FPGA/ProductDocuments/UserGuides/ip_cores/directcores/CoreGPIO_HB.pdf}{CoreGPIO}
output data register is at offset address 0xA0, i.e., at address 0x4000\_01A0.
%
\item The LED1 to 7 CoreGPIO output register settings are:
\begin{itemize}
\item 0x4000\_01A0 = 0x01 (LED1 on) or 0x00 (LED1 off)
\item 0x4000\_01A0 = 0x02 (LED2 on) or 0x00 (LED2 off)
\item 0x4000\_01A0 = 0x04 (LED3 on) or 0x00 (LED3 off)
\item 0x4000\_01A0 = 0x08 (LED4 on) or 0x00 (LED4 off)
\item 0x4000\_01A0 = 0x10 (LED5 on) or 0x00 (LED5 off)
\item 0x4000\_01A0 = 0x20 (LED6 on) or 0x00 (LED6 off)
\item 0x4000\_01A0 = 0x40 (LED7 on) or 0x00 (LED7 off)
\end{itemize}
%
\item Use the Eclipse Memory View to add a new monitor at address 0x4000\_0100.
\item Change the byte at 0x4000\_01A0 to turn each of the LEDs on and off.
\end{itemize}
\textcolor{OliveGreen}{\bf Success! We have blinked LED1 through LED7 using CoreGPIO!}
\end{enumerate}
%
Understanding how to perform direct register accesses is a powerful debug tool.
Peripheral accesses are normally performed using a software API, eg.,
the System Tick function shown in Figure~\ref{fig:pfs_disco_reference_software_design}
blinks LED1 using the function call MSS\_GPIO\_set\_output().
If blinking the other LEDs using this function call does not work, then the
fact that direct register accesses have confirmed that all LEDs can be blinked
shows that any problem must be with the API usage.

% -----------------------------------------------------------------------------
\subsection{Software Modification}
% -----------------------------------------------------------------------------

The manual memory edits that determined the steps needed to toggle the LEDs
can be incorporated into the \texttt{mpfs-gpio-interrupt} by editing
\verb+u54_1.c+ as follows:
%
\begin{enumerate}
\item \textbf{Enable LED8 for output}

Insert one of the following code snippets
in the body of \verb+u54_1()+ at line 120:

\begin{enumerate}
\item Direct memory access
\begin{lstlisting}[style=c]
// LED8 (pin MSSIO23 = GPIO_1_(23-14) = GPIO_1_9)
*(volatile uint32_t *)0x20121024 = 0x05;
\end{lstlisting}
%
\item MSS GPIO API
\begin{lstlisting}[style=c]
// LED8 (pin MSSIO23 = GPIO_1_(23-14) = GPIO_1_9)
MSS_GPIO_init(GPIO1_LO);
MSS_GPIO_config(GPIO1_LO, MSS_GPIO_9, MSS_GPIO_OUTPUT_MODE);
\end{lstlisting}
\end{enumerate}

\item \textbf{Modify the System Tick code to generate an LED count}

Replace the System Tick code with one of the following:

\begin{enumerate}
\item Direct memory access
\begin{lstlisting}[style=c]
void SysTick_Handler_h1_IRQHandler(void)
{
    uint32_t hart_id = read_csr(mhartid);
    static volatile uint16_t value = 0u;
    uint8_t led, led1to7, led8;
    uint32_t gpio1_out;
    uint32_t gpio2_out;

    if (1u == hart_id)
    {
        value++;

        // Slow the LED count rate
        led = (value >> 2) & 0xFF;

        // LED8 and LED[1:7] settings
        led8    = led & 1;
//      led8    = (led >> 7) & 1;
        led1to7 = led & 0x7F;

        // LED8 output
        gpio1_out = led8 << 9;
        *(volatile uint32_t *)0x20121088 = gpio1_out;

        // LED[1:7] output via MSS GPIO
        gpio2_out = led1to7 << 17;
        *(volatile uint32_t *)0x20122088 = gpio2_out;

        // LED[1:7] output via CoreGPIO
//      gpio2_out = led1to7;
//      *(volatile uint32_t *)0x400001A0 = gpio2_out;
    }
}
\end{lstlisting}
%
\item MSS GPIO API
\begin{lstlisting}[style=c]
void SysTick_Handler_h1_IRQHandler(void)
{
    uint32_t hart_id = read_csr(mhartid);
    static volatile uint16_t value = 0u;
    uint8_t led, led1to7, led8;
    uint32_t gpio1_out;
    uint32_t gpio2_out;

    if (1u == hart_id)
    {
        value++;

        // Slow the LED count rate
        led = (value >> 2) & 0xFF;

        // LED8 and LED[1:7] settings
        led8    = led & 1;
//      led8    = (led >> 7) & 1;
        led1to7 = led & 0x7F;

        // LED8 output using single-bit API
//      MSS_GPIO_set_output(GPIO1_LO, MSS_GPIO_9, led8);

        // LED8 output using multi-bit API
        gpio1_out = led8 << 9;
        MSS_GPIO_set_outputs(GPIO1_LO, gpio1_out);

        // LED[1:7] output via MSS GPIO
        gpio2_out = led1to7 << 17;
        MSS_GPIO_set_outputs(GPIO2_LO, gpio2_out);
    }
}
\end{lstlisting}
%
\end{enumerate}
\end{enumerate}
%
After editing the \texttt{mpfs-gpio-interrupt} source, build the project, download in debug mode,
and resume the application. The LEDs on the board will show an incrementing count. The count rate
can be adjusted by editing the System Tick code to adjust the \verb+value+ right-shift value.
The LED8 blink rate in the code above is the same as LED1, but the code can be edited to have
LED8 blink as the MSB of an 8-bit LED count.

The modifications to \texttt{mpfs-gpio-interrupt} were a bit too hacky. Closer inspection of the
code shows several more calls to the MSS GPIO API for GPIO2. For example, the three GPIO IRQ
handlers each contain the call
\verb+MSS_GPIO_set_outputs(GPIO2_LO, 0u)+, and this clears all
bits in the fabric GPIO output register. An application that needs to independently control
the Raspberry Pi signals on GPIO\_2\_[0:16] and the LEDs on GPIO\_2\_[17:23] would need to
perform atomic read-modify-write accesses to the GPIO2 output register, eg., see
\href{https://github.com/polarfire-soc/polarfire-soc-documentation/blob/master/bare-metal-embedded-software/bare-metal-driver-user-guides/polarfire-soc-mss-driver-user-guides/mss-gpio/mss-gpio-driver-user-guide.md#example2}
{Example2}.

