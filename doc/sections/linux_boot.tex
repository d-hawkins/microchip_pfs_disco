% =============================================================================
\section{Example 3: Linux Boot}
% =============================================================================
\label{sec:linux_boot}

This section documents the Microchip resources that describe how to configure
the Discovery Kit to boot Linux.
%
The PolarFire SoC Discovery Kit Linux image requires that the kit be programmed
with both the Hardware Reference Design and Hart Software Services (HSS).
The HSS is a first-stage bootloader that is used to load U-boot from SD card,
and then U-boot boots Linux.

% -----------------------------------------------------------------------------
\subsection{Hardware Reference Design and HSS}
% -----------------------------------------------------------------------------

Download the Hardware Reference Design and HSS release:
%
\begin{enumerate}
% --------------------------------------
\item \textbf{Download the Release File}
% --------------------------------------
%
\begin{itemize}
\item
\href{https://www.microchip.com/en-us/development-tool/mpfs-disco-kit}
{Microchip Discovery Kit web site}

The PolarFire SoC \textbf{Discovery Kit Reference Design} links to $\dots$

\item
\href{https://github.com/polarfire-soc/polarfire-soc-discovery-kit-reference-design}
{PolarFire SoC Discovery Kit Reference Design}

The releases link of this page $\dots$

\item
\href{https://github.com/polarfire-soc/polarfire-soc-discovery-kit-reference-design/releases/tag/v2025.07}
{Discovery Kit Reference Design Release v2025.07}

Contains the release zip file $\dots$

\item
\href{https://github.com/polarfire-soc/polarfire-soc-discovery-kit-reference-design/releases/download/v2025.07/MPFS_DISCOVERY_KIT_2025_07.zip}
{MPFS\_DISCOVERY\_KIT\_2025\_07.zip}
\end{itemize}

% --------------------------------------
\item \textbf{Extract the release files}
% --------------------------------------
%
\begin{itemize}
\item Download the zip file to a temporary directory, eg.,

\begin{verbatim}
C:\github\microchip_pfs_disco\linux
\end{verbatim}

\item Unzip and inspect the zip file contents (eg., using WSL):

\begin{verbatim}
$ ls -1 MPFS_DISCOVERY_KIT_2025_07
MPFS_DISCOVERY_KIT_2025_07.cfg
MPFS_DISCOVERY_KIT_2025_07.digest
MPFS_DISCOVERY_KIT_2025_07.job
MPFS_DISCOVERY_KIT_2025_07.xml
\end{verbatim}
%$
The FlashPro Express programming file is the .job file.

The .digest contains the PolarFire SoC image signatures.
\end{itemize}

% --------------------------------------
\item \textbf{Program the Discovery Kit}
% --------------------------------------
%
\begin{itemize}
\item Start FlashPro Express, create a new job file, and program the kit (eg., see Section~\ref{sec:factory_restore}).
\item Compare the digests reported by FlashPro with the contents of the .digest file, and confirm they match.
\item (Optional) power-cycle the board and verify the image.
\end{itemize}

\newpage
% --------------------------------------
\item \textbf{HSS Console Test}
% --------------------------------------
%
\begin{itemize}
\item Power the Discovery Kit
\item Connect to MMUART\_1 using a serial terminal console at 115200 baud, 8N1\newline (eg., TeraTerm)
\item Press SW2 (the switch furthest from the edge of the board)
\item The console will display the HSS boot messages

The HSS boot messages end with:
%
\begin{verbatim}
[0.945592] startup_service :: [init] -> [boot]
[0.951702] ipi_poll_service :: [Init] -> [Monitoring]
Press a key to enter CLI, ESC to skip
Timeout in 1 second
..
[4.208612] CLI boot interrupt timeout
[4.213577] loop 315165 took 606530793 ticks (max 606530793 ticks)
[4.221502] Initializing Boot Image ...
[4.226562] Trying to get boot image via MMC ...
[4.232482] Attempting to select SDCARD ... Failed
\end{verbatim}
%
The SD card failure message is expected, since an SD card is not installed.

The console message shows how to access the HSS command-line interface (CLI).

\item Press SW2 and then hit enter in the HSS console

\begin{verbatim}
Press a key to enter CLI, ESC to skip
Timeout in 1 second
.[3.411049] Character 13 pressed

[3.415632] Type HELP for list of commands
>> help
Supported commands:
YMODEM BOOT RESET HELP VERSION UPTIME DEBUG MEMTEST QSPI
EMMC MMC SDCARD PAYLOAD SPI ECC
\end{verbatim}
%
This confirms that the Discovery Kit is ready to boot Linux.
\end{itemize}
\end{enumerate}

\clearpage
% -----------------------------------------------------------------------------
\subsection{Linux SD Card Image}
% -----------------------------------------------------------------------------

Download the Linux SD Card Image for the Yocto BSP:
%
\begin{enumerate}
% --------------------------------------
\item \textbf{Download the SD card image}
% --------------------------------------
%
\begin{itemize}
\item
\href{https://www.microchip.com/en-us/development-tool/mpfs-disco-kit}
{Microchip Discovery Kit web site}

The \textbf{Running Linux} section on this page links to the Yocto Linux
BSP $\dots$

\item
\href{https://github.com/polarfire-soc/meta-polarfire-soc-yocto-bsp}
{Microchip PolarFire SoC Yocto BSP}

This repository is an archive.

The repo README.md links to the new Yocto repo $\dots$

\item
\href{https://github.com/linux4microchip/meta-mchp}
{Microchip Yocto Project BSP}

The releases link of this page $\dots$

\item
\href{https://github.com/linux4microchip/meta-mchp/releases/tag/linux4microchip%2Bfpga-2025.07}
{linux4microchip+fpga-2025.07}

Contains the Linux SD card image $\dots$

\item
\href{https://github.com/linux4microchip/meta-mchp/releases/download/linux4microchip+fpga-2025.07/mchp-base-image-mpfs-disco-kit.rootfs-20250725104508.wic.gz}
{mchp-base-image-mpfs-disco-kit.rootfs-20250725104508.wic.gz}
\end{itemize}
%
% --------------------------------------
\item \textbf{Program the SD card image}
% --------------------------------------
%
\begin{itemize}
\item The
\href{https://github.com/polarfire-soc/polarfire-soc-documentation/blob/master/reference-designs-fpga-and-development-kits/mpfs-discovery-kit-embedded-software-user-guide.md}
{Discovery Kit Embedded Software User Guide} links to
\href{https://github.com/polarfire-soc/polarfire-soc-documentation/blob/master/reference-designs-fpga-and-development-kits/updating-linux-in-mpfs-kit.md}
{Updating linux in MPFS kit}, and this page has QSPI, eMMC, and SD card programming instructions.

\item Download and setup the SD card imaging tool: \href{https://bztsrc.gitlab.io/usbimager}{USBImager}

\item Follow the
\href{https://github.com/polarfire-soc/polarfire-soc-documentation/blob/master/reference-designs-fpga-and-development-kits/updating-linux-in-mpfs-kit.md#sd-card-content-update-procedure}
{SD Card content update procedure}
instructions.

\item The .wic.gz SD card image needs to be uncompressed prior to writing to SD card.

\item The Linux image was programmed to a 64GB SanDisk SDXC card.

Windows \emph{Disk Management} showed:
\begin{itemize}
\item boot (D:) 64MB (Primary Partition)
\item 8MB (Primary Partition)
\item 439MB (Primary Partition)
\item 58.98GB (Unallocated)
\end{itemize}
%
\end{itemize}
\end{enumerate}

\clearpage
% -----------------------------------------------------------------------------
\subsection{Boot Linux}
% -----------------------------------------------------------------------------

\begin{itemize}
\item Power the Discovery Kit
\item Connect to MMUART\_1 using a serial terminal console at 115200 baud, 8N1
\item Connect to MMUART\_4 using a serial terminal console at 115200 baud, 8N1
\item Press SW2 (the switch furthest from the edge of the board)
\item The first console will display the HSS boot messages
\item The second console will display U-Boot and Linux boot messages
\item Login with the user name \verb+root+ (no password is required)
\item Example console output:

\begin{verbatim}
OpenEmbedded nodistro.0 mpfs-disco-kit ttyS1

mpfs-disco-kit login: root
This is version v2025.07 of the Polarfire SoC Yocto BSP.

Updated images and documentation are available at:
        https://github.com/polarfire-soc/
root@mpfs-disco-kit:~# uname -a
Linux mpfs-disco-kit 6.12.22-linux4microchip+fpga-2025.07-g032a7095303a
  #1 SMP Tue Jul 22 10:04:20 UTC 2025 riscv64 GNU/Linux
\end{verbatim}
%
The command \verb+cat /proc/cpuinfo+ will show the properties of
the four U51 processors. The properties of the first two are:
%
\begin{verbatim}
root@mpfs-disco-kit:~# cat /proc/cpuinfo
processor       : 0
hart            : 1
isa             : rv64imafdc_zicntr_zicsr_zifencei_zihpm_zca_zcd
mmu             : sv39
uarch           : sifive,u54-mc
mvendorid       : 0x1cf
marchid         : 0x1
mimpid          : 0x0
hart isa        : rv64imafdc_zicntr_zicsr_zifencei_zihpm_zca_zcd

processor       : 1
hart            : 2
isa             : rv64imafdc_zicntr_zicsr_zifencei_zihpm_zca_zcd
mmu             : sv39
uarch           : sifive,u54-mc
mvendorid       : 0x1cf
marchid         : 0x1
mimpid          : 0x0
hart isa        : rv64imafdc_zicntr_zicsr_zifencei_zihpm_zca_zcd
\end{verbatim}
\end{itemize}
%
The default Yocto image does not include the \verb+devmem+ tool (which could
be used to blink the LEDs).
%
This blog for from
\href{https://www.controlpaths.com/2021/12/20/getting-started-with-microchips-fpga-icicle-kit-and-polarfire-soc}
{Pablo Trujillo} shows how to update the Yocto image (for the Icicle Kit).

\clearpage
% -----------------------------------------------------------------------------
\subsection{U-Boot Blinky LED}
% -----------------------------------------------------------------------------

\begin{enumerate}
% --------------------------------------
\item \textbf{Access the U-Boot Console}
% --------------------------------------
%
\begin{itemize}
\item Shutdown Linux:
\begin{verbatim}
root@mpfs-disco-kit:~# shutdown -h now
\end{verbatim}
\item Press SW2 to reboot
\item Hit enter in the second console to interrupt U-Boot
\item Example console output:
\begin{verbatim}
U-Boot 2023.07.02-linux4microchip+fpga-2025.07 (Jul 22 2025 - 09:10:12 +0000)

CPU:   rv64imafdc
Model: Microchip PolarFire-SoC Discovery Kit
DRAM:  1 GiB (effective 1.8 GiB)
Core:  50 devices, 12 uclasses, devicetree: separate
MMC:   mmc@20008000: 0
Loading Environment from FAT... OK
In:    serial@20106000
Out:   serial@20106000
Err:   serial@20106000
Net:   eth0: ethernet@20110000
Hit any key to stop autoboot:  0
RISC-V #
\end{verbatim}
\end{itemize}
%
% --------------------------------------
\item \textbf{Toggle LED8}
% --------------------------------------
%
\begin{itemize}
\item Enable LED8 output driver
\begin{verbatim}
# mw.l 20121024 05
\end{verbatim}
%
\item Toggle LED8
\begin{verbatim}
# mw.l 20121088 0200
# mw.l 20121088 0
\end{verbatim}
\end{itemize}
%
% --------------------------------------
\item \textbf{Toggle LED[7:1]}
% --------------------------------------
%
\begin{itemize}
\item Toggle LED1
\begin{verbatim}
# mw.l 20122088 020000
# mw.l 20122088 0
\end{verbatim}
%
\item Toggle LED[7:1]
\begin{verbatim}
# mw.l 20122088 FE0000
# mw.l 20122088 0
\end{verbatim}
\end{itemize}
%
% --------------------------------------
\item \textbf{Turn on all the yellow LEDs}
% --------------------------------------
%
\begin{verbatim}
# mw.l 20121088 0
# mw.l 20122088 AA0000
\end{verbatim}
%
% --------------------------------------
\item \textbf{Turn on all the red LEDs}
% --------------------------------------
%
\begin{verbatim}
# mw.l 20121088 0200
# mw.l 20122088 540000
\end{verbatim}
%
%
\end{enumerate}
