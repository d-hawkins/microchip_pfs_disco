% =============================================================================
\section{PolarFire SoC Discovery Power, Reset, and Clocks}
% =============================================================================
\label{sec:power}

This section reviews the implementation of the PolarFire SoC Discovery
Kit power supplies, resets, and clocks. This information is specific to
the board design, and is required to define pin constraints (I/O standards).
The kit has power supply jumper options, and it is important to understand
how these jumpers need to be configured to avoid damaging the board when
connecting to external devices using the expansion connectors.

% -----------------------------------------------------------------------------
\subsection{Power Supply Jumpers}
% -----------------------------------------------------------------------------

The PolarFire SoC Discovery Kit schematic~\cite{Microchip_DISCO_SCH_2023}
contains several jumpers:
%
\begin{itemize}
%
\item \textsf{5P0V\_IN}: 5V power source select
\begin{itemize}
\item Jumper on schematic page 13
\item J47 selects between the USB power or power jack (J7)
\end{itemize}
%
\item \textsf{VDDI1\_5}: FPGA Bank 1 and 5 power
and \textsf{VDDAUX1}: FPGA VDDAUX1 power
\begin{itemize}
\item Jumpers on schematic page 11
\item J45 and J46 select between 2.5V or 3.3V (default)
\item The schematic notes are:
\begin{itemize}
\item 2.5V for MIPI and Ethernet PHY operation
\item 3.3V for RPi and MikroBus operation
\end{itemize}
\item J45 and J46 must be set to the same voltage (p8~\cite{Microchip_DISCO_UG_2025})
\item Bank 1 contains GPIO signals (p9~\cite{Microchip_DISCO_UG_2025})
\item Bank 5 contains MSS SGMII signals (p9~\cite{Microchip_DISCO_UG_2025})
\item The SGMII PHY is the Microchip (previously Vitesse) VSC8221~\cite{Microchip_VSC8221_2006}
\item Schematic pages 5 and 6 shows that \textsf{VDDI1\_5} powers the VSC8221 VDDIO pins
\item The VSC8221 data sheet indicates VDDIO can be 2.5V or 3.3V~\cite{Microchip_VSC8221_2006}
\item Schematic page 5 shows that \textsf{VDDI1\_5} powers the 125MHz oscillator
\item The 125MHz oscillator schematic symbol has a note that it supports 2.25V to 3.63V
\item The \href{https://www.microchip.com/en-us/product/at24cm01}{AT24CM01} EEPROM supports 1.7V to 5.5V
\item Schematic page 7 has the Bank 1 FPGA connections, which include:
\begin{itemize}
\item The Raspberry Pi header (RPi prefixed signals)
\item The MikroBus header (MBUS prefixed signals)
\item The Ethernet PHY  (VSC prefixed signals)
\item The Raspberry Pi MIPI connector (GPIO\_MIPI prefixed signals)
\end{itemize}
\item \textcolor{magenta}{The J45 and J46 jumpers need to be set correctly for any external device}
\end{itemize}
%
\item \textsf{VCCB}: 7-segment display power
\begin{itemize}
\item Jumper on schematic page 7
\item J49 selects between 3.3V or 5.0V
\item The dual-voltage buffer U6 (TX0108PWR) performs voltage translation
from 3.3V or 5.0V to 1.8V for the FPGA interface
\item The buffer protects the FPGA from damage
\item \textcolor{magenta}{The J49 jumper needs to be set correctly for the external 7-segment display}
\end{itemize}
%
\end{itemize}

\clearpage
% -----------------------------------------------------------------------------
% Resets
% -----------------------------------------------------------------------------
%
\begin{figure}[t]
  \begin{center}
    \includegraphics[width=\textwidth]
    {figures/pfs_resets}
  \end{center}
  \caption{PolarFire SoC resets.}
  \label{fig:pfs_resets}
\end{figure}
% -----------------------------------------------------------------------------

% -----------------------------------------------------------------------------
\subsection{Reset}
% -----------------------------------------------------------------------------

The PolarFire SoC power-on-reset sequence is described in detail in the
\emph{PolarFire Family Power-Up and Resets User Guide}~\cite{Microchip_PFSoC_PU_2025}.
%
Figure~\ref{fig:pfs_resets} shows the PolarFire SoC resets:
\begin{itemize}
\item External device reset (DEVRST\_N) pin resets the System Controller
\item The System Controller manages the resets to the processor
\item The System Controller manages the resets to the fabric
\end{itemize}
%
The next sections describe how to generate fabric resets, and how the
System Controller manages the relative timing of the
fabric reset deassertion versus processor reset deassertion.

\clearpage
% -----------------------------------------------------------------------------
% Power-on to functional timing
% -----------------------------------------------------------------------------
%
\begin{figure}[t]
  \begin{center}
    \includegraphics[width=\textwidth]
    {figures/pfs_power_on_to_functional_timing.png}
  \end{center}
  \caption{PolarFire SoC power-on to functional timing
           (per Figure 2-2, p8~\cite{Microchip_PFSoC_PU_2025}.}
  \label{fig:pfs_power_on_to_functional}
\end{figure}

% -----------------------------------------------------------------------------
\subsubsection{Fabric Power-on-Reset}
% -----------------------------------------------------------------------------

Flash-based FPGAs required a power-on-reset to ensure fabric registers are
initialized.
%
Figure~\ref{fig:pfs_power_on_to_functional} shows the PolarFire SoC resets.
%
The PolarFire SoC initialization monitor (PFSOC\_INIT\_MONITOR), visible on
the left side of the figure, provides System Controller configuration
status signals to the fabric logic.
%
Fabric logic \emph{must} use the PFSOC\_INIT\_MONITOR signals
to qualify any external reset inputs, since the I/Os are not enabled until
after the fabric logic is enabled. The initialization monitor signals are
asynchronous to any fabric clocks, so fabric resets based on the monitor
signals must be synchronized to their respective clock domains using
reset synchronizer components.
The Microchip SmartDesign examples instantiate the CORERESET\_PF reset
synchronizer component.
The \emph{PolarFire Family Power-Up and Resets User Guide} has SmartDesign
examples of PolarFire and PolarFire SoC initialization,
eg., see Figure 3-3 on page 46, and Figure 3-3 on
page 52~\cite{Microchip_PFSoC_PU_2025}.
%
The blinky LED example in Section~\ref{sec:fabric_blinky} generates the fabric
reset using the IP Catalog \emph{PolarFireSoC Initialization Monitor}
(v1.0.309) component, combinatorial logic, and a custom reset synchronizer
component.

Figure~\ref{fig:pfs_power_on_to_functional} shows the PolarFire SoC power-on
to functional timing (copied from Figure 2-2, p8~\cite{Microchip_PFSoC_PU_2025}).
The power-on-reset timing diagrams can also be found in the device
datasheet~\cite{Microchip_PFSoC_DS_2025}. The PolarFire SoC documentation
refers to the power-on-reset time as the \emph{power-up to functional time}
(PUFT).

\clearpage
% -----------------------------------------------------------------------------
% Probes
% -----------------------------------------------------------------------------
%
\begin{table}[t]
\caption{Discovery Kit Raspberry Pi header pins used to probe power-on-reset.}
\label{tab:pfs_disco_probes}
\begin{center}
\begin{tabular}{|c|l|c|c|l|l|}
\hline
RPi & Schematic        & PFSoC & Probe & Probed & Note\\
Pin & Net Name         & Pin   & Index & Signal &\\
\hline\hline
&&&&&\\
 1 & 3.3V              &       &       & Power &\\
&&&&&\\
 3 & RPi\_GPIO2\_SDA   & E18   & 0     & Logic low              & 10k pull-up\\
 5 & RPi\_GPIO3\_SCL   & F18   & 1     & DEVICE\_INIT\_DONE     & 10k pull-up\\
 7 & RPi\_GPIO4\_GCLK  & E12   & 2     & BANK\_0\_CALIB\_STATUS & \\
11 & RPi\_GPIO17\_GEN0 & G18   & 3     & Design reset (RST\_N)  & \\
&&&&&\\
39 & Ground            &       &       & Power                  &\\
&&&&&\\
\hline
\end{tabular}
\end{center}
\end{table}
% -----------------------------------------------------------------------------

The Discovery Kit FPGA fabric power-on-reset to functional timing was investigated
using the Raspberry Pi (RPi) 40-pin header and the fabric blinky LED design
(see Section~\ref{sec:fabric_blinky}).
Table~\ref{tab:pfs_disco_probes} shows the RPi header pins
probed using two oscilloscope channels.
%
Figures~\ref{fig:pfs_disco_power_on_a} and~\ref{fig:pfs_disco_power_on_b} show
the measured power-on-reset timing:
%
\begin{enumerate}
\item \textbf{Device Reset}

The Discovery kit schematic (p9~\cite{Microchip_DISCO_SCH_2023}) shows the
device reset pin (DEVRST\_N) is driven by a Microchip MCP121-315 3.3V
voltage supervisor~\cite{Microchip_MCP121_2023} which has a power-on delay of 120ms.

Figure~\ref{fig:pfs_disco_power_on_a}(a) shows the 3.3V power supply on channel 1
and an FPGA output driven low on channel 2. The pulse observed on channel 2
has a rising-edge that follows the 3.3V power supply due to the signal 10k
pull-up to 3.3V, and a falling-edge that occurs slightly after 120ms
due to the voltage monitor 120ms power-on delay time plus the fabric ready time.
Figure~\ref{fig:pfs_power_on_to_functional} shows the fabric ready delay is
the time taken until FPGA\_POR\_N deasserts.

\item \textbf{Device Initialization Done}

Figure~\ref{fig:pfs_disco_power_on_a}(b) shows the falling-edge of the low
output on channel 1 and DEVICE\_INIT\_DONE on channel 2. The channel 2
signal starts out high due to the 10k pull-up to 3.3V on the on the signal,
transitions low when the output driver is enabled, and then transitions
high when DEVICE\_INIT\_DONE asserts.
%
Figure~\ref{fig:pfs_power_on_to_functional} indicates that the time between
the deassertion of FPGA\_POR\_N and the assertion of DEVICE\_INIT\_DONE
is design dependent: for the Discovery Kit blinky LED design this time is
just under 600$\mu$s.

\item \textbf{Bank Calibration Done}

Figure~\ref{fig:pfs_disco_power_on_b}(a) shows the falling-edge of the low
output on channel 1 and BANK\_0\_CALIB\_STATUS on channel 2.
%
The Discovery Kit fabric drive LEDs (LED1 through LED7) are located on Bank 0,
so the blinky LED design uses BANK\_0\_CALIB\_STATUS as one of the design
reset sources.

\item \textbf{Fabric Logic Reset}

Figure~\ref{fig:pfs_disco_power_on_b}(b) shows the falling-edge of the low
output on channel 1 and the blinky LED reset on channel 2.
The blinky LED reset is synchronized to the 50MHz clock.
%
Figures~\ref{fig:pfs_disco_power_on_b}(a) and (b) look the same, as
BANK\_0\_CALIB\_STATUS is one of the inputs to the design reset synchronizer.

\end{enumerate}

% -----------------------------------------------------------------------------
% Power-on-reset waveforms
% -----------------------------------------------------------------------------
%
\begin{figure}[p]
  \begin{center}
    \includegraphics[width=0.8\textwidth]
    {figures/pfs_disco_power_on_0_output_low.png}\\
    (a) 3.3V and a fabric output driven low\\
    \vskip5mm
    \includegraphics[width=0.8\textwidth]
    {figures/pfs_disco_power_on_1_device_init_done.png}\\
    (b) Fabric output driven low and BANK\_0\_CALIB\_STATUS
  \end{center}
  \caption{Discovery Kit power-on-reset waveforms.}
  \label{fig:pfs_disco_power_on_a}
\end{figure}

\begin{figure}[p]
  \begin{center}
    \includegraphics[width=0.8\textwidth]
    {figures/pfs_disco_power_on_2_calib_done.png}\\
    (a) Fabric output driven low and CALIB\_DONE\\
    \vskip5mm
    \includegraphics[width=0.8\textwidth]
    {figures/pfs_disco_power_on_3_reset.png}\\
    (b) Fabric output driven low and 50MHz clock-domain active-low reset
  \end{center}
  \caption{Discovery Kit power-on-reset waveforms.}
  \label{fig:pfs_disco_power_on_b}
\end{figure}

% -----------------------------------------------------------------------------
\subsubsection{Fabric Reset Push Button}
% -----------------------------------------------------------------------------

The Discovery Kit FIR filter design uses SWITCH1 as a push button reset.
%
The top-level SmartDesign uses a debounce component
(see Figure 2-2, p5~\cite{Microchip_AN5165_2024}).
%
The PolarFire SoC Discovery Kit schematic shows that the switches have
an RC-filter with a time-constant of 1ms. The PolarFire SoC inputs can be
configured to enable a Schmitt trigger input. The combination of RC-filter
and Schmitt trigger should be sufficient, so a debounce circuit is not
required. The SmartDesign debouncer is likely something that
was inherited from the FIR filter design from a different development kit,
eg., this FIR filter design is available for
\href{https://www.microchip.com/en-us/application-notes/an4753}{RTG4},
\href{https://www.microchip.com/en-us/application-notes/DG0438}{SmartFusion2}, and
\href{https://www.microchip.com/en-us/application-notes/DG0504}{IGLOO2} kits.

The \href{https://github.com/polarfire-soc/polarfire-soc-discovery-kit-reference-design}
{Discovery Kit Reference Design} script was run with the FIR\_DEMO argument and
the I/O Editor used to view the pin constraints: the Schmitt trigger was not
enabled on the SWITCH1 input.
%
Figure~\ref{fig:pfs_disco_reset} shows the Discovery Kit reset waveforms measured
from the blinky LED example in Section~\ref{sec:fabric_blinky} with the SWITCH1
Schmitt trigger input enabled. Similar waveforms were observed with the Schmitt
trigger disabled. Conclusion: the Discovery Kit does not need switch debouncing.

% -----------------------------------------------------------------------------
% Reset push button waveforms
% -----------------------------------------------------------------------------
%
\begin{figure}[p]
  \begin{center}
    \includegraphics[width=0.8\textwidth]
    {figures/pfs_disco_reset_a.png}\\
    (a) 3.3V and reset assertion at 100ns/div\\
    \vskip5mm
    \includegraphics[width=0.8\textwidth]
    {figures/pfs_disco_reset_b.png}\\
    (b) 3.3V and reset assertion at 20ms/div
  \end{center}
  \caption{Discovery Kit push button reset waveforms.}
  \label{fig:pfs_disco_reset}
\end{figure}

\clearpage
% -----------------------------------------------------------------------------
\subsubsection{Processor Reset and Boot}
% -----------------------------------------------------------------------------


The PolarFire SoC MSS reset sequence is described in detail in the
\emph{PolarFire Family Power-Up and Resets User Guide}~\cite{Microchip_PFSoC_PU_2025}.
%
Figure~\ref{fig:pfs_mss_boot_modes}(a) shows the MSS Pre-Boot execution flow
that leads to the selection of one of four different boot modes.
%
The two boot modes used in this tutorial are:
%
\begin{itemize}
\item \textbf{Boot mode 0: Idle boot}

Figure~\ref{fig:pfs_mss_boot_modes}(b) shows the idle boot mode flow.
The PolarFire SoC MSS processors enter busy loops until the debugger
connects (p29~\cite{Microchip_PFSoC_PU_2025}).

\item \textbf{Boot mode 1: Non-secure boot}

Figure~\ref{fig:pfs_mss_boot_modes}(c) shows the non-secure boot mode flow.
The PolarFire SoC MSS processors execute from reset vectors stored in
eNVM (see Table 2-5 on p30 for the reset vector addresses~\cite{Microchip_PFSoC_PU_2025}).
\end{itemize}
%
Section~\ref{sec:riscv_blinky} exercises these two boot modes.

%------------------------------------------------------------------------------
% PolarFire SoC MSS Boot Modes
%------------------------------------------------------------------------------
%
\begin{figure}[p]
  \begin{minipage}{\textwidth}
    \begin{center}
    \includegraphics[width=\textwidth]
    {figures/pfs_mss_boot_modes.png}\\
    (a) Boot modes
    \end{center}
  \end{minipage}
  \vskip10mm
  \begin{minipage}{0.49\textwidth}
    \begin{center}
    \includegraphics[width=0.45\textwidth]
    {figures/pfs_mss_boot_mode_0.png}\\
    (b) Boot mode 0: Idle boot flow
    \end{center}
  \end{minipage}
  \hfil
  \begin{minipage}{0.49\textwidth}
    \begin{center}
    \includegraphics[width=0.45\textwidth]
    {figures/pfs_mss_boot_mode_1.png}\\
    (c) Boot mode 1: Non-secure boot flow
    \end{center}
  \end{minipage}
  \caption{PolarFire SoC MSS boot modes.}
  \label{fig:pfs_mss_boot_modes}
\end{figure}
% -----------------------------------------------------------------------------

\clearpage
% -----------------------------------------------------------------------------
% Clocks
% -----------------------------------------------------------------------------
%
\begin{figure}[t]
  \begin{center}
    \includegraphics[width=\textwidth]
    {figures/pfs_disco_mss_clocks}
  \end{center}
  \caption{PolarFire SoC Discovery Kit MSS clock configuration with 125MHz reference clock.}
  \label{fig:pfs_disco_mss_clocks}
\end{figure}
% -----------------------------------------------------------------------------

% -----------------------------------------------------------------------------
\subsection{Clocks}
% -----------------------------------------------------------------------------

The PolarFire SoC Discovery Kit has two external clock
sources (p5 and p13~\cite{Microchip_DISCO_UG_2025}):
%
\begin{itemize}
\item 125MHz MSS Reference clock (DSC1123BL5-125.0000)

The \href{https://www.microchip.com/en-us/product/DSC1123}{DSC1123} is a MEMS
oscillator with a startup time of 5ms (max).

\item 50 MHz Oscillator (DSC1001DL5-050.0000)

The \href{https://www.microchip.com/en-us/product/DSC1001}
{DSC1001} is a CMOS oscillator with a startup time of 1.3ms (max).

\end{itemize}
%
The MCP121 voltage supervisor power-on-reset time of 120ms
is sufficient for the oscillators to be stable when used by the PolarFire SoC.
%
Figure~\ref{fig:pfs_disco_mss_clocks} shows the MSS Configurator view of the
PolarFire Discovery Kit clocks using the
\href{https://github.com/polarfire-soc/polarfire-soc-discovery-kit-reference-design/blob/main/script_support/MPFS_DISCOVERY_KIT_MSS.cfg}
{reference design configuration}.
%
\begin{itemize}
\item The RISC-V processor clocks are configured for 600MHz
\item The AXI and L2 cache is configured for 300MHz
\item The AHB/APB bus configured for 150MHz
\item The DDR clock is configured for 800MHz (1600Mbps data rate)
\item The SGMII interface is configured for 625MHz (1.25Gbps serial rate)
\end{itemize}

