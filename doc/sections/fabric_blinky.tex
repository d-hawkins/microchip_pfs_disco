% =============================================================================
\section{Example 1: Fabric-only Blinky LEDs}
% =============================================================================
\label{sec:fabric_blinky}

% -----------------------------------------------------------------------------
% Libero GUI
% -----------------------------------------------------------------------------
%
\begin{figure}[t]
  \begin{center}
    \includegraphics[width=\textwidth]
    {figures/ex1_fabric_blinky.pdf}
  \end{center}
  \caption{Fabric-only Blinky LEDs block diagram.}
  \label{fig:ex1_diagram}
\end{figure}
% -----------------------------------------------------------------------------

Figure~\ref{fig:ex1_diagram} shows a block diagram of the fabric-only Blinky
LEDs design. This section demonstrates how to use the Libero SoC GUI to manually
create a project, create physical design constraints, and timing constraints,
and then how to reproduce the design using scripts.
%
Table~\ref{tab:ex1_source} shows the source and scripts described in this
section.

% -----------------------------------------------------------------------------
% Project files
% -----------------------------------------------------------------------------
%
\begin{table}[b]
\caption{Discovery Kit fabric-only blinky LEDs project source.}
\label{tab:ex1_source}
\begin{center}
\begin{tabular}{|l|l|}
\hline
Filename & Description\\
\hline\hline
&\\
\texttt{designs/fabric\_blinky}              & Design directory\\
\quad\texttt{src/pfs\_disco.sv}              & Top-level design\\
\quad\texttt{scripts/pfs\_disco.sdc}         & Timing constraints\\
\quad\texttt{scripts/pfs\_disco\_io.pdc}     & Pin constraints\\
\quad\texttt{scripts/pfs\_disco\_fp.pdc}     & Floorplan constraints\\
\quad\texttt{scripts/pfs\_disco.ndc}         & Output register constraints\\
\quad\texttt{scripts/pfs\_disco.fdc}         & TMR constraints\\
\quad\texttt{scripts/libero.tcl}             & Libero build script\\
\quad\texttt{scripts/synplify.tcl}           & Synplify build script\\
&\\
\texttt{ip}                                    & Common IP directory\\
\quad\texttt{blinky/src/blinky.sv}             & Blinky LED design\\
\quad\texttt{cdc\_sync\_bit/src/cdc\_sync\_bit.sv}         & Synchronizer\\
\quad\texttt{pfs\_init\_monitor/src/pfs\_init\_monitor.sv} & PFS\_INIT\_MONITOR wrapper\\
&\\
\hline
\end{tabular}
\end{center}
\end{table}

\clearpage
% -----------------------------------------------------------------------------
\subsection{Libero Manual Project Creation}
% -----------------------------------------------------------------------------

Create the Libero SoC project as follows:
%
\begin{enumerate}
\item Start \emph{Libero SoC 2025.1}
\item Select \emph{Project$\rightarrow$New Project}
\item \textbf{Project details}

Configure the project name and location, eg.,
\begin{itemize}
\item Project name: \verb+pfs_disco+
\item Project location:
\verb+$TUTORIAL/designs/fabric_blinky/build+%$

where \verb+TUTORIAL+ is the path to the tutorial source, eg.,\newline
\verb+TUTORIAL = C:/github/microchip_pfs_disco+.
\item Click \emph{Next}
\end{itemize}
The Libero project is created in directory \verb+build/pfs_disco+. The scripted
flow uses the directory name \verb+build/libero+.
%
\item \textbf{Device selection}

Select the PolarFire Discovery device:
\begin{itemize}
\item Change the part filter family to PolarFireSoC
\item Change the die to MPFS095T
\item Change the package to FCSG325
\item Select part number MPFS095T-1FCSG325E (-1 speed grade)
\item Click \emph{Next}
\end{itemize}
%
\item \textbf{Device settings}
\begin{itemize}
\item The default settings are acceptable, i.e.,
\item Core Voltage: 1.0
\item Default I/O technology: LVCMOS 1.8V
\item (Checked) Reserve pins for probes
\item (Unchecked) System controller suspend mode
\item Click \emph{Next}
\end{itemize}
%
\newpage
\item \textbf{Add HDL source files}
\begin{itemize}
\item Click on the \emph{Link File} button and link the SystemVerilog files:
\begin{itemize}
\item \verb+$TUTORIAL/designs/fabric_blinky/src/pfs_disco.sv+
%$
\item \verb+$TUTORIAL/ip/blinky/src/blinky.sv+
%$
\item \verb+$TUTORIAL/ip/cdc_sync_bit/src/cdc_sync_bit.sv+
%$
\item \verb+$TUTORIAL/ip/pfs_init_monitor/src/pfs_init_monitor.sv+
%$
\end{itemize}
\item Click \emph{Next}
\item Click \emph{Finish}
\end{itemize}
%
\item To configure Libero SoC to use SystemVerilog, select
\emph{Project$\rightarrow$Project Settings}, then \emph{Design flow},
change the Verilog language radio button to SystemVerilog, click \emph{Save}, and
then \emph{Close}.
%
\item Click on the \emph{Design Hierarchy} tab in the Libero GUI.
\item Click on the \emph{Build Hierarchy} button.
\item Select the top-level of the hierarchy, \verb+pfs_disco+, right-mouse-click and
select \emph{Set As Root}.
\item Click on the \emph{Design Flow} tab in the Libero GUI.
\item Click on the \emph{Save} button.
%
\item Configure the synthesis tool using \emph{Project$\rightarrow$Tool Profiles},
and then \emph{Synthesis}.
%
\item Click on the green play button to synthesize and place-and-route the design.
\item The Libero GUI will update with green check marks next to \emph{Implement Design},
\emph{Synthesize}, and \emph{Place and Route}.
Figure~\ref{fig:pfs_disco_libero_gui_manual} shows a screen shot of the Libero SoC 2025.1 GUI.
%
\item The Tcl commands correspondng to most of the GUI actions can be written to a script using:

Select \emph{Project$\rightarrow$Export Script File}, and configure the settings to
\begin{itemize}
\item Script file: \verb+$TUTORIAL/designs/fabric_blinky/build/exported.tcl+
%$
\item Check: Include commands from current session only
\item Select: Relative file names
\item Click \emph{OK}.
\end{itemize}
The Tcl commands in exported.tcl script were the basis for \verb+scripts/libero.tcl+.
\end{enumerate}

\vskip5mm
\begin{center}
\fbox{\parbox{0.85\linewidth}{
\textbf{\textcolor{red}{WARNING:}}\newline
The design is not ready for hardware download, as pin assignments were not defined!}}
\end{center}

\clearpage
% -----------------------------------------------------------------------------
% Libero GUI
% -----------------------------------------------------------------------------
%
\begin{landscape}
\begin{figure}[p]
  \begin{center}
    \includegraphics[width=215mm]
    {figures/ex1_libero_gui_manual.png}
  \end{center}
  \caption{Libero GUI after manual project creation.}
  \label{fig:ex1_libero_gui_manual}
\end{figure}
\end{landscape}

\clearpage
% -----------------------------------------------------------------------------
\subsection{Pin Constraints}
% -----------------------------------------------------------------------------

In the Libero GUI:
%
\begin{enumerate}
\item Select the \emph{Design Flow} tab.
\item Select the \emph{Manage Constraints} step.
\item Right-mouse-click and select \emph{Open Constraints Manager View}.
\item The \emph{Constraints Manager} opens with the \emph{I/O Attributes}
tab selected.
\item Click the button/pull-down \emph{Edit$\rightarrow$Edit with I/O Editor}
button to edit the pin constraints.
\item Figure~\ref{fig:ex1_libero_gui_pins_default} shows the default pin
assignments. These pin assignments do not represent any specific hardware
design and need to be customized for the target hardware.

\item The pin assignments editor can be used to define the pin constraints specific
to a board, or can be used to create a template I/O constraints .pdc file, which
can then be edited with the board-specific constraints.
%
\item Update the \emph{I/O Editor} column for \emph{Pin Number} to match
Table~\ref{tab:ex1_pins}.

The default \emph{I/O Standard} of LVCMOS18 is correct for the clock,
asynchronous reset input and LED outputs, but is not correct for the probes.
These issues were resolved by exporting the .PDC file, and then manually
editing the bank voltages and I/O standard settings.

The \emph{I/O Editor} view includes a slew rate column. GPIO pins support
PDC slew rate constraints, whereas HSIO pins do not.
The GPIO default setting of OFF results in the fastest slew rate
(p26~\cite{Microchip_PFSoC_IO_2025}). The probes do not need fast slew rate,
so the slew rate was reduced.

\item Export a Physical Design Constraints I/O file:
\begin{itemize}
\item \emph{File$\rightarrow$Export Physical Constraint (PDC)$\rightarrow$I/O Constraint}
\item Change the exported filename to\newline
\verb+$TUTORIAL/designs/fabric_blinky/build/pfs_disco_io.pdc+%$
\item Check the \emph{Full Constraints} checkbox
\item Click \emph{OK}
\item Exit \emph{I/O Editor} and ignore the pin constraints changes.
\end{itemize}

The exported PDC was used as the basis for \verb+scripts/pfs_disco_io.pdc+.
\item Exit the I/O Editor GUI.
\item In the \emph{Constraint Manager}, \emph{I/O Attributes} tab, link the
pin constraints file\newline
\verb+$TUTORIAL/designs/fabric_blinky/scripts/pfs_disco_io.pdc+,\newline %$
check the  \emph{Place and Route} checkbox, and click the \emph{Save} button.

\item Click the green play button to run place-and-route with the new pin assignments.

\item Figure~\ref{fig:ex1_libero_gui_pins} shows the Discovery Kit pin
assignments.

\end{enumerate}

\vskip5mm
\begin{center}
\fbox{\parbox{0.85\linewidth}{
\textbf{\textcolor{red}{WARNING:}}\newline
Since this design does not have any interfaces with critical timing, this
design could now be downloaded to hardware. However, in general, designs
should have timing constraints, and be verified to meet those constraints,
before testing on hardware.}}
\end{center}

\clearpage
% -----------------------------------------------------------------------------
% Discovery Kit Blinky LED Pin Assignments
% -----------------------------------------------------------------------------
%
\begin{table}[p]
\caption{Discovery Kit blinky LED pin assignments.}
\label{tab:ex1_pins}
\begin{center}
\begin{tabular}{|l|c|c|c|c|l|}
\hline
Port name & Pin Name & Voltage & Drive & Slew & Note\\
\hline\hline
&&&&&\\
\texttt{arst\_n}  & T19   & LVCMOS18 &&& Push button SWITCH1\\
&&&&&\\
\texttt{clk}      & R18   & LVCMOS18 &&& 50MHz CLKIN input\\
&&&&&\\
\texttt{led[0]}   & T18   & LVCMOS18 & 4 & OFF & HSIO, VDDI0 = 1.8V\\
\texttt{led[1]}   & V17   & LVCMOS18 & 4 & OFF &\\
\texttt{led[2]}   & U20   & LVCMOS18 & 4 & OFF &\\
\texttt{led[3]}   & U21   & LVCMOS18 & 4 & OFF &\\
\texttt{led[4]}   & AA18  & LVCMOS18 & 4 & OFF &\\
\texttt{led[5]}   & V16   & LVCMOS18 & 4 & OFF &\\
\texttt{led[6]}   & U15   & LVCMOS18 & 4 & OFF &\\
&&&&&\\
\texttt{uart\_rx} & W21   & LVCMOS18 & 4 & OFF &\\
\texttt{uart\_tx} & Y21   & LVCMOS18 & 4 & OFF &\\
&&&&&\\
\texttt{probe[0]} & E18   & LVCMOS33 & 4 & ON  & GPIO, VDDI1 = 3.3V\\
\texttt{probe[1]} & F18   & LVCMOS33 & 4 & ON  &\\
\texttt{probe[2]} & E12   & LVCMOS33 & 4 & ON  &\\
\texttt{probe[3]} & G18   & LVCMOS33 & 4 & ON  &\\
&&&&&\\
\hline
\end{tabular}
\end{center}
\end{table}

\clearpage
% -----------------------------------------------------------------------------
% Libero GUI I/O Editor
% -----------------------------------------------------------------------------
%
\begin{landscape}
\begin{figure}[p]
  \begin{center}
    \includegraphics[width=215mm]
    {figures/ex1_libero_gui_pins_default.png}
  \end{center}
  \caption{Libero I/O Editor default pin assignments after manual project creation.}
  \label{fig:ex1_libero_gui_pins_default}
\end{figure}
\end{landscape}

\begin{landscape}
\begin{figure}[p]
  \begin{center}
    \includegraphics[width=215mm]
    {figures/ex1_libero_gui_pins.png}
  \end{center}
  \caption{Libero I/O Editor with Discovery Kit pin assignments.}
  \label{fig:ex1_libero_gui_pins}
\end{figure}
\end{landscape}

\clearpage
% -----------------------------------------------------------------------------
\subsection{Timing Constraints}
% -----------------------------------------------------------------------------

In the Libero GUI:
%
\begin{enumerate}
\item Select the \emph{Design Flow} tab.
\item Select the \emph{Manage Constraints} step.
\item Right-mouse-click and select \emph{Open Constraints Manager View}.
\item The \emph{Constraints Manager} opens with the \emph{I/O Attributes}
tab selected.
\item Click on the \emph{Timing} tab.
\item Click on the button/pull-down \emph{Edit$\rightarrow$Edit Synthesis Constraints}.
\item The \emph{Constraints Editor} allows you to construct timing constraints
that adhere to Synopsys Design Constraints (SDC) syntax.
Constraints are saved in the Libero project in the file
\verb+constraint\user.sdc+. For example, a 50MHz clock constraint was
created and saved, and the corresponding SDC Tcl command was:
%
\begin{verbatim}
create_clock -name {clk_50mhz} -period 20 -waveform {0 10} \
    [get_ports {clk_50mhz}]
\end{verbatim}

\item Exit the \emph{Constraints Editor} GUI.
\item Remove \verb+user.sdc+ from the project.
\item In the \emph{Constraint Manager}, \emph{Timing} tab, link the
timing constraints file\newline
\verb+$TUTORIAL/designs/fabric_blinky/scripts/pfs_disco.sdc+,\newline %$
leave the \emph{Synthesis} checkbox unchecked,
check the \emph{Place and Route} checkbox,
check the \emph{Timing Verification} checkbox,
and click the \emph{Save} button.

The \emph{Synthesis} checkbox was left unchecked, as this checkbox is associated
with the timing constraint file passed to Synplify for synthesis. The default
Synplify settings are sufficient for most designs, so a Synplify SDC file
is not required.

\item Open the SDC file and review the LED clock-to-output timing parameters.

\item Click the green play button to run place-and-route with constraints.

\item In the \emph{Design Flow} GUI, select the \emph{Open SmartTime} step,
right-mouse-click and select \emph{Open Interactively}.
\end{enumerate}

The SmartTime GUI can be used to view different timing paths within the
design, and can be used to generate a schematic diagram of the delay paths.
%
Figures~\ref{fig:ex1_libero_smarttime_fabric_max}
and~\ref{fig:ex1_libero_smarttime_fabric_min} show the maximum and
minimum delay paths for the LED clock-to-output delays with the output
registers in the fabric.
The reported slack is under 0.1ns, as the SDC constraint for the maximum delay
was set to equal the reported delay rounded up to the nearest 0.1ns,
while the minimum delay was set equal to the reported delay rounded down
to the nearest 0.1ns.
%
Figure~\ref{fig:ex1_libero_smarttime_path}(a) shows the delay
path for the LED output registers in the fabric: the output register
is an SLE.
%
Figure~\ref{fig:ex1_libero_chipplanner}(a) shows the ChipPlanner
view for the LED output registers in the fabric: the blinky logic
was selected so it is highlighed in white, while the I/O pads
are highlighed in yellow.

\clearpage
% -----------------------------------------------------------------------------
% LED clock-to-output delay
% -----------------------------------------------------------------------------
%
\begin{table}
\caption{Blinky LED clock-to-output delays (in nanoseconds).}
\label{tab:ex1_tco}
\begin{center}
\begin{tabular}{|l||r|r||r|r|}
\hline
          & \multicolumn{2}{c||}{Fabric Registers} &
           \multicolumn{2}{c|}{I/O Registers}\\
\cline{2-5}
Parameter & Min & Max & Min & Max\\
\hline\hline
&\phantom{XXXX}&\phantom{XXXX}&\phantom{XXXX}&\phantom{XXXX}\\
Clock-to-Output & 3.769 & 6.651 &  3.587 & 6.010\\
Constraint      & 3.700 & 6.700 &  3.500 & 6.100\\
Slack           & 0.069 & 0.049 &  0.087 & 0.090\\
&&&&\\
\hline
\end{tabular}
\end{center}
\end{table}

% -----------------------------------------------------------------------------
\subsection{I/O Register Constraints}
% -----------------------------------------------------------------------------

The placement of I/O registers in PolarFire SoC designs can be
contolled using a Netlist Design Constraint (NDC) file. In the Libero GUI:
%
\begin{enumerate}
\item Select the \emph{Design Flow} tab.
\item Select the \emph{Manage Constraints} step.
\item Right-mouse-click and select \emph{Open Constraints Manager View}.
\item The \emph{Constraints Manager} opens with the \emph{I/O Attributes}
tab selected.
\item Click on the \emph{Netlist Attributes} tab.
\item Link the netlist constraints file\newline
\verb+$TUTORIAL/designs/fabric_blinky/scripts/pfs_disco.ndc+,\newline %$
check the \emph{Synthesis} checkbox,
and click the \emph{Save} button.
\item Edit \verb+pfs_disco.ndc+ to disable (or enable) the use of I/O registers.
\item Edit \verb+pfs_disco.sdc+ to set the timing constraints for fabric (or I/O) registers.

\item Click the green play button to run place-and-route.

\item In the \emph{Design Flow} GUI, select the \emph{Open SmartTime} step,
right-mouse-click and select \emph{Open Interactively}.
\end{enumerate}
%
Figures~\ref{fig:ex1_libero_smarttime_ioff_max}
and~\ref{fig:ex1_libero_smarttime_ioff_min} show the maximum and
minimum delay paths for the LED clock-to-output delays with the output
registers in the I/O.
The reported slack is under 0.1ns, as the SDC constraint for the maximum delay
was set to equal the reported delay rounded up to the nearest 0.1ns,
while the minimum delay was set equal to the reported delay rounded down
to the nearest 0.1ns.
%
Figure~\ref{fig:ex1_libero_smarttime_path}(b) shows the delay
path for the LED output registers in the I/O: the output register
is an IOREG.
%
Figure~\ref{fig:ex1_libero_chipplanner}(b) shows the ChipPlanner
view for the LED output registers in the I/O.
A close comparison of the white and yellow highlighted components in
Figure~\ref{fig:ex1_libero_chipplanner}(a) and (b) shows
that (a) has additional registers near the output pads that are not
highlighted in (b): these are the fabric output registers.

Table~\ref{tab:ex1_tco} summarizes the LED clock-to-output delays for
the output registers placed in the fabric vs I/O. The constraint
values can be used as \emph{datasheet} values when analyzing external
device interface timing. If external device interface timing requirements
cannot be met with either of these output register configurations shown in
Table~\ref{tab:ex1_tco}, then the clock-to-output delay can be adjusted
using the output drive strength, or using the programmable output delay,
or using BUFD delays within the fabric.

\clearpage
% -----------------------------------------------------------------------------
% Libero GUI SmartTime Timing Analyzer
% -----------------------------------------------------------------------------
%
% Bring up both the max and min GUIs prior to screen capture, so that the
% window title boxes contain the project path.
%
\begin{landscape}
\begin{figure}[p]
  \begin{center}
    \includegraphics[width=215mm]
    {figures/ex1_libero_smarttime_fabric_max.png}
  \end{center}
  \caption{Libero SmartTime maximum delay with LED output registers in the fabric.}
  \label{fig:ex1_libero_smarttime_fabric_max}
\end{figure}
\end{landscape}

\begin{landscape}
\begin{figure}[p]
  \begin{center}
    \includegraphics[width=215mm]
    {figures/ex1_libero_smarttime_fabric_min.png}
  \end{center}
  \caption{Libero SmartTime minimum delay with LED output registers in the fabric.}
  \label{fig:ex1_libero_smarttime_fabric_min}
\end{figure}
\end{landscape}

\clearpage
% -----------------------------------------------------------------------------
% Libero GUI SmartTime Timing Analyzer
% -----------------------------------------------------------------------------
%
% Bring up both the max and min GUIs prior to screen capture, so that the
% window title boxes contain the project path.
%
\begin{landscape}
\begin{figure}[p]
  \begin{center}
    \includegraphics[width=215mm]
    {figures/ex1_libero_smarttime_ioff_max.png}
  \end{center}
  \caption{Libero SmartTime maximum delay with LED output registers in the I/O.}
  \label{fig:ex1_libero_smarttime_ioff_max}
\end{figure}
\end{landscape}

\begin{landscape}
\begin{figure}[p]
  \begin{center}
    \includegraphics[width=215mm]
    {figures/ex1_libero_smarttime_ioff_min.png}
  \end{center}
  \caption{Libero SmartTime minimum delay with LED output registers in the I/O.}
  \label{fig:ex1_libero_smarttime_ioff_min}
\end{figure}
\end{landscape}

\clearpage
% -----------------------------------------------------------------------------
% Libero GUI SmartTime Timing Analyzer Path Delays
% -----------------------------------------------------------------------------
%
\begin{landscape}
\begin{figure}[p]
  \begin{center}
    \includegraphics[width=215mm]
    {figures/ex1_libero_smarttime_fabric_path.png}\\
    (a) Fabric registers\\
    \vskip5mm
    \includegraphics[width=215mm]
    {figures/ex1_libero_smarttime_ioff_path.png}\\
    (b) I/O registers
  \end{center}
  \caption{Libero SmartTime timing path for with LED output registers in the fabric vs I/O.}
  \label{fig:ex1_libero_smarttime_path}
\end{figure}
\end{landscape}

% -----------------------------------------------------------------------------
% Libero GUI ChipPlanner Views
% -----------------------------------------------------------------------------
%
\begin{figure}[p]
  \begin{center}
    \includegraphics[width=0.73\textwidth]
    {figures/ex1_libero_chipplanner_fabric.png}\\
    (a) Fabric registers\\
    \vskip5mm
    \includegraphics[width=0.73\textwidth]
    {figures/ex1_libero_chipplanner_ioff.png}\\
    (b) I/O registers
  \end{center}
  \caption{Libero ChipPlanner view with LED output registers in the fabric vs I/O.}
  \label{fig:ex1_libero_chipplanner}
\end{figure}

\clearpage
% -----------------------------------------------------------------------------
\subsection{Floorplan Constraints}
% -----------------------------------------------------------------------------

The Libero ChipPlanner view of a design can be accessed via:
%
\begin{enumerate}
\item Select the \emph{Design Flow} tab.
\item Select the \emph{Manage Constraints} step.
\item Right-mouse-click and select \emph{Open Constraints Manager View}.
\item The \emph{Constraints Manager} opens with the \emph{I/O Attributes}
tab selected.
\item Click on the \emph{Floor Planner} tab.
\item Click on the \emph{View} button.
\item The ChipPlanner tool is then launched as a separate application.
\end{enumerate}
%
The ChipPlanner view shows the PolarFire SoC die, the logic within the die,
and the logic in the I/O banks.
%
Figure~\ref{fig:ex1_libero_chipplanner} shows the ChipPlanner view of
the LED design: the logic in the fabric is mainly the counter used to
blink the LEDs.

Floorplan constraints are user-defined constraints that lock logic at a
specific physical location on the die. Components that should be considered
for explicit floorplanning are clock buffers, reset buffers, and reset
synchronizers. Reset synchronizers are an interesting case, as the Libero
tool does not have any unique constraint for identifying registers that
should be placed tightly together. Designs that include synchronizer
components should also include floorplanning constraints and timing
constraints to ensure the synchronizers are implemented correctly.

The reset synchronizer registers placement was investigated as follows;
%
\begin{enumerate}
\item \textbf{No additional constraints}

Figure~\ref{fig:ex1_libero_chipplanner_reset}(a) shows the
ChipPlanner view of the synchronizer registers, with the synchronizer
registers highlighted in white:
the four registers were not placed in the same row of registers.

\item \textbf{PDC placement constraints}

Floorplan constraints were used to place the four synchronizer registers
into the a row of registers located close to the global clock and reset
buffers that are located near the center bottom of the die.
Figure~\ref{fig:ex1_libero_chipplanner_reset}(c) shows the
ChipPlanner view of the registers, with the synchronizer registers
highlighted in white: the four registers are placed in a row of
registers just above the global reset buffer.

\item \textbf{PDC placement plus SDC delay constraints}

SDC delay constraints were added and adjusted until the slack for
the minimum and maximum was under 10ps.

\item \textbf{SDC delay constraints}

The PDC constraints were removed from the project (unchecked in the
constraints manager view) to determine if tight SDC constraints were
sufficient to place the registers in the same row.

Figure~\ref{fig:ex1_libero_chipplanner_reset}(b) shows the
SDC constraints did result in the registers being placed in the same row,
however, the figure shows that the synchronizer registers were not packed
tightly together. The figure also shows that the synchronizer was placed
by the counter, rather than the desired location, by the global reset buffer.

\end{enumerate}
%
This test demonstrated that both PDC and SDC constraints should be used
for synchronizer placement. The PDC constraints ensure the registers are
placed where the user wants them, while the SDC constraints are
\emph{insurance} that checks the register placement is as the user intended.

\clearpage
% -----------------------------------------------------------------------------
% Libero GUI ChipPlanner view of reset synchronizer
% -----------------------------------------------------------------------------
%
\begin{figure}[p]
  \begin{center}
    \includegraphics[width=\textwidth]
    {figures/ex1_libero_chipplanner_reset_a.png}\\
    (a) No SDC or floorplan constraints\\
    \vskip5mm
    \includegraphics[width=\textwidth]
    {figures/ex1_libero_chipplanner_reset_b.png}\\
    (b) With SDC and without floorplan constraints
    \vskip5mm
    \includegraphics[width=\textwidth]
    {figures/ex1_libero_chipplanner_reset_c.png}\\
    (c) With SDC and floorplan constraints
  \end{center}
  \caption{Libero ChipPlanner view of the reset synchronizer registers.}
  \label{fig:ex1_libero_chipplanner_reset}
\end{figure}

\clearpage
% -----------------------------------------------------------------------------
% Synplify Elite Hierarchy
% -----------------------------------------------------------------------------
%
\begin{figure}[t]
  \begin{center}
    \includegraphics[width=\textwidth]
    {figures/ex1_synplify_elite_hierarchy.png}
  \end{center}
  \caption{Synplify Elite \emph{Hierarchical Area Report}.}
  \label{fig:ex1_synplify_elite_hierarchy}
\end{figure}

% -----------------------------------------------------------------------------
\subsection{Triple Modular Redundancy (TMR)}
% -----------------------------------------------------------------------------
\label{sec:ex1_tmr}

Synopsys Synplify Elite can be used to implement various TMR schemes,
eg., Local TMR (LTMR), Distributed TMR (DTMR), and Block TMR (BTMR).
%
The following process was used to triplicate the blinky LED component
within the Discovery Kit design:
%
\begin{enumerate}
\item In the Libero GUI, select the \emph{Design Flow} tab.
\item Use the \emph{Constraints Manager}, \emph{Floor Planner} tab, to
link the floorplan constraints file\newline
\verb+$TUTORIAL/designs/fabric_blinky/scripts/pfs_disco_fp.pdc+,\newline %$
check the \emph{Place and Route} checkbox,
and click the \emph{Save} button.

Review the FDC file for the floorplan options.
\item Use the \emph{Constraints Manager}, \emph{Netlist Attributes} tab, to
link the TMR constraints file\newline
\verb+$TUTORIAL/designs/fabric_blinky/scripts/pfs_disco.fdc+,\newline %$
check the \emph{Synthesis} checkbox,
and click the \emph{Save} button.

Review the FDC file for the DTMR constraint syntax.
\item Select the \emph{Design Flow} \emph{Synthesize} step, right-mouse-click and select
\emph{Clean}.

This clears the synthesis results created by \emph{Synplify Base}.

\item Use \emph{Project$\rightarrow$Tool Profiles},
then \emph{Synthesis}, and select the \verb+Synplify Elite+ tool profile.
\item Select the \emph{Synthesize} step, right-mouse-click and select
\emph{Open Interactively}.

\item (Optional) Delete the (empty) FDC constraint file:
\verb+synthesis.fdc+%$
\item Click the \emph{Run} button and wait for synthesis to complete.
\item In the \emph{Project Status} window, click the \emph{Hierarchical Area Report} link.

Figure~\ref{fig:ex1_synplify_elite_hierarchy} shows the report with the
triplicated \verb+blinky_31s_7s+ hierarchy expanded.

\item In the Libero GUI, the \emph{Synthesize} step will now be green.

\item Exit \emph{Synplify Elite}.

\item In the Libero GUI, click the green play button to run place-and-route
for the triplicated design.

\item Open \emph{ChipPlanner} to view the TMR instance placement.
\end{enumerate}

Figure~\ref{fig:ex1_libero_chipplanner_tmr} shows the ChipPlanner view
with blinky instance TMR1 selected. The instance was selected using the
\emph{ChipPlanner} \emph{Logical} tab, by expanding the \verb+u5+ component
to see the three TMR instances, and then selecting instance TMR1.
Figure~\ref{fig:ex1_libero_chipplanner_tmr} shows that logic for the
three TMR instances is intermingled on the die.

The triplicated logic can be floorplanned to be placed into unique (mutually
exclusive) regions.
%
Figure~\ref{fig:ex1_libero_chipplanner_tmr_fp}(a) shows the ChipPlanner
view after three unique regions were defined (see the floorplan PDC file) and
the three blinky TMR instances assigned to a unique region.
%
Figure~\ref{fig:ex1_libero_chipplanner_tmr_fp}(b) shows the ChipPlanner
view for a floorplanned non-TMR design.
%
The floorplan regions in
Figure~\ref{fig:ex1_libero_chipplanner_tmr_fp} were created in the
same location on the die as the placement in
Figure~\ref{fig:ex1_libero_chipplanner_tmr}, to help show how
floorplanning can \emph{clean up} the placement of the TMR instances.
%
The Chipplanner views in
Figures~\ref{fig:ex1_libero_chipplanner_tmr}
and~\ref{fig:ex1_libero_chipplanner_tmr_fp} can be reproduced
by editing the floorplan FDC, the NDC (to place the LED output registers),
and the SDC (to configure the timing constraints).

% -----------------------------------------------------------------------------
% ChipPlanner view of DTMR without floorplanning
% -----------------------------------------------------------------------------
%
\begin{landscape}
\begin{figure}[p]
  \begin{center}
    \includegraphics[width=210mm]
    {figures/ex1_libero_chipplanner_tmr.png}
  \end{center}
  \caption{Libero ChipPlanner view of the TMR design without floorplanning.}
  \label{fig:ex1_libero_chipplanner_tmr}
\end{figure}
\end{landscape}

% -----------------------------------------------------------------------------
% ChipPlanner view of No TMR vs DTMR with floorplanning
% -----------------------------------------------------------------------------
%
\begin{figure}[p]
  \begin{center}
    \includegraphics[width=0.73\textwidth]
    {figures/ex1_libero_chipplanner_tmr_fp.png}\\
    (a) TMR\\
    \vskip5mm
    \includegraphics[width=0.73\textwidth]
    {figures/ex1_libero_chipplanner_no_tmr_fp.png}\\
    (b) No TMR
  \end{center}
  \caption{Libero ChipPlanner view of the TMR and non-TMR design with floorplanning.}
  \label{fig:ex1_libero_chipplanner_tmr_fp}
\end{figure}

\clearpage
% -----------------------------------------------------------------------------
\subsection{Libero Scripted Flow}
% -----------------------------------------------------------------------------

Table~\ref{tab:pfs_disco_variants} shows a subset of the Discovery Kit blinky
LED design variants that have been analyzed in this tutorial, along with
additional variants that are used to investigate the clock-to-output
delay variation with drive strength and load capacitance.
%
It is important to be able to reproduce the results of each variant,
and that is where scripting helps.

The requirements for the Libero scripted flow are:
%
\begin{enumerate}
\item Customize the RTL parameters.
\item Customize the SDC timing constraints.
\item Customize the PDC pin constraints.
\item Customize the PDC floorplan constraints.
\item Customize the NDC netlist constraints.
\item Customize the FDC netlist constraints (for TMR).
\item Use Synplify Base or Elite (for TMR).
\item Generate custom timing reports.
\end{enumerate}
%
The Discovery Kit variant scripts support all of these requirements.
The variant constraints files each differ with minor  parameter changes,
so template files are used, and the variant scripts update the variant
parameters, and create the customized constraint files in the build area.

The Libero scripted flow is implemented using the following design scripts:
%
\begin{itemize}
\item \verb+variant.tcl+ Design variant procedures
\item \verb+libero_variants.tcl+ Libero script that generates all variants
\item \verb+timing_report.tcl+ LED output timing report
\end{itemize}
%
The design variants are built as follows:
%
\begin{enumerate}
\item Start \emph{Libero SoC 2025.1}
\item Select \emph{Project$\rightarrow$Execute Script}
\item Use the \emph{Script file:} browse button to select\newline
\verb+$TUTORIAL/designs/fabric_blinky/scripts/libero_variants.tcl+%$
\item Click \emph{Run}
\end{enumerate}
%
The script generates Libero projects for all of the design variants in
Table~\ref{tab:pfs_disco_variants}, runs synthesis, place-and-route,
and generates timing reports (total time for all projects is about 35 minutes).

\clearpage
The timing analysis results are reproducible. A reference copy of the
timing reports are located in
\verb+$TUTORIAL/designs/fabric_blinky/timing_reports+ %$
along with a bash script that can be used to compare (diff) the
reference timing reports to those in the build area.
%
For example, under Windows 10 using a WSL2 bash terminal, the timing
reports in the build area are checked as follows:
%
\begin{verbatim}
$ export TUTORIAL=/mnt/c/github/microchip_pfs_disco
$ cd $TUTORIAL/designs/fabric_blinky/timing_reports
$ ./compare_to_build.sh
Variant 1: PASS
Variant 2: PASS
Variant 3: PASS
Variant 4: PASS
Variant 5: PASS
Variant 6: PASS
Variant 7: PASS
Variant 8: PASS
Variant 9: PASS
Variant 10: PASS
Variant 11: PASS
Variant 12: PASS
\end{verbatim}
%
The clock-to-output delays in Table~\ref{tab:pfs_disco_variants} were
obtained from the timing report files. The timing report clock-to-output
delays match the SmartTime reported values, but reviewing text files is
faster than repeatedly opening a Libero project, then SmartTime, and then
reviewing the SmartTime clock-to-output delay for maximum (default SmartTime
window) and minimum (another SmartTime window).

% -----------------------------------------------------------------------------
% Design Variants
% -----------------------------------------------------------------------------
%
\begin{table}[p]
\caption{Discovery Kit design variants}
\label{tab:pfs_disco_variants}
\begin{center}
\begin{tabular}{|c||c|c|c||c|c||c|c|}
\hline
Variant & \multicolumn{3}{c||}{Output} & Floorplan & TMR &
          \multicolumn{2}{c|}{Clock-to-Output}\\
\cline{2-4}
\cline{7-8}
        & Registers & Drive & Load & Region & & Min & Max\\
\hline\hline
&&&&&&\phantom{XXXXX}&\phantom{XXXXX}\\
 1 &   Fabric &  4 &  5 & No  & No  & 3.769 & 6.605\\
 2 &   I/O    &  4 &  5 & No  & No  & 3.587 & 6.010\\
&&&&&&&\\
 3 &   Fabric &  4 &  5 & Yes & No  & 3.769 & 6.605\\
 4 &   I/O    &  4 &  5 & Yes & No  & 3.587 & 6.010\\
&&&&&&&\\
 5 &   Fabric &  4 &  5 & Yes & Yes & 3.769 & 6.605\\
 6 &   I/O    &  4 &  5 & Yes & Yes & 3.587 & 6.010\\
&&&&&&&\\
 7 &   Fabric &  8 &  5 & Yes & No  & 3.571 & 6.397\\
 8 &   I/O    &  8 &  5 & Yes & No  & 3.373 & 5.802\\
 9 &   Fabric & 12 &  5 & Yes & No  & 3.454 & 6.271\\
10 &   I/O    & 12 &  5 & Yes & No  & 3.254 & 5.676\\
&&&&&&&\\
11 &   Fabric &  4 & 10 & Yes & No  & 4.138 & 6.972\\
12 &   I/O    &  4 & 10 & Yes & No  & 3.955 & 6.377\\
13 &   Fabric &  4 & 20 & Yes & No  & 4.877 & 7.706\\
14 &   I/O    &  4 & 20 & Yes & No  & 4.678 & 7.111\\
&&&&&&&\\
\hline
\end{tabular}
\end{center}
\end{table}
