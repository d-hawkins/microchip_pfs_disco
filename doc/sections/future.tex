% =============================================================================
\section{Future Designs}
% =============================================================================
\label{sec:future}

The following are a list of ideas for future designs:
%
\begin{enumerate}
\item \textbf{Boot Linux}

Boot Linux.

\item \textbf{Boot vectors}

Write assembly code for the boot vector to blink the LEDs.

\item \textbf{HSS Payload}

Use the HSS to boot a user application, eg., as a first-stage bootloader.

\item \textbf{FIC AXI4 and AHB BFMs}

Demonstrate the use of the simulation BFMs, eg., start with the reference
hardware design, generate the BFM\_SIMULATION version, and blink the LEDs.

\item \textbf{AN5165: FIR Filter design}

Create a bit-true model of the digital filter and the FFT logic.
Use the simulation to understand the UART protocol.
Reimplement the simulation using CocoTB.

For the bit-true model, investigate the bit depths used in the design:
%
\begin{itemize}
\item 16-bits data times 16-bits coeffs = 32-bit products
\item 127 programmable coefficients have log2(127) = 7-bits growth
\item FIR filter worst-case output bit width is 32+7 = 39-bits
\item The FIR code show the output connects to FIRO[29:14], so that is
only 16-bits, but it is not the worst-case 16-bits.
\item Will this FIR saturate or wrap?
\item Is the FIR GUI scaling the coefficients and input signal to
accommodate the known bit-widths? (Review the text input and output view)
\end{itemize}

\item \textbf{FIR Filter design with RISC-V interface}

Reimplement the FIR filter design:
\begin{itemize}
\item AXI4-Stream interface FIR filter
\item AXI4-Stream interface FFT
\item AXI4-Stream FIFOs connecting components
\item Memory mapped FIFOs for input and output data
\item AXI4 or AHB control registers interface
\item DMA controller for input and output buffer loading and unloading
\end{itemize}
%
Implement the AXI4 or AHB control interface using the MSS BFMs.
Also implement the control using a UART-to-AXI4 or UART-to-AHB bridge.
That way this design could also work on the FPGA only kits.

\item \textbf{Heater}

Does the board have current sensors? How about the Video Kit?

Buy a USB-C interface/bridge that measures current. I would like the USB power
monitor to have a USB-C interface and I would like to be able to read the
measurements over USB (not just an LCD screen on the unit), so that I can
plot a power time-series.

\end{enumerate}

